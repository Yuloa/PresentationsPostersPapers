\section{Conclusion}
In Run~II the new scheme for the software trigger at LHCb allows the
alignment and calibration to be performed in real time. A dedicated framework
has been put in place to parallelise the alignment and calibration tasks
on the multi-core farm infrastructure used for the trigger in order to meet the
computing time constraints. Data collected
at the start of the fill are processed in a few minutes and the output is used to
update the alignment, while the RICH calibration constants are evaluated for
each run. The same framework is used to perform  finer calibration less
frequently and to monitor the alignment quality of various subdetector. 
This procedure allows a
more stable alignment quality, more effective trigger selections and
online-offline consistency thanks also to the same online-offline reconstruction.
Physics analysis can be performed directly on the trigger output with the same online-offline performance.

