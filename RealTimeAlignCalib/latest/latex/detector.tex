\section{The LHCb detector}
\label{sec:Detector}
\textcolor{red}{It is needed a more detailed detector description to be able
to understand all the different alignment and calibration task description. To be
see for example if in the detector peformance paper we have a better or more 
complete description.}

The \lhcb detector~\cite{Alves:2008zz,LHCb-DP-2014-002} is a single-arm forward
spectrometer covering the \mbox{pseudorapidity} range $2<\eta <5$,
designed for the study of particles containing \bquark or \cquark
quarks. The detector coordinate system is such that \textbf{z} is along the beamline and
\textbf{x} is along the horizontal bending plane. The detector includes a high-precision tracking system
consisting of a silicon-strip vertex detector (\velo) surrounding the $pp$
interaction region~\cite{LHCb-DP-2014-001}\verb!*!, a large-area silicon-strip detector located
upstream of a dipole magnet with a bending power of about
$4{\rm\,Tm}$, and three stations of silicon-strip detectors (\st) and straw
drift tubes~\cite{LHCb-DP-2013-003}\verb!*! (\ot) placed downstream of the magnet.
The tracking system provides a measurement of momentum, \ptot, of charged particles with
a relative uncertainty that varies from 0.5\% at low momentum to 1.0\% at 200\gevc.
The minimum distance of a track to a proton-proton collision (primary vertex),
the impact parameter, is measured with a resolution of $(15+29/\pt)\mum$,
where \pt is the component of the momentum transverse to the beam, in\,\gevc.
Different types of charged hadrons are distinguished using information
from two ring-imaging Cherenkov detectors~\cite{LHCb-DP-2012-003}\verb!*!. 
Photons, electrons and hadrons are identified by a calorimeter system consisting of
scintillating-pad (\spd) and preshower detectors, an electromagnetic
calorimeter and a hadronic calorimeter. Muons are identified by a
system composed of alternating layers of iron and multiwire
proportional chambers~\cite{LHCb-DP-2012-002}\verb!*!.
%The online event selection is performed by a trigger~\cite{LHCb-DP-2012-004}\verb!*!, 
%which consists of a hardware stage, based on information from the calorimeter and muon
%systems, followed by a software stage, which applies a full event
%reconstruction.
The \lhcb detector datataking is divided into fills and runs. A fill is a single period
of proton-proton collisions delimited by the announcement of stable beam conditions and
the dumping of the beam by the \lhc. A fill is subdivided into runs, each of which lasts
a maximum of one hour.

Detector simulation has been used in the tuning of most reconstruction
and selection algorithms discussed in this paper. In simulated LHCb events,
$pp$ collisions are generated using \pythia~\cite{Sjostrand:2006za,*Sjostrand:2007gs}
with a specific \lhcb configuration~\cite{LHCb-PROC-2010-056}.  Decays of hadronic particles
are described by \evtgen~\cite{Lange:2001uf}, in which final-state
radiation is generated using \photos~\cite{Golonka:2005pn}. The
interaction of the generated particles with the detector, and its response,
are implemented using the \geant
toolkit~\cite{Allison:2006ve, *Agostinelli:2002hh} as described in
Ref.~\cite{LHCb-PROC-2011-006}.


