\section{Alignment and Calibration Tasks}

\subsection{Tracking Alignment}
\subsubsection{Alignment method}
\textcolor{red}{give more details}

The tracking alignment is based on an iterative procedure where the
residuals of a Kalman track fit are minimised.  The magnetic field and
material effects are taken into account and information from vertices
and particle masses can also be used as constraints to avoid global
distortions \cite{Hulsbergen,AlignConstraints}. 

Given a set of tracks reconstructed using the alignment parameters
$\alpha_0$, the new set of alignment constants can be found solving
the system of equations
\begin{equation}
\alpha = \alpha_0 -\left.\left(\frac{d^2 \chi^2}{d\alpha^2}
\right)^{-1} \right|_{\alpha_0}  \left.\frac{d
  \chi^2}{d\alpha}\right|_{\alpha_0},
\label{eq:minimize}
\end{equation}
where the derivatives of the total $\chi^2$ with respect to the
alignment parameters are obtained by summing the contributions from
all the tracks:
\begin{equation}
\begin{aligned}
\frac{d \chi^2}{d\alpha} &= 2 \sum_{\mbox{tracks}}
\frac{dr}{d\alpha}^T V^{-1} r ,\\ \frac{d^2 \chi^2}{d\alpha^2} &= 2
\sum_{\mbox{tracks}} \frac{dr}{d\alpha}^T V^{-1} R V^{-1}
\frac{dr}{d\alpha}.\\
\end{aligned}
\label{eq:derivatives}
\end{equation}
Here $V$ is the covariance matrix of the measurement coordinates, $r$
is the track residuals (the distance between the hit position and the
track intercept point), and $R$ is the covariance matrix of the
residuals. It is assumed that the $\chi^2$ has been minimised with
respect to the track parameters for the alignment $\alpha_0$.

\subsubsection{VELO Alignment}
\textcolor{red}{ it should include (random order):stability plots, variation of the constants, time needed to run the jobs, event selections. Performance before/aftera alignment. Trend of the constants and evaluation of the variation. Maybe trends of physics quantity like mass or time resolution}

\subsubsection{Tracker Alignment}

\subsubsection{Muon Alignment}

