\section{Propagation of Constants}

\begin{table}
\begin{tabular}{l l}
  Task                  & Used by                 \\ \hline
  Velo alignment        & HLT1, HLT2 and offline  \\ 
  tracker alignment     & HLT1, HLT2 and offline  \\
  muon alignment        & L0 Muon and monitoring  \\
  ECAL calibration      & update of ECAL high voltages \\
  RICH mirror alignment & HLT2 and offline        \\
  RICH calibrations     & HLT2 and offline        \\
  OT calibration        & HLT1, HLT2 and offline
\end{tabular}
\caption{Processing phases or detector systems that require the result of
  calibration and alignment tasks.}
\label{tab:results_needed}
\end{table}

Once a calibration or alignment task has completed, it writes the constants it
produced to one or more XML file. To allow propagation of these constants to the
data processing tasks, such tasks have been configured to read relevant
calibration and alignment constants from these XML files. Whenever a task
receives an event that belongs to a new data taking period, a new set of XML
files is read to obtain the latest constants. 

\Cref{tab:results_needed} gives on overview of where constants are used. The
different phases where constants are needed broadly fall into three categories:
HLT1, HLT2 and offline; HLT2 and offline; and detector subsystems. If constants
are used in HLT1, HLT2 and offline, they are used across all of these phases
starting from a fixed point in time after the result has been obtained. This
usually means the start of the next data-taking period of up to an hour. If
constants are not needed in HLT1, but only by HLT2 and offline, HLT2 will not
start processing data from a given data-taking period until the constants for
that data-taking period have been produced. Results of tasks used only by
subsystems are interpreted by subsystem experts and used to make updates to the
configuration of those system if and when required.

If constants are needed for offline processing, they are added to the
conditions database as soon as all constants for a given data taking period are
available. Once the constants have been added to the conditions database, a
cross-check is performed by configuring one task to read the constants from the
XML files, another task to read them from the conditions database, and comparing
the constants they obtain. If any mismatches are detected, the corresponding
data taking period is not declared ready for offline processing. Experts then
investigate the source of the mismatches and resolve the problem.
