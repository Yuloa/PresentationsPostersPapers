\section{Introduction}
\textcolor{red}{For me we should have a different introduction
1) PID/Align performance needed to get optimal performance
2) Possibility to have in trigger together with a new strategy
=> imply better ferformance RICH PID}

The \lhcb experiment is a single-arm forward spectrometer at the \lhc.
It has been designed as a dedicated heavy flavour physics experiment
focused on the reconstruction of \Pc~and~\Pb~hadrons. During Run~1 its
physics programme  has been extended to electroweak, soft QCD and even
heavy ion physics.  This was made possible in large part due to its
versatile real-time reconstruction and analysis (trigger), which is
responsible for reducing the rate of proton-proton collisions which
need to be saved for offline analysis by approximately three orders of
magnitude. LHCb's Run~1 trigger executed a simplified version of the
full offline event reconstruction. Most charged particles above
300~MeV of transverse momentum were available to classify the event,
and particle identification and the use of neutral particles such as
photons or \piz mesons was available on-demand to specific
classification algorithms. Although this trigger enabled a great deal
of LHCb's physics programme, the lack of very low momentum charged
particles and full particle identification information limited the
performance for \Pc~hadron physics in particular. In addition,
resolution differences between the online and offline reconstructions
led to difficulties in understanding efficiencies with a high degree
of precision.

For these reasons, the LHCb trigger system was redesigned during the
2013-2015 long shutdown to perform the full offline event
reconstruction, and the entire data processing framework was
redesigned to enable a single, coherent, real-time detector alignment
and calibration, as well as real-time analyses using information
directly from the trigger system. The key objectives of this redesign
were twofold : firstly to enable the full offline reconstruction to
run in the trigger, greatly increasing the efficiency with which charm
and strange hadron decays could be selected, and secondly to achieve
the same alignment and calibration quality within the trigger as could
be achieved offline in Run~I, enabling the entire analysis to be
performed at the trigger level. 

In order to achieve this objective, the alignment and calibration of
all the different components of the \lhcb detector have been fully
automated. Specific data streams are defined which supply each
alignment and calibration algorithm with the events it needs, the
algorithms themselves have been parallelized to enable them to run
online within, for the most part, a matter of minutes, and a
monitoring framework has been designed to ensure that any problems
with alignment and calibration are spotted and rectified before they
can affect data quality. 

This paper describes the novel real-time alignment and calibration of
the \lhcb detector, to the best of our knowledge the first of its kind
in High Energy Physics.  Two companion papers describe the design and
performance of the new uniform online-offline reconstruction
sequence~\cite{} and the real-time analysis~\cite{} framework, which
are made possible by the real-time alignment and calibration.
