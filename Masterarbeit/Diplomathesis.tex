\documentclass{wissdoc}
% AutorInnen: Stefan Fliescher 2008, Anna Nelles 2010
% ----------------------------------------------------------------
% Diplomathesis
% ----------------------------------------------------------------
%%
%%
%%
% wissdoc Optionen: draft, relaxed, pdf --> siehe wissdoc.cls
% ------------------------------------------------------------------
% Weitere packages: (Dokumentation dazu durch "latex <package>.dtx")
%\usepackage{varioref}
%\usepackage{verbatim}
%\usepackage{float}    %z.B. \floatstyle{ruled}\restylefloat{figure}
\usepackage{subfigure}
%\usepackage{color}    % Farbiger/grauer Text
%\usepackage{colortbl}   % Farbige/graue Tabellenzeilen und -spalten!! <--
%\usepackage{fancybox} % fuer schattierte,ovale Boxen etc.
%\usepackage{tabularx} % automatische Spaltenbreite
%\usepackage{supertab} % mehrseitige Tabellen
%% -------------to find the position of bremsstrahlung photon candidates in--- end of usepackages -------------
\usepackage{latexsym}
\usepackage{amsfonts}
\usepackage{amsmath}
\usepackage{amssymb}
\usepackage{bm}
\usepackage{graphicx}
\usepackage{lscape}
\usepackage{url}
\usepackage{xspace}
\usepackage{listings}
\usepackage{floatflt}
\usepackage{sidecap}
%\usepackage{subfig}\captionsetup{labelfont=bf}
\usepackage[bf]{caption}%\captionsetup{captionlabel=it}
%\usepackage{subcaption}
\usepackage{xcolor}
\definecolor{cred1}{HTML}{CA0614}
\definecolor{cred2}{HTML}{FD3232}
%
%\def\Offline{\mbox{$\overline{\rm
%Off}$\hspace{.05em}\raisebox{.3ex}{$\underline{\rm line}$}}\xspace}
\def\OfflineB{\mbox{$\bf\overline{\rm\bf Off}$\hspace{.05em}
\raisebox{.2ex}{$\bf\underline{\rm\bf line}$}}\xspace}
\newcommand{\HRule}{\rule{\linewidth}{1mm}}

\def\emptyline{\vspace{12pt}}
%\renewcommand{\floatpagefraction}{.6}% vorher: .5
%\renewcommand{\textfraction}{.15} % vorher: .2
%% Informationen f�r die PDF-Datei
% \pdfinfo{
%  /Title  (Diplomathesis.pdf)
%  /Author (Stefan Fliescher)
%  /Subject (Diplomathesis)
%  /Keywords(Auger, Radio, Cosmic Rays)
% }

% Macros, nicht unbedingt notwendig
\input{macros}
% %%%%%%%%%%%%%%%%%%%%
%  for LHCb aliases
% %%%%%%%%%%%%%%%%%%%%
\usepackage{ifthen}
\newboolean{articletitles}
\setboolean{articletitles}{true} % False removes titles in references
\newboolean{uprightparticles}
\setboolean{uprightparticles}{false} %Set to true to get roman particle symbols
\usepackage{amssymb}
\usepackage{amsfonts}
\usepackage{paralist}
\usepackage{wrapfig}
\usepackage{upgreek} % Adds in support for greek letters in roman typeset
% Get hyperlinks to captions and in references.
% These do not work with revtex. Use "hypertext" as class option instead.
\usepackage{hyperref}    % Hyperlinks in references
\usepackage[all]{hypcap} % Internal hyperlinks to floats.
\input{lhcb-symbols-def} % Add in the predefined LHCb symbols


% Print URLs not in Typewriter Font
\def\UrlFont{\rm}
\newcommand{\vc}[1]{\mbox{\boldmath$ #1 $}}
\newcommand{\blankpage}{% Leerseite ohne Seitennummer, n�chste Seite rechts
\clearpage{\pagestyle{empty}\cleardoublepage}
}
\newcommand{\entspricht}{\mathrel{\widehat{=}}}

\def\Offline{\mbox{$\overline{\textrm%
{Off}}$\hspace{.05em}\protect\raisebox{.4ex}%
{$\protect\underline{\textrm{line}}$}}\xspace}

%% Einstellungen fuer das gesamte Dokument

% Trennhilfen
% Wichtig!
% Im german-paket sind zusaetzlich folgende Trennhinweise enthalten:
% "- = zusaetzliche Trennstelle
% "| = Vermeidung von Ligaturen und moegliche Trennung (bsp: Schaf"|fell)
% "~ = Bindestrich an dem keine Trennung erlaubt ist (bsp: bergauf und "~ab)
% "= = Bindestrich bei dem Worte vor und dahinter getrennt werden duerfen
% "" = Trennstelle ohne Erzeugung eines Trennstrichs (bsp: und/""oder)

% Trennhinweise fuer Woerter hier beschreiben
\hyphenation{
ma-the-ma-ti-cal
ex-pec-tan-cy
me-thods
pro-ducts
}

% Index-Datei oeffnen
% \ifnotdraft{\makeindex}
%%%%%%%%%%%%%% includeonly %%%%%%%%%%%%%%%%%%%
% Es werden nur die Teile eingebunden, die hier
% aufgefuehrt sind!
% \includeonly{
% titelseite,
% Motivation,
% % CPVerletzung,
% % Standardmodell,
% % Neutralebmesonen,
% % Experimente,
% % Literatur
% }
%%%%%%%%%%%%%%%%%%%%%%%%%%%%%%%%%%%%%%%%%%%%%%
\begin{document}
\lstset{language=C++, basicstyle =\ttfamily}
\selectlanguage{british}
%% Titelseite
%% Vorlage $Id: titelseite.tex,v 1.3 2002/09/09 11:30:42 bless Exp $
\titlehead{%\large\resizebox{!}{0.5cm}{%
%\includegraphics*[width=2cm]{logos/logo_schwarz_rechts.jpg}}~\raise 4.5mm%
%\hbox{ }\hfill%
} % end titlehead
\titlefoot{%
%\hfill\raise4mm\hbox{}\ \resizebox{!}{1cm}{%
%\includegraphics*{logos/logo_blau}
% }%
}

\newsavebox{\Prof}
\savebox{\Prof}{Prof.~Dr.~}

\begin{titlepage}
%\let\footnotesize\small \let\footnoterule\relax
\begin{center}
\hbox{} %\vfill 
\vspace*{2.5cm}
{\Huge\bfseries Model-independent measurement of the CKM angle $\gamma $ through \BToDK decays with \lhcb and CLEO-c \par}
\vskip 1.5cm
\large{ \textbf{Claire Prouve}}\normalsize\\
\vspace{2.cm}
\large{\textbf{Third year report}}\normalsize\\
\vspace*{1.5cm}
Presented to the \\
School of Physics\\
of the \\
University of Bristol\\

\selectlanguage{british}
\vskip 1.cm \textbf{29 May 2016}\\%\today\\
\vspace*{0.5cm}
\end{center}
%\vfill
\vspace*{1.3cm}
%\includegraphics[width=\textwidth]{logos.jpg}
\end{titlepage}
%% Titelseite Ende

\blankpage % Leerseite auf Titelrueckseite

\begin{minipage}[c]{0.49\textwidth}
\vskip 20cm
Assessor 

\vspace{0.7cm}
Prof. Dr. Thomas Hebbeker

III. Physikalisches Institut A

RWTH Aachen
\end{minipage}
\begin{minipage}[c]{0.49\textwidth}
\vskip 20cm
Secondary assessor and supervisor

\vspace{0.7cm}
Prof. Dr. Marie-H\'{e}l\`{e}ne Schune

Laboratoire de l'Acc\'{e}l\'{e}rateur Lin\'{e}aire
Universit\'{e} Paris-Sud 11
\end{minipage} 



% \blankpage % Leerseite auf Erklaerungsrueckseite
%}

%% *************** Hier geht's ab ****************
%% ++++++++++++++++++++++++++++++++++++++++++
%% Verzeichnisse
%% ++++++++++++++++++++++++++++++++++++++++++
\pagenumbering{roman}
\ifnotdraft{
% \renewcommand{\contentsname}{Contents}
\tableofcontents
% \blankpage
% \def\listoffiguresname{List of Figures}
%\renewcommand{\listfigurename}{Abbildungstafeln}
%\listoffigures
% \blankpage
% \listoftables
\blankpage
}
%% ++++++++++++++++++++++++++++++++++++++++++
%% Hauptteil
%% ++++++++++++++++++++++++++++++++++++++++++
\graphicspath{{bilder/}} \pagenumbering{arabic}
\selectlanguage{british}
\chapter{Introduction}
\chaptermark{Chapter1}
\label{Chapter0}
The Standard Model (SM) of particle physics is a very successful theory in that it explains a vast range of phenomena observed in nature. Its predictive powers exceed those of any previous theory and its mathematical structure is well defined and organised. The SM describes the fundamental aspects of matter and energy by expressing matter particles (fermions) and particles that mediate interactions (bosons) in a unified theory. The latest achievement of particle physics was the discovery of the Higgs Boson, whose existence had been predicted over forty years ago as inherent part of the SM, by \cms \cite{higgscms} and \atlas \cite{higgsatlas}.\\ 
Despite these achievements there are still questions unanswered by the SM. One open issue is the inclusion of gravity and the explanation of the hierarchy of the fundamental interactions. Another unexplained phenomenon is the discrepancy between the amount of matter and anti-matter observed in the universe. While the SM does implement the mechanism of charge-parity (CP) violation, the generated effects within the SM are not large enough to explain today's observations.\\
For this reason, many other theories -- such as the supersymmetry (SUSY) or the Left-Right Symmetric Model (LRSM) -- have been developed as extensions of the SM. These theories are designed so that their predictions agree with the SM at low energies but differ at higher energies. Usually new particles are introduced with heavier masses, changing the physics at high energy scales with respect to the SM.\\
\\
All high energy physics experiments aim at performing precision measurements of SM parameters or at discovering physics beyond the SM. The \lhc has introduced an optimal environment with an unprecedented center of mass energy and a high luminosity that provides the allocated experiments with a high amount of statistics. The experiments \atlas and \cms \ -- the so-called 'general purpose' detectors -- mainly aim at finding new particles through direct searches. The \lhcb experiment however, performs indirect searches for new physics, for example by identifying deviations from SM predictions in loop-diagram and box-diagram processes. The \lhcb experiment focusses on rare decays of \B and \D mesons and hadrons and performs measurements of CP violation.\\
The \BdKstee decay is of special interested, because it proceeds through a flavour changing neutral current (FCNC) and is therefore forbidden at tree-level in the SM. It has to proceed through loop diagrams and thus is particularly sensitive to new physics and particularly useful for testing the quantum structure of the underlying theory. Since relative contributions from new physics are greater for these processes than for tree-level processes they are more likely to be detectable through thorough analysis.\\
The main contribution to the \BdKstee comes from the transition of \bsg where the photon is virtual and decays into an electron-positron pair. While the branching ratio for this transition has been found to be consistent with SM predictions (measurement via \BdKstg \cite{pdg}), new physics could still be detectable in details of the decay such as the photon polarisation. Due to the chirality structure of the weak interaction, the photons from the \bsg transitions are dominantly left-handed. A measurement of a significant amount of right handed photons from this transitions would be an unambiguous sign of new physics beyond the SM.
The photon polarisation can be probed by performing an angular analysis of the \BdKstee decay. In this case, the photon from the \bsg is virtual and the information about its polarisation is conserved in the kinematics of the electron-positron pair.\\
\\
This master's thesis provides the fundamental basis for an angular analysis of the \BdKstee decay. In the course of this work the reconstruction of the\\ \BdKstee candidates is studied and different reconstruction algorithms are evaluated. Hereby the focus is put on the reconstruction of bremsstrahlung emitted by the electrons and the reconstruction of the invariant mass of the electron-positron pair. As a result the resolution on the \Bd mass is improved by about 27\% and the reconstruction efficiency is augmented in the region of low electron positron invariant mass.\\
Furthermore a new selection procedure for the \BdKstee events in 3\invfb of data collected by \lhcb in 2011 and 2012 is developed. Therefore a method implementing two identical Boosted Decision Trees trained on different parts of the dataset is applied. The selection of the \BdKstee events is optimised and the number of reconstructed and selected \BdKstee events is extracted by fitting a probability density function -- composed of a probability density function for the signal and the background respectively -- to the dataset. This results in a total signal yield of 130 $\pm$ 17 \BdKstee events.\\
\chapter{Theory and phenomenology of the \BdKstee decay}
\chaptermark{Chapter2}
\label{chapter1}
The \BdKstee decay proceeds via a flavour-changing neutral current (FCNC) and is therefore particularly sensitive to contributions from physics beyond the Standard Model (SM). The decay is forbidden at tree-level and will proceed through higher order diagrams. New physics can manifest in the loop and cause deviations from SM predictions.\\
This chapter starts with an introduction to the effective theories used to describe the \BdKstll decays. The first section is followed by an explication of the \bsg transition which is the main contribution to the \BdKstee decay. In the framework of the SM, the photons from the \bsg transition are predicted to be dominantly left-handed and therefore the measurement of a significant right-handed polarisation amplitude would be a sign for new physics.\\
The measurement of the photon polarisation can be performed by an angular analysis of the \BdKstll decay which is explored in Section \ref{sec:ksll}.\\
The advantages of using the electron-final state in \BdKstee at low invariant dilepton mass $q$ are illustrated at the end of this chapter.


\section{Effective Theories and effective Hamiltonian}
The transitions in the \BdKstll decay are carried out by the weak interaction (energy scale $\mu \sim O(m_b)$) while the binding of the quarks to mesons happens by the strong interaction (energy scale $\mu \sim O(1 \gev )$). Therefore the \BdKstll decay is a multi-scale process that can be described using the Effective Field Theories. Due to the strong hierarchy between external and internal scales a mathematical separation between low and high energy terms can be performed within the framework of the Operator Product Expansion (OPE) \cite{buras}.\\
The effective Hamiltonian can then be expressed in terms of effective point-like vertices which are represented by the local operators $\Opei(\mu)$  and their associated Wilson coefficients $\Ci(\mu)$ which can be regarded as coupling constants associated with these effective vertices.
The effective point-like vertices are introduced because Feynman diagrams such as those in Figures \ref{fig:bsg} and \ref{fig:feynbox} with full \W , \Z and \tquark propagators only represent the interaction at very short distance scales ($\mu \sim O(m_W, m_Z, m_t)$) while the long distance operators with scales of $\mu \sim O(m_b)$ have to be taken into account too. The resulting effective Hamiltonian can be expressed as
\vspace*{-0.3cm}
\begin{equation}
\mathcal{H}_{eff} = -\frac{4G_F}{\sqrt{2}}\Vtb \Vts^* \sum_{i=1}^{10} [\Ci(\mu)\Opei(\mu) + \Cpi(\mu)\Opepi(\mu)]\\
\vspace*{-0.4cm}
\label{eq:genhamilton}
\end{equation}
The dashed variables $\Cpi(\mu)$ and $\Opepi(\mu)$ represent the chirality-flipped Wilson coefficients and local operators respectively.\\
Since the energy scale in \B physics is typically chosen to be $\mu \sim O(m_b)$, the local operators $\Opei(\mu \sim O(m_b))$ have to be calculated using non-perturbative methods. The Wilson coefficients on the other hand, are calculated at $\mu \sim O(m_W)$ -- where perturbative techniques can be applied -- up to NNLO (Next to Next to Leading Order). Renormalisation techniques are then applied to find their values at the appropriate energy scales $\mu \sim O(m_b)$.\\

\section{The photon polarisation in the \bsg transition}
The \bsg transition is the main contribution to the \BdKstll decay at low dilepton invariant mass $q$. The corresponding Feynman diagrams are shown in Figure \ref{fig:bsg}.\\  
The process can be expressed in the effective theory by the magnetic-operators $\squarkbar_L \sigma_{\mu \nu} \bquark_R$ and $\squarkbar_R \sigma_{\mu \nu} \bquark_L$ which introduce the helicity structure $\bquark_R \rightarrow \squark_L \g_L $ and \\$\bquark_L \rightarrow \squark_R \g_R $ \cite{emi}. Due to the fact that the \W bosons only couple to fermions of left-handed chirality, the magnetic operators are weighted with the mass of the \bquark quark and \squark quark respectively $m_{\bquark} \squarkbar_L \sigma_{\mu \nu} \bquark_R$ and $m_{\squark} \squarkbar_R \sigma_{\mu \nu} \bquark_L$.\\
As a result, the second process is suppressed by the factor $m_{\squark} / m_{\bquark}$ which means that the photons from the \bsg transition are dominantly left-handed. Expressed in amplitudes of the photon polarisation this yields a suppression factor of $ \frac{A_R}{A_L} \sim \frac{m_{\squark } }{ m_{\bquark } } $ ($A_R$: right-handed polarisation amplitude, $A_L$: left-handed polarisation amplitude). Taking into account additional gluon contributions in the penguin-diagram and QCD corrections, the SM prediction for the ratio of right-handed amplitude to left-handed amplitude is $\frac{A_R}{A_L} \approx 4 \,\% $  \footnote{Assuming no \CP violation in the \bsg, the statements for the \bsg transition are also true for the \CP transformed process. This means that the photons from the \absg transition are dominantly right-handed and $ \frac{ \bar{A} _L}{ \bar{A} _R} \approx 4 \,\% $.}\cite{ananote}. Therefore a measurement of a significant right-handed amplitude is an unambiguous signal for new physics.\\
\begin{figure}[ht]
  \begin{center}
  	\subfigure{\includegraphics[width=0.45\textwidth]{feynbsg1.jpg}}
   \subfigure{\includegraphics[width=0.49\textwidth]{feynbsg2.jpg}}
  \vspace*{-0.5cm}
  \end{center}
  \caption{\textit{The Feynman diagrams for the \bsg transition.}}
  \label{fig:bsg}
\end{figure}
\newpage
\section{Angular analysis of the \BdKstee}
\label{sec:ksll}
While various ways to measure the photon polarisation have been presented by theorists \cite{ananote} \cite{gross}, the angular analysis of a decay of the form \BdKstll (with $\Kstarz \rightarrow \Kp \pim$) is one of the most promising.\\
This decay proceeds predominantly via the \bsll transition with a virtual photon decaying into two leptons. Since the photon is virtual the kinematics in the \bsll transition are the same as in the \bsg transition, particularly the helicity structure is the same.\\
In \B decays into two vectors (here the virtual photon and the \Kstarz) the subsequent decays of the vectors contain their relative polarisation information \cite{krueger}. Thus, the angular distribution in the relative angles of the $\Kstarz \rightarrow \Kp \pim$ decay plane and the plane of the lepton-pair can be used to determine the helicity amplitudes in the \BdKstll decay and therefore the polarisation amplitudes $A_R$ and ${A_L}$ of the photon.\\

\subsection{Contributions to the \BdKstll Hamiltonian}
The \BdKstll decay proceeds at low lepton-pair invariant mass predominantly via the \bsll transition but other transitions such as the diagram of Figure \ref{fig:bsg} where the photon is replaced by a \Z or the box-diagram in Figure \ref{fig:feynbox} add a non-negligible contribution to the effective Hamiltonian.
\begin{figure}[ht]
\begin{center}
  	\vspace*{-0.5cm}
    \includegraphics[width=0.48\textwidth]{feynbox.jpg}
  \vspace*{-1cm}
  \end{center}
  \caption{\textit{The box-diagram that contributes to \BdKstll.}}
  \label{fig:feynbox}
\vspace*{-0.5cm}
\end{figure}


The leading order contributions to the \BdKstll effective Hamiltonian (see Equation \ref{eq:genhamilton}) come from the operators $ \Ope7 $ , $ \Ope9 $ and $ \Ope10 $ \cite{krueger}.\\
\begin{eqnarray}
\mathcal{H}_{eff} = -\frac{4G_F }{ \sqrt{2} } \Vtb \Vts^* [& \C7 (\mu) \Ope7 (\mu) + \Cp7 (\mu) \Opep7 (\mu) & \nonumber \\
& \C9 (\mu) \Ope9 (\mu) + \Cp9 (\mu) \Opep9 (\mu) & \ + \ \C10 (\mu) \Ope10 (\mu) + \Cp10 (\mu)\Opep10 (\mu) ]  \nonumber
\label{eq:hamilton}
\end{eqnarray}
with the local operators \cite{gross}
\begin{eqnarray}
& \Ope7 = \frac{e}{16\pi^2} \squarkbar \sigma_{\mu \nu}(m_{\bquark}  P_R \ +\ m_{\squark}P_L)\bquark F^{\mu \nu} &\\
&\Ope9 = \frac{e^2}{16\pi^2} (\squarkbar \gamma_{\mu} P_L \bquark)(\bar{\electron}\gamma^{\mu}\electron) \qquad \Ope10 = \frac{e^2}{16\pi^2} (\squarkbar \gamma_{\mu} P_L \bquark)(\bar{\electron}\gamma^{\mu} \gamma_5 \electron) & \nonumber
\end{eqnarray}

The $ \Ope7 $ corresponds to the SM \bsll transition (\bsg transition) with a left-handed weak-charged current. Since there are no right-handed weak-charged currents in the SM, the $ \Cp7(\mu) $ is only non-zero for contributions from new physics. It is therefore appreciable to measure quantities connected to \Cp7 . 
\\The operators $ \Ope9 $ and $ \Ope10 $ represent the contributions to the \BdKstll decay that do not come from the \bsg transition but for example from the box-diagram in Figure \ref{fig:feynbox}.\\
The branching ratio \BR(\BdKstll) increases significantly towards low $q^2$ values
 (where $q^2$ denotes the square of the dilepton invariant mass)\cite{jaeger}. This is due to the strong increase of the \BR(\bsll) while the other contributions to \BdKstll vary much more smoothly and show no particular increase at low $q^2$. These other contributions make up for a few \% of the events with $20\mevcc < q < 1\gevcc $.\\
At \lhcb the angular analysis of \BdKstll decays is performed with muons and electrons in different bins of $q^2$. The combination of these analyses gives global information about all polarisation amplitudes over a large range of $q^2$ and the relative contributions from $\C9 (\mu)$ and $ \C10 (\mu)$ can be determined. 
\begin{figure}[ht]
  \centering
    \includegraphics[width=0.55\textwidth]{BR.jpg}
  \caption{\textit{The \BR(\BdKstee) for the low $q^2$ region. Solid (red) and dashed (green) lines correspond to the SM prediction for two different models. The error band is calculated from different theoretical uncertainties such as hadronic and CKM uncertainties and renormalization scale dependence.}\cite{jaeger}}
  \label{fig:BR}
\end{figure}

\subsection{Polarisation amplitudes for the \BdKstee}
\label{sub:polamp}
In the framework of the effective theory, the transversity polarisation amplitudes $A^{L,R}_{ \perp }$, $A^{L,R}_{ \parallel }$ and $A^{L,R}_0$ can be expressed in terms of the Wilson coefficients \cite{krueger}. The indices $L$ and $R$ refer to the chirality of the lepton current. In this calculation the leptons are presumed to be massless which is a very good approximation for the \BdKstee. In the region low $q^2$ region which corresponds to large recoil of the \Kstarz, the amplitudes can be expressed as:
\begin{equation}
A^{L,R}_{ \perp }(q^2) = \frac{ \sqrt{2} N (M^2_B - q^2) }{ M_B }[(\C9 + \Cp9 ) \mp (\C10 + \Cp10 )+\frac{ 2 m_b M_B }{ q^2 } (\C7 + \Cp7 )] \xi_{ \perp }(q^2)
\end{equation}
\begin{equation}
A^{L,R}_{ \parallel }(q^2) = \frac{ \sqrt{2} N (M^2_B - q^2) }{ M_B } [(\C9 + \Cp9 ) \mp (\C10 + \Cp10 )+ \frac{ 2 m_b M_B }{ q^2 }(\C7 - \Cp7 )] \xi_{ \perp }(q^2)
\end{equation}
\begin{equation}
A^{L,R}_0 (q^2) = \frac{ \sqrt{2} N (M^2_B - q^2) }{M_B } [(\C9 + \Cp9 ) \mp (\C10 + \Cp10 )+ \frac{2 m_b }{M_B }(\C7 - \Cp7 )] \xi_{ \parallel }(q^2)
\end{equation}
 
In the above formulae $ \lambda = M_B^4 + M_{\Kstarz}^4 + q^2 - 2(M_B^2 M_{\Kstarz}^2 + M_{\Kstarz}^2 q + M_B^2 q  ) $ and
\begin{equation}
N = \Bigg[ \frac{G_F^2 \alpha^2}{3 \cdot 2^{10}\pi^5 M_B^3} |\Vtb \Vts^* |^2 q \lambda^{1/2} \Big( 1 - \frac{4 m_l^2}{q} \Big)^{1/2} \Bigg]
\end{equation}
and $M_B$: mass of the \B meson, $M_{\Kstarz} $: mass of the \Kstarz meson, $m_b $: mass of the \bquark quark, $m_l $: mass of the lepton. The  $\xi_{ \perp }(q^2)$ and  $\xi_{ \parallel }(q^2)$ factors are the form factors of the \Kstarz meson and contain the non-pertubative physics. As can be seen in those formulae, the polarisation amplitudes are sensitive to new physics that might appear in the Wilson coefficients.\\
Four parameters can be expressed as a combination of the transversity amplitudes and subsequently fitted:
\begin{equation}
F_L(q^2) = \frac{ |A^{L}_0|^2 + |A^{R}_0|^2 }{ |A^{L}_0|^2 + |A^{L}_{ \perp } |^2 + | A^{L}_{ \parallel } |^2 +  |A^{R}_0|^2  + | A^{R}_{ \perp } |^2 + | A^{R}_{ \parallel } |^2 }
\end{equation}
\begin{equation}
A_{FB}(q^2) = \frac{3}{2} \frac{ \Re ( A^L_{ \parallel }( A^L_{ \perp } )^* ) - \Re (A^R_{ \parallel }( A^R_{ \perp } )^* ) }{ | A^L_0 |^2 + | A^L_{ \perp } |^2 + | A^{L}_{ \parallel } |^2 +  | A^{R}_0 |^2  + | A^{R}_{ \perp } |^2 + |A^R_{ \parallel } |^2 }
\end{equation}

\begin{equation}
A^{(2)}_T(q^2) = \frac{|A^{L}_{ \perp }|^2 + |A^{R}_{ \perp }|^2 - |A^{L}_{ \parallel }|^2 - |A^{R}_{ \parallel }|^2 }{|A^{L}_{ \perp }|^2 + |A^{R}_{ \perp }|^2 + |A^{L}_{ \parallel }|^2 + |A^{R}_{ \parallel }|^2}
\end{equation}

\begin{equation}
A^{Im}_T(q^2) = \frac{2 \Im (A^{L}_{ \parallel } (A^{L}_{ \perp })^* + A^{R}_{ \parallel } (A^{R}_{ \perp })^*)}{|A^{L}_{ \perp }|^2 + |A^{R}_{ \perp }|^2 + |A^{L}_{ \parallel }|^2 + |A^{R}_{ \parallel }|^2}
\end{equation}
It can be seen from this formulae that the $A^{(2)}_T(q^2)$ and the $A^{Im}_T(q^2)$ don't depend on the hadronic form factors $\xi_{ \perp }(q^2)$ and $\xi_{ \parallel }(q^2)$ any more, thus a great systematic uncertainty from the theoretical part is removed.\\
The amplitudes for the photon polarisation $A_L$ and $A_R$ can be related to the transversity amplitudes
\begin{equation}
A_{\perp} = \frac{A_R - A_L}{\sqrt{2}} \qquad \qquad  A_{\parallel} = \frac{A_R + A_L}{\sqrt{2}}
\end{equation}
and to $A^{(2)}_T$
\begin{equation}
A^{(2)}_T(q^2) = -2 \Re \Big( \frac{A^*_R(q^2)A_L(q^2)}{|A_R| ^2+ |A_L|^2} \Big) \sim - 2\, \frac{A_R}{A_L}
\end{equation}
The approximation holds up if $\frac{A_R}{A_L}$ is small and real.
%For the low invariant dilepton mass limit $q^2 \rightarrow 0$ the contributions from the Wilson coefficients \Ope9 and \Ope10 become negligible and the amplitude $A^{(2)}_T(q^2)$ can be simplified to
%\begin{equation}
%A^{(2)}_T(q^2 \rightarrow 0) = \frac{2 \Re ( \C7 \, (\Cp7 )^*)}{| \C7 |^2 + | (\Cp7 )^*|^2} = 2\, \frac{A_R}{A_L}
%\end{equation}
This shows clearly that the variable $A^{(2)}_T(q^2)$ is the one that contains the information about the photon polarisation.\\
\newpage
\subsection{Angular distribution}
\label{sec:angles}
The \BdKstll decay is completely defined by four independent kinematic variables; namely the lepton-pair invariant mass $q$ and the three angles $\theta_l$, $\theta_K$ and $\Phi$ which are illustrated in Figure \ref{fig:winkel} \cite{krueger} \cite{ananote}. The angles are defined as:
\begin{itemize}
\item $\theta_l$: the angle between the direction of the vector of the \en (\ep) in the dilepton rest frame, and the direction of the dilepton in the $B^0$ ($\bar{B}^0$) rest frame
\item $\theta_K$: the angle between the direction of the vector of the \pion in the \Kstarz (\Kstarzb) rest frame, and the direction of the \Kstarz (\Kstarzb) in the $B^0$ ($\bar{B}^0$) rest frame
\item $\Phi$: the angle between the planes defined by the \Kstarz daughters and the dileption daughters, in the rest frame of the \B meson.
\end{itemize}

\begin{figure}[!h]
  \begin{center}
  	\vspace*{-0.5cm}
 \includegraphics[width=0.6\textwidth]{winkel.jpg}
  \vspace*{-0.8cm}
  \end{center}
  \caption{\textit{Definition of the angles $\theta_l$, $\theta_K$ and $\Phi$ in the \BdKstll decay.}}
  \label{fig:winkel}
\end{figure}

The normalized differential decay width can be expressed as
\begin{equation}
\frac{1}{\Gamma}\frac{d^4 \Gamma}{dq^2 \, \ctl \, \ctk \, d\Phi} = \frac{9}{32 \pi}\sum_{i=1}^{9} I_i'(q^2, \theta_K) f_i(\theta_l , \Phi)
\end{equation}
where the $I_i'$ depend on $q^2$, $\theta_K$ and the helicity polarisation amplitudes $F_L(q^2)$, $A_{FB}(q^2)$, $A^{(2)}_T(q^2)$ and $A^{Im}_T(q^2)$. The $f_i$ are the corresponding angular distribution functions
\begin{eqnarray}
f_1 = 1 \qquad & f_2 = \cos(2\theta_l) \qquad & f_3 = \sin^2(\theta_l) \cos(2\Phi)\nonumber\\
f_4 = \sin(2\theta_l) \cos(\Phi) \qquad & f_5 = \sin(\theta_l) \cos(\Phi) \qquad & f_6 = \cos(\theta_l)\\
f_7 = \sin(\theta_l)\sin(\Phi) \qquad & f_8 = \sin(2\theta_l)\sin(\Phi)\qquad & f_9 = \sin^2(\theta_l)\sin(2\Phi)\nonumber
\end{eqnarray}
The $I_i$ which are sensitive to the ratio of photon polarisations $\frac{A_R}{A_L}$ are $I_3$ and $I_9$. The partial decay width can then be simplified without any loss in information about the photon polarisation by folding the angular distributions for $\Phi$ and $\theta_l$. $\Phi$ is folded by moving all values with $\Phi <0$ to $\Phi + \pi$. This removes the terms proportional to $\cos(\Phi)$ and $\sin(\Phi)$ without affecting the $\cos(2\Phi)$ and $\sin(2\Phi)$ terms. The angle $\theta_l$ is folded by mapping the interval $(\pi /2, \pi)$ onto $(0, \pi /2)$ which removes the term proportional to $\cos(\theta_l)$. 
The decay width after these transformations and neglecting the lepton mass term -- which is legitimate for the \BdKstee decay where $\frac{m^2_{\electron}}{q^2} << 1$ -- is
 \begin{eqnarray}
 \frac{1}{\Gamma}\frac{d^4\, \Gamma}{dq^2 \, \ctl \, \ctk \, d\Phi} = \frac{9}{32 \pi} [  I_1'(q^2, \cos\theta_K)\ +   I_2'(q^2, \cos\theta_K)\, (2\cos^2\theta_l -1) \nonumber \\
  +  I_3'(q^2, \cos\theta_K)\, \sin^2\theta_l \, \cos2\Phi \ +  I_9'(q^2, \cos\theta_K)\, \sin^2\theta_l \, \sin2\Phi ]
 \label{eq:partw}
 \end{eqnarray}
The $I_i'$ terms can be expressed as
\begin{eqnarray}
 I_1'(q^2, \cos\theta_K) &=& \frac{3}{4} (1-F_L(q^2)) \cdot (1-\cos^2\theta_K) + F_L(q^2) \cdot \cos^2\theta_K \nonumber \\
 I_2'(q^2, \cos\theta_K) &=& \frac{1}{4} (1-F_L(q^2)) \cdot (1-\cos^2\theta_K) - F_L(q^2) \cdot \cos^2\theta_K \nonumber \\
 I_3'(q^2, \cos\theta_K) &=& \frac{1}{2} (1-F_L(q^2)) \cdot A^{(2)}_T(q^2) \cdot (1 - \cos^2\theta_K)   \label{eq:is} \\
   I_9'(q^2, \cos\theta_K) &=& \frac{1}{2} (1-F_L(q^2)) \cdot A^{Im}_T(q^2) \cdot (1 - \cos^2\theta_K)  \nonumber 
\end{eqnarray}
By performing an angular analysis, all three remaining parameters $F_L(q^2)$, $A^{(2)}_T(q^2)$ and $A^{Im}_T(q^2)$ can be fitted at the same time and the information about the photon polarisation can be extracted. \\

\subsection{The advantages of the low $q^2$ bin in \BdKstee}
This masters thesis is part of an angular analysis that focusses on the low dilepton invariant mass $q$ region in the \BdKstee decay.\\
The choice of performing the analysis with electrons as opposed to with muons is that the sensitivity to new physics -- which could affect the photon polarisation -- is increased. This is due to the fact that the electron mass is negligible compared to $q^2$: $m^2_e<< q^2$. For a non-negligible lepton mass the $I_3'$ and $I_9'$ in Equations \ref{eq:is} are multiplied by $(1-x)$ with $ x = \frac{4m^2_{l}}{q^2}$ which thus degrades the sensitivity on $A^{(2)}_T(q^2)$ and $A^{Im}_T(q^2)$. In addition the kinematically allowed $q^2$ range is larger.\\
The choice of $20 \mevcc < q < 1\gevcc $ has two distinct advantages. The first advantage is that the contribution coming from the non-\bsll transitions in \Ope9 and \Ope10 are very small (a few \%) and therefore $A^{(2)}_T(q^2) \sim -2\, \frac{A_R}{A_L}$\footnote{Note that the angles in Section \ref{sec:angles} are defined such that for each the \bsll and the \bbsll decay $A^{(2)}_T(q^2)$ corresponds to the ratio of suppressed amplitude over favoured amplitude.} (see Section \ref{sub:polamp}). The analysis of the \BdKstee decay at low $q^2$ is therefore directly sensitive to the \bsll transitions and the photon polarisation. The second advantage of constraining the dilepton invariant mass lies in the $q^2$ dependence of the $F_L(q^2)$ amplitude shown in Figure \ref{fig:sens}. $F_L(q^2)$ is low for small $q^2$ values and increases with $q^2$. Since the amplitude $A^{(2)}_T(q^2)$ appears in $I_3'$ in Equation \ref{eq:is} in combination with $(1-F_L(q^2))$ the sensitivity to $A^{(2)}_T(q^2)$ is highest for low $q^2$. \\
The effects of the low lepton mass of the electrons opposed to muons combined with the effects of $q^2$ on $F_L(q^2)$ are shown in Figure \ref{fig:sens}.\\
\begin{figure}[ht]
\vspace*{-0.5cm}
  \begin{center}
  	\subfigure{\includegraphics[width=0.52\textwidth]{Fl.pdf}}
   \subfigure{\includegraphics[width=0.47\textwidth]{sensitivity.jpg}}
  \vspace*{-0.5cm}
  \end{center}
  \caption{\textit{The parameter $F_L(q^2)$ (left) and the sensitivity on the parameter $A^{(2)}_T(q^2)$ for muons and electrons (right). $F_L(q^2)$ was determined by analysis of $\Bd \rightarrow K^{*0} \mup \mun$ for different bins of $q^2$.}\cite{mumu}}
  \label{fig:sens}
\end{figure}

\subsection{The lower cut: $q\, > 20\mevcc$}
Although the \BdKstee events at low $q^2$ carry the most information about the photon polarisation a cut of $q\, > 20\mevcc$ has to be performed. At dielectron invariant mass $m(\epem)\approx 0$ the two electrons are nearly collinear and therefore the angle between them is very small. At the limit of $q^2 \rightarrow 0$ the dilepton plane cannot be defined any more and thus the $\Phi$ angle cannot be determined. By experiencing multiple scattering of the electrons in the detector the plane defined by the (\epem) pair is extremely distorted. This leads to a great uncertainty on the $\Phi$ angle which makes these events useless for the angular analysis. Furthermore the measured (\epem) invariant mass is increased due to added opening angle between the electron and the positron due to multiple scattering. \\
To allow for a small uncertainty on the measured $\Phi$ angle a study has been performed on the resolution of the $\Phi$ angle. Figure \ref{fig:phires} shows the resolution of the $\Phi$ angle with respect to the invariant mass of the electron pair. The uncertainty on $\Phi$ drops off exponentially and remains stable from $150 \mevcc$ onwards.\\
The cut of $q\, > 20\mevcc$ was chosen as the $\Phi$ resolution is acceptable at from this value on. This cut also allows to keep a large amount if events with low electron pair invariant mass.
\begin{figure}[ht]
\vspace{-0.4cm}
  \begin{center}
\includegraphics[width=0.5\textwidth]{phires.jpg}
  \vspace*{-0.8cm}
  \end{center}
  \caption{\textit{The resolution of the $\Phi$ angle for different bins of (\epem) invariant mass obtained from \BdKstee Monte Carlo.}}
  \label{fig:phires}
\end{figure}
\chapter{Binned \KlPiPi vs \4Pi yields }

\section{Overview}
The values for $MM_i$ are obtained using:
\begin{equation}
MM_i = (N_i^{meas} - B_i^{peak} - B_i^{flat} - B_i^{cont})/ \epsilon_i
\end{equation}
$MM_i$ : absolute number of \4Pi vs \KlPiPi events in bin i. This is the quantity that enters the fit for $F_+$.\\
$N_i^{meas}$ : total number of reconstructed and selected events in bin i. \\
$B_i^{peak}$ : number of reconstructed and selected peaking bkg. events in bin i.\\
$B_i^{flat}$ : number of reconstructed and selected flat bkg. events in bin i.\\
$B_i^{cont}$ : number of reconstructed and selected continuum bkg. events in bin i.\\
$\epsilon_i$ : reconstruction and selection efficiency for \4Pi vs \KlPiPi events in bin i. \\



\section{Raw number of selected \4Pi vs \KlPiPi events }
The selection of \4Pi vs \KlPiPi was done as outlined in Chris' report. The resulting numbers are listed below.\\
\begin{table}[!h]
	\begin{center}
	\begin{tabular}{c| l}
		bin i & $N^{meas}_i\ \pm \sqrt{N^{meas}_i}$  \\
		\hline 
		\hline
1 & 172 $\pm$ 13.11 \\ 
2 & 71 $\pm$ 8.43 \\ 
3 & 60 $\pm$ 7.75 \\ 
4 & 25 $\pm$ 5 \\ 
5 & 59 $\pm$ 7.68 \\ 
6 & 32 $\pm$ 5.66 \\ 
7 & 71 $\pm$ 8.43 \\ 
8 & 99 $\pm$ 9.95 \\ 
\end{tabular}
\end{center}
\caption{\textit{Number $N^{meas}_i$ of reconstructed \4Pi vs \KlPiPi events summed over bin i and -i. The errors are taken as $\sqrt{N^{meas}_i}$.}}
\end{table}


\section{Determination of number of peaking background events}
\label{sec:klpeak}
Three decays are identified as peaking bkg (peaking in missing mass squared) for \KlPiPi vs \4Pi using the gen MC (see Figure \ref{fig:missmasssq}):
\begin{itemize}
\item \KlPiPi vs \KsPiPi : 8\% of the events in the signal window
\item \KsPiPi vs \4Pi : 1.9\% of the events in the signal window
\item \KsPiPi vs \KsPiPi : 0.3\% of the events in the signal window
\end{itemize}
(signal window: $0.2\, GeV^2$ < miss. mass sq < $0.3\, GeV^2$). Due to the very low contribution the \KsPiPi vs \KsPiPi bkg is being neglected.\\
\begin{figure}[!h]
	\vspace*{-0.cm}
	\begin{center}
		\includegraphics[width=1.\textwidth]{missmasssq.pdf}
		\vspace*{-1.5cm}
	\end{center}
	\caption{\textit{Missing mass squared distribution of different types of events reconstructed as  \KlPiPi vs \4Pi according to the gen MC.}}
	\label{fig:missmasssq}
\end{figure}\\
As for the \KsPiPi vs \4Pi the number of peaking bkg events per bin is determined from
\begin{equation}
B_i^{peak} = B_{tot}^{peak} \cdot a_i^{peak}
\end{equation}
for both bkgs separately, where $B_{tot}^{peak}$ denotes the total number of peaking bkg events in the selected data sample and $a_i^{peak}$ the percentage of peaking bkg  events in bin i.\\

\subsection{Total number of peaking background events in selected data sample}
The total number of peaking bkg events in the selected sample is determined using the generic MC. The same reconstruction and selection as to the data is applied to the generic MC. This results in $98 \pm 9.90$ and $419 \pm 20.47$ selected \KlPiPi vs \KsPiPi events in the lumix10 and lumix20 sample respectively and $15 \pm 3.87$ and $90 \pm 9.49$ \KsPiPi vs \4Pi events (the errors are taken to be the sqrt of the yields).
These numbers are then scaled appropriately to give the expected number of total bkg events in the reconstructed data sample and no error on the scaling factors is assumed.
\begin{equation}
B_{tot}^{peak}(\KlPiPi\, vs\, \KsPiPi) = 35.51 
\end{equation}
and 
\begin{equation}
B_{tot}^{peak}(\KsPiPi\, vs\, 4\pion) = 9.42
\end{equation}
Since the branching fraction of $\D \rightarrow \KlPiPi$ is unknown, the fraction of \KlPiPi events in the gen MC sample can't be completely accurate. This will be accounted for by applying a systematic to the number of events obtained from the gen MC (by assuming that the branching fraction used by CLEO is 20\% too small/big).\\

\subsection{Distribution of \KlPiPi vs \KsPiPi background events over bins}
\label{sec:klclean}
As for the \KsPiPi vs \KsPiPi bkg in \KsPiPi vs \4Pi the distribution of \KlPiPi vs \KsPiPi events per bin is determined by applying a 'reversed \KS veto' on the data. The gen MC sample suggests that after the reversed \KS veto sample contains:
\begin{itemize}
\item \KlPiPi vs \KsPiPi : 92\%
\item flat events: 5\% 
\item \KlPiPi vs \4Pi : 1.5\%
\item \KsPiPi vs \KsPiPi : 1.5\%
\end{itemize} 
The last two are neglected for now (a systematic will be appointed later).\\
\\
In order to obtain the distribution of \KlPiPi vs \KsPiPi events over the bins, the contribution of flat bkg events is estimated as in Section \ref{sec:flat} and removed from the data.\\ 

\subsubsection{Selection-bias}	
Unfortunately 'reversed \KS veto' introduced a bias in the distribution of expected \KlPiPi vs \KsPiPi events over the bins. \\
To compensate for this bias the values for $a_i^{peak}$ are being weighted with the ratio of reversed \KS veto over the \KS veto efficiencies determined from the \KlPiPi vs \KsPiPi signal MC sample.\\

\subsection{Distribution of \KsPiPi vs \4Pi background events over bins}
The 'true' distribution of \KsPiPi vs \4Pi events in phasespace is taken from the results of the \KsPiPi vs \4Pi analysis, namely the values of Table \ref{tab:MMKs}.\\


\subsection{Resulting number of peaking background events in selected data sample}
Applying these scaling values to the factors $a_i$ yields the total amount of peaking bkg contamination in the data sample listed below.

\begin{table}[!h]
	\begin{center}
		\begin{tabular}{c| c}
			bin i & $B^{peak}_i(\KlPiPi\, vs\, \KsPiPi)$    \\
			\hline 
			\hline
1 & 12.267 $\pm$ 3.5 $\oplus$ 1.06294 \\ 
2 & 3.81339 $\pm$ 2 $\oplus$ 0.666631 \\ 
3 & 2.85437 $\pm$ 2 $\oplus$ 0.608983 \\ 
4 & 1.48931 $\pm$ 1.5 $\oplus$ 0.442086 \\ 
5 & 3.68781 $\pm$ 2 $\oplus$ 0.63337 \\ 
6 & 1.72474 $\pm$ 1.5 $\oplus$ 0.415589 \\ 
7 & 4.03395 $\pm$ 2 $\oplus$ 0.709566 \\ 
8 & 5.63443 $\pm$ 2.5 $\oplus$ 0.804392 \\ 
	\end{tabular}
	\end{center}
	\caption{\textit{Number of \KsPiPi vs \KlPiPi background events per bin in the data sample.The first error is the Poisson error on the yields, the second error comes from the sys- tematic uncertainty on the distribution over the bins. }}
	\vspace*{1cm}
\end{table}

\begin{table}[!h]
	\begin{center}
		\begin{tabular}{c| c}
			bin i & $B^{peak}_i(\KsPiPi\, vs\, 4\pion)$    \\
			\hline 
			\hline
1 & 1.32987 $\pm$ 1 $\oplus$ 0.302816 \\ 
2 & 0.856693 $\pm$ 1 $\oplus$ 0.227813 \\ 
3 & 0.707903 $\pm$ 0.5 $\oplus$ 0.194539 \\ 
4 & 0.438423 $\pm$ 0.5 $\oplus$ 0.146814 \\ 
5 & 2.38231 $\pm$ 1.5 $\oplus$ 0.349486 \\ 
6 & 0.910659 $\pm$ 1 $\oplus$ 0.220425 \\ 
7 & 1.19564 $\pm$ 1 $\oplus$ 0.258476 \\ 
8 & 1.59349 $\pm$ 1.5 $\oplus$ 0.294614 \\ 

	\end{tabular}
	\end{center}
	\caption{\textit{Number of \4Pi vs \KsPiPi background events per bin in the data sample. The first error is the Poisson error on the yields, the second error comes from the sys- tematic uncertainty on the distribution over the bins. }}
	\vspace*{1cm}
\end{table}

\section{Determination of number of flat and continuum background events}
The total number of flat bkg events and continuum bkg events is determined using a form of the 'Powell method'.\\
In this method the missing mass squared region is divided into three regions: the lower region ($0.05\,GeV^2$ - $0.2\,GeV^2$), the signal region ($0.2\,GeV^2$ - $0.3\,GeV^2$) and the higher region ($0.45\,GeV^2$ - $0.8\,GeV^2$).
 The raw event yield $D_i$ in rach sideband i is expressed as:
\begin{equation}
D_i = T_{Si} + T_{Pi} + T_{Ci} + T_{Fi}
\end{equation}
where $T_{Si}$ denotes the raw signal yield in sideband i, $T_Pi$ the raw yields of peaking bkg events, $T_Fi$ the raw yields of flat bkg events and $T_{Ci}$ the raw yields of continuum events.\\
Since the raw peaking bkg yield has been calculated earlier, we get a system of three equations, one for each region of missing mass squared:
\begin{equation}
D_i -T_{Pi} = T_{Si} + T_{Ci} + T_{Si}
\end{equation}
By assuming that the ratio of numbers of events in each sideband is modulated correctly by the MC samples - e.g. $T_{S1} = \frac{m_{S1}}{m_{S2}} T_{S2}$, where $m_{Si}$ is the number of signal events in the MC sample - these three equations can be expressed in terms of three unknowns:
\begin{equation}
D_1 -T_{P1} =  \frac{m_{S1}}{m_{S2}} T_{S2}  + T_{C1} + \frac{m_{F1}}{m_{F3}} T_{F3}  \\
\end{equation}
\begin{equation}
D_2 -T_{P2} =   T_{S2}  + \frac{m_{C2}}{m_{C1}} T_{C1} + \frac{m_{F2}}{m_{F3}} T_{F3} \\
\end{equation}
\begin{equation}
D_3 -T_{P3} =  \frac{m_{S1}}{m_{S3}} T_{S2}  + \frac{m_{C3}}{m_{C1}} T_{C1} +  T_{F3} \\
\end{equation}
This system can be solved by a simple matrix inversion and thus the number of continuum and flat bkg events in the signal window found (Chris thesis Section 3.6.3). \\
The difference to method described in Chris thesis is that instead of dividing the flat bkg into 'peaking in low sideband' and 'peaking in high sideband' the bkg is divided into continuum and flat bkg.

%The total number of continuum events is extrapolated from the CLEO continuum MC sample. This results in 38.93 $\pm$ 7.70 continuum events in the signal window of our data sample.\\
%Furthermore the continuum bkg is assumed to be flat in phasespace, so the 38 events are distributed across the bins according to Table \ref{tab:a_i}. This results in the following number of continuum events per bin:
%
%\begin{table}[!h]
%	\begin{center}
%		\begin{tabular}{c| l}
%			bin i & $B^{cont}_i$   \\
%			\hline 
%			\hline
%1 & 12.87 $ \pm$ 2.55 \\ 
%2 & 4.44 $ \pm$ 0.88 \\ 
%3 & 2.49 $ \pm$ 0.49 \\ 
%4 & 2.30 $ \pm$ 0.45 \\ 
%5 & 5.19 $ \pm$ 1.03 \\ 
%6 & 3.15 $ \pm$ 0.62 \\ 
%7 & 3.28 $ \pm$ 0.65 \\ 
%8 & 5.22 $ \pm$ 1.03 \\ 
%	\end{tabular}
%	\end{center}
%	\caption{\textit{Number of continuum background events per bin in the data sample. }}
%\end{table}
%
%
%\section{Determination of number of flat background events}

This results in 42.07 flat bkg events in the data sample and 17.06 continuum events.\\
The continuum bkg is also assumed to be flat in phasespace, so the 17 events are distributed according to the area of the bins. This results in the following number of continuum events per bin:
\clearpage
\begin{table}[!h]
	\begin{center}
		\begin{tabular}{c| c}
			bin i & $B^{cont}_i$   \\
			\hline 
			\hline
1 & 5.63723 $\pm$ 2.5 \\ 
2 & 1.94538 $\pm$ 1 \\ 
3 & 1.08944 $\pm$ 1 \\ 
4 & 1.00759 $\pm$ 1 \\ 
5 & 2.27215 $\pm$ 1.5 \\ 
6 & 1.37909 $\pm$ 1.5 \\ 
7 & 1.43683 $\pm$ 1.5 \\ 
8 & 2.28738 $\pm$ 1.5 \
	\end{tabular}
	\end{center}
	\vspace*{-0.5cm}
	\caption{\textit{Number of continuum bkg events per bin in the data sample. The error is the Poisson error on the bin-yields.}}
\end{table}

The number of flat bkg events in bin i are then:
\begin{table}[!h]
	\begin{center}
		\begin{tabular}{c| c}
			bins & $B_i^{flat}$  \\
			\hline
1 & 13.9052 $\pm$ 4 \\ 
2 & 4.79864 $\pm$ 2 \\ 
3 & 2.6873 $\pm$ 1.5 \\ 
4 & 2.4854 $\pm$ 1.5 \\ 
5 & 5.60467 $\pm$ 2.5 \\ 
6 & 3.40177 $\pm$ 1.5 \\ 
7 & 3.5442 $\pm$ 1.5 \\ 
8 & 5.64223 $\pm$ 2.5 \\ 
\end{tabular}
\end{center}
\vspace*{-0.5cm}
\caption{\textit{Number of flat bkg events per bin in the data sample. The error is the Poisson error on the bin-yields.}}
\end{table} 

\section{Determination of the signal selection-efficiency}
The signal selection efficiency in each bin is determined using a sample of 50\,000 \KlPiPi vs \4Pi signal MC events. The results are listed below.
\begin{table}[!h]
	\begin{center}
		\begin{tabular}{c| c }
			bin & $\epsilon$ [ \%] \\
			\hline
1 & 0.27805 $\pm$ 0.00155862 \\ 
2 & 0.270163 $\pm$ 0.00262954 \\ 
3 & 0.2553 $\pm$ 0.00345042 \\ 
4 & 0.258504 $\pm$ 0.00360249 \\ 
5 & 0.262955 $\pm$ 0.00241227 \\ 
6 & 0.26856 $\pm$ 0.00311724 \\ 
7 & 0.266931 $\pm$ 0.00304808 \\ 
8 & 0.267617 $\pm$ 0.00241776 \\
			\end{tabular}
	\end{center}
	\vspace*{-0.5cm}
	\caption{\textit{Signal efficiency per bin.}}
\end{table} 
\clearpage
\section{Resulting $MM_i$}
Since $M_i$	and $M_{-i}$ are the same, the numbers are directly evaluated for the sum of $M_i$ and $M_{-i}$ and listed on the table below.
\begin{table}[!h]
	\begin{center}
		\begin{tabular}{c| c }
			bin & $MM_i$  \\
			\hline
1 & 134.113 $\pm$ 13.8788 $\oplus$ 0.751773 $\oplus$ 1.02659 \\ 
2 & 59.2287 $\pm$ 8.89065 $\oplus$ 0.576484 $\oplus$ 0.662635 \\ 
3 & 55.3925 $\pm$ 8.62598 $\oplus$ 0.748637 $\oplus$ 0.640571 \\ 
4 & 20.3397 $\pm$ 5.73716 $\oplus$ 0.283453 $\oplus$ 0.459255 \\ 
5 & 46.0104 $\pm$ 8.63545 $\oplus$ 0.422086 $\oplus$ 0.646829 \\ 
6 & 24.5822 $\pm$ 6.22456 $\oplus$ 0.285332 $\oplus$ 0.415563 \\ 
7 & 61.1564 $\pm$ 8.97011 $\oplus$ 0.698343 $\oplus$ 0.71385 \\ 
8 & 84.1326 $\pm$ 10.7023 $\oplus$ 0.76009 $\oplus$ 0.807176 \\ 
	\end{tabular}
\end{center}
\caption{\textit{$M_i$ + $M_{-i}$ The first error comes from the Poisson error on the raw data yields and the bkg yields and the second error from the systematic uncertainty on the peaking bkg distribution and the third from the signal reconstruction and selection efficiency.}}
\label{tab:MMKl}
\end{table} 

\section{Summary}
\begin{table}[!h]
	\begin{center}
		\begin{tabular}{c| c | c | c| c|c|c }
		bin & raw yields & $B^{peak}_i(\KlPiPi\, vs\, \KsPiPi)$ & $B^{peak}_i(\KsPiPi\, vs\, 4\pion)$ & $B_i^{cont}$ & $B_i^{flat}$ & signal  \\
		\hline 
		\hline
1 & 172 & 12.26 & 1.32 & 13.90 & 5.63 & 134.11 \\ 
2 & 71 & 3.81 & 0.85 & 4.79 & 1.94 & 59.22 \\ 
3 & 60 & 2.85 & 0.70 & 2.68 & 1.08 & 55.39 \\ 
4 & 25 & 1.48 & 0.43 & 2.48 & 1.00 & 20.33 \\ 
5 & 59 & 3.68 & 2.38 & 5.60 & 2.27 & 46.01 \\ 
6 & 32 & 1.72 & 0.91 & 3.40 & 1.37 & 24.58 \\ 
7 & 71 & 4.03 & 1.19 & 3.54 & 1.43 & 61.15 \\ 
8 & 99 & 5.63 & 1.59 & 5.64 & 2.28 & 84.13 \\	
		\end{tabular}
	\end{center}
	\caption{\textit{Summary of contributions to the raw yields per bin.}}
\end{table}

\chapter{Study of the reconstruction of the \BdKstee invariant mass}
\chaptermark{Chapter4}
\label{chapter3}
%\addcontentsline{toc}{chapter}{Study of the reconstruction of the \BdKstee invariant mass}
The event reconstruction is a crucial part of the analysis of any decay. At \lhcb, the offline event reconstruction is made in two steps. The first step is executed by the software \brunel that performs subdetector and global reconstruction using pattern recognition for both Monte Carlo and real data. \brunel takes the detector information and reconstructs \textit{tracks} from the hits. The information for each \textit{track} is then stored for further analysis.\\
The second step, the one accessible and adjustable by every user, is executed by the \davinci physics and analysis framework. At the beginning of every event reconstruction \davinci builds lists of \textit{particles}. For each particle-type, such as electron or kaon, there are several lists ranging from very loose to very tight requirements on the \textit{tracks}. For each list, \davinci selects all the \textit{tracks} provided by \brunel that fulfil the list conditions and fully reconstructs them as \textit{particles}, quantities of physics with a four-momentum. In the course of this reconstruction the mass of the \textit{particle} is fixed to the particle-type's mass defined by the list.
%These \textit{particles} are made from the \textit{track} information provided by \brunel and are quantities of physics with a four-momentum. In the course of computing the \textit{particles} properties their masses are fixed to the value corresponding to the condition of the list, e.g. all particles in the list \textsc{SdtAllLooseKaons} have a mass of $493 \mevcc$.\\
From the lists of \textit{particles} \davinci reconstructs intermediate particles (e.g. \Kstarz from \Kp and \pim) and finally the entire \BdKstee decay. \davinci allows the user to choose the \textit{particle} lists and the algorithms that compute their properties, additionally to the way the \textit{particles} are being combined.\\ % einfuegen von bezug auf arbeit
\\
The greatest difficulty in reconstructing the \BdKstee decays is the measurement of the electron\footnote{Throughout this chapter electron stands for electrons and positrons if not mentioned otherwise.} energy. This difficulty arises from bremsstrahlung radiation and propagates to a vast degradation of the \Bd mass resolution. The effects occurring due to bremsstrahlung of the electrons and the means of limiting the loss in precision -- namely different event reconstruction tools implemented in the \davinci framework -- are presented in the following Section. \\
In Section \ref{sec:DiElectronMaker} the focus is put on the reconstruction on the invariant mass of the electron pair $M_{inv}(\epem)$.\\
\newpage
\section{Bremsstrahlung radiation}
\label{sec:bremsstrahlung}
When traversing the material of the detector charged particles undergo a probability of emitting bremsstrahlung. The cross-section for bremsstrahlung emission $\sigma_{bs}$ is anti-proportional to the square of the charged particle's mass $m$ \cite{WRLeo}.
\begin{equation}
 d\sigma_{bs} \propto \frac{e^2}{(mc^2)^2}
\end{equation}
Since the amount of material is low in \lhcb the probability of emitting bremsstrahlung photons is negligible for heavier particles, but very light particles such as electrons and positrons can lose a significant amount of energy through bremsstrahlung radiation. The amount of energy emitted through bremsstrahlung for muons $E_{bs}^{\mmu}$ for instance, is suppressed by a factor $40\ 000$ with respect to that of electrons $E_{bs}^{\electron}$.
\begin{equation}
\frac{E_{bs}^{\electron}}{E_{bs}^{\mmu}} = \frac{m_{\electron}^2}{m_{\mmu}^2} = 2.3 \cdot 10^{-5} 
\label{eq:evsmu}
\end{equation}
%At energies below a few \gev only electrons emit a significant amount of bremsstrahlung radiation in the \lhcb detector.\\
The radiated energy spectrum for the electrons from \BdKstee before the \lhcb magnet is shown in Figure \ref{fig:totalEnergyLossBrems}. The distribution was obtained by performing a Fast-Sim \cite{ananote} on Monte Carlo at simulation level under the assumption of complete screening \cite{pdg}.
\begin{figure}[ht]
  \begin{center}
    \includegraphics[width = 0.55\textwidth]{EnergyLossBrems.pdf}
    \end{center}
    	\vspace*{-0.8cm}
  \caption{\textit{Percentage of energy loss of electrons/positrons traversing the \lhcb detector before the magnet. $E_{brems}(\epm)$ denotes the energy radiated by the electron/ positron while $E_0(\epm)$ stands for the initial electron/positron energy. Distribution obtained by performing a Fast-Sim on Monte Carlo Data at simulation level under the assumption of complete screening.}}
  \label{fig:totalEnergyLossBrems}
\end{figure}

The bremsstrahlung radiation process is of statistical nature. Figure \ref{fig:totalEnergyLossBrems} shows the amount of radiated energy. It is therefore crucial to identify photons in the detector that have been emitted by electrons through bremsstrahlung radiation and add their four-momentum to the four-momentum of the electron track. The \Bd mass distribution without any kind of bremsstrahlung reconstruction can be seen in Figure \ref{fig:noBremReco}.
\newpage
\begin{figure}[ht]
  	\centering
    \includegraphics[width = 0.55 \textwidth]{oldBrem_noBremReco.jpg}
  \caption{\textit{The \Bd mass distribution of \BdKstee \lhcb Monte Carlo without any reconstruction of bremsstrahlungs photons.}}
  \label{fig:noBremReco}
\end{figure}

\subsection{Bremsstrahlung recovery}
\label{sec:bremsstrahlungrecovery}
If the emission of the bremsstrahlung happens before the magnet, the electron will be deflected from its initial trajectory by the magnetic field while the bremsstrahlung photon's momentum will not change, as shown in Figure \ref{fig:bremsstrahlung}. Thus, the electron and its photon will deposit their energies in different positions in the \ecal. To allow nonetheless for an assignment of the bremsstrahlung photon with its electron \textit{bremsstrahlung recovery algorithms} have been developed. These algorithms search for bremsstrahlung photon candidates in the \ecal and try to match them to the corresponding electron track.Two different bremsstrahlung recovery algorithms implemented in the \lhcb physics analysis framework will be presented in Sections \ref{sub:oldBrem} and \ref{sub:newBrem}.
\begin{figure}[ht]
\vspace*{-0.4cm}
  \begin{center}
    \includegraphics[width=0.6\textwidth]{BremReco.jpg}
  \end{center}
  \vspace*{-0.5cm}
  \caption{\textit{Schematic illustration of bremsstrahlung emission of electrons in the \lhcb detector.}}
  \label{fig:bremsstrahlung}
\end{figure}
If the emission of the bremsstrahlung happens after the magnet, the electron and its photon will deposit their energies in the same \ecal cells. Thus the photon energy is automatically added to the energy of its electron. \\


\subsubsection{Bremsstrahlung recovery algorithm in \davinci v29 }
\label{sub:oldBrem}
The first bremsstrahlung recovery tool was the standard algorithm implemented in all version of \davinci up to v29. An illustration of its methodology is shown in Figure \ref{fig:oldBremAdd}.
\begin{figure}[ht]
\vspace*{-0.5cm}
  \begin{center}
  \includegraphics[width=0.6\textwidth]{oldBremAdder.jpg}
  \end{center}
  \vspace*{-0.5cm}
  \caption{\textit{Schematic illustration of the bremsstrahlung recovery algorithm \davinci v29. Black line: curve of the electron track. Dark blue dotted line: linear extrapolation of the electron track from its very first state to the \ecal. Clear blue dotted line: linear extrapolation of the electron track from its last state before the magnet to the \ecal. Red area in the \ecal: area from where photon candidates will be matched to the electron track.}}
  \label{fig:oldBremAdd}
\end{figure}
The algorithm starts with a \textit{particle} list of electrons and searches for reconstructed photon candidates coming from the electron tracks.
To predict the position of bremsstrahlung photon candidates in the \ecal, the algorithm linearly extrapolates the electron track from the \ttracker (last state before the magnet) to the \ecal. All neutral clusters in the \ecal whose barycentric positions match the position of the extrapolated electron track with a $\chi^2 < 300$ are accepted as bremsstrahlung photons. Their four-momentum is added to the four-momentum of the electron. Furthermore there is no limit of bremsstrahlung photon candidates that can be added to one electron.\\


\subsubsection{Bremsstrahlung recovery algorithm in \davinci v30}
\label{sub:newBrem}
The second bremsstrahlung recovery tool is implemented in \davinci v30. Its methodology is illustrated in Figure \ref{fig:newBremAdd}.
\begin{figure}[ht]
  \begin{center}
  \label{fig:newBremAdder}
  \includegraphics[width=0.6\textwidth]{newBremAdder.jpg}
  \end{center}
  \vspace{-0.8cm}
  \caption{\textit{Schematic illustration of the bremsstrahlung recovery algorithm \davinci v30. Black line: curve of the electron track. Dark blue dotted line: linear extrapolation of the electron track from its very first state to the \ecal. Clear blue dotted line: linear extrapolation of the electron track from its last state before the magnet to the \ecal. Red area in the \ecal: area from where photon candidates will be matched to the electron track.}}
  \label{fig:newBremAdd}
\end{figure}
For each \textit{particle} in the electron list two linear extrapolations of the track are computed, namely the extrapolation of the electron track from its very first state in the \velo to the \ecal and from its state in the \ttracker to the \ecal. Photon candidates must satisfy tighter conditions than for the algorithm in \davinci v29 \footnote{Photon candidates for the \davinci v30 bremsstrahlung recovery algorithm must satisfy the conditions of a PhotonID greater than $-0.5$ and a $p_T$ greater than $75 \mevc$.}.
The $x$ position of the photon candidates in the \ecal must lie between the $x$ positions of the two electron track extrapolations within $\pm 2 \sigma_{x}$. The $y$ position of the photon candidate in the \ecal must match the $y$ position of the electron track extrapolations within $\pm 2 \sigma_{y}$. The $\sigma_{x,y}$ denote the square root of the quadratic sum of the photon cluster spread with the error on the electron track extrapolations. \\



\subsection{The mechanism of double counting}
\label{sec:doublecounting}
Both bremsstrahlung recovery algorithms implemented in \davinci have the disadvantage of generating \textit{double counting}. Double counting occurs when one bremsstrahlung photon candidate can be associated to more than one electron track. This happens particularly often for the \BdKstee decay mode where the $e^+e^-$ invariant mass is very low, yielding small angles between the electron and the positron, as is illustrated in Figure \ref{fig:schemdc}. The distribution of double counting events depending on the $M_{inv}(e^+e^-)$ is shown in Figure \ref{fig:Minveedouble counting}. In the case of double counting the bremsstrahlung recovery algorithms add the photon's four momentum to both the electron's and the positron's four momentum. In the case of \BdKstee, these events will show a non-physically high $B^0$ mass and lead to a non-typical symmetric \Bd mass distribution that can be seen in Figure \ref{fig:oldBremAddernodcc}. 
\begin{figure}[ht]
\vspace*{-0.5cm}
  \begin{center}
  \label{fig:newBremAdder}
  \includegraphics[width=0.6\textwidth]{doublecounting.jpg}
  \end{center}
  \vspace*{-0.5cm}
  \caption{\textit{Schematic illustration of the mechanism of double counting. Due to a very low $e^+e^-$ invariant mass the angle between the electron and the positron is so small that the photon can be associated to both.}}
  \label{fig:schemdc}
  \vspace*{-0.5cm}
\end{figure}

\begin{figure}[ht]
  \begin{center}
  \vspace*{-.6cm}
  \subfigure{\includegraphics[width=0.48\textwidth]{dccats_Minvee_oldBrem.pdf}}
    \subfigure{\includegraphics[width=0.48\textwidth]{dccats_Minvee_newBrem.pdf}}
  \vspace*{-1.0cm}
  \end{center}
  \caption{\textit{$M_{inv}(e^+e^-)$ distribution obtained by reconstructing \lhcb Monte Carlo with \davinci v29 (left) and \davinci v30 (right) respectively.. The pink distribution are the events with double counting.}}
  \label{fig:Minveedouble counting}
  \vspace*{-0.5cm}
\end{figure}


\begin{figure}[ht]
\vspace*{-0.5cm}
  \begin{center}
  \subfigure{\includegraphics[width=0.49\textwidth]{MC_Bmass_dielectron_TM_nodccorrection_simplehisto.pdf}} 
  \subfigure{\includegraphics[width=0.49\textwidth]{MC_Bmass_dielectron_TM_nodccorrection_simplehisto_newBrem.pdf}}
  \vspace*{-1.0cm}
  \end{center}
  \caption{\textit{\Bd mass shape. The pink distribution are the events with double counting. The \Bd distribution is obtained by reconstructing \lhcb Monte Carlo with \davinci v29 (left) and \davinci v30 (right) respectively.}}
  \label{fig:oldBremAddernodcc}
\end{figure}
As can be seen from Figures \ref{fig:oldBremAddernodcc} and \ref{fig:Minveedouble counting} the amount of double counting is higher for the \davinci v29 bremsstrahlung recovery. The \lhcb Monte Carlo sample shows that double counting events account for 27 \% of all the signal events in \davinci 29 while it's only about 14\% of the signal events for \davinci v30. \\



\subsubsection{The double counting correction}
\label{sub:doublecountingcorrection}
In the course of the analysis four different methods to encounter the problem of double counting have been developed and tested:
\begin{compactenum}[a)]
\item The energy of the double counted photon candidate is assigned to the lepton with lower energy.
\item The energy of the double counted photon candidate is assigned to the lepton with higher energy.
\item The energy of the double counted photon candidate is assigned to a randomly chosen lepton. 
\item The energy of the double counted photon candidate is equally divided between the two leptons.
\end{compactenum}
After assigning the bremsstrahlung photon's energy, the four-momenta of the leptons is recalculated. From these new lepton four-momenta and the original four-momenta of the Kaon and the Pion the \Bd mass is recalculated.\\
The recalculated \Bd mass distributions for the four different methods applied to the \BdKstee \lhcb Monte Carlo sample made with the \davinci v29 are shown in Figure \ref{fig:double countingcorrection}. \\
While the first three methods yield extremely similar results, the fourth method does not reproduce the \Bd mass shape correctly. For further analysis the third method of encountering double counting is chosen. Assigning of the bremsstrahlung photon's energy to one randomly chosen electron does not undergo the risk of biasing any energy distribution and yields the most accurate \Bd mass reconstruction of the four tested algorithms.\\
\begin{figure}[ht]
\vspace*{-0.4cm}
  \begin{center}
    \includegraphics[scale=0.6]{minvB_nodccorrection.pdf}
  \vspace*{-1.0cm}
  \end{center}
  \caption{\textit{\Bd mass shapes for different algorithms of double counting correction. Black: no double counting correction; red: a); yellow: b); blue: c); green: d). }}
  \label{fig:double countingcorrection}
\end{figure}
\\

\subsection{Comparison of bremsstrahlung recovery algorithms}
The two bremsstrahlung recovery tools are applied to the \BdKstee \lhcb Monte Carlo, together with the double counting correction developed in the previous section. Additionally the selection developed for the 2011 dataset \cite{michellesthesis} is applied to the samples to estimate the effects on the final dataset.\\
To quantify the quality of the reconstruction tools a double Crystal-Ball distribution \cite{crystal} (for more details see Section \ref{sec:pdfs}) is fitted to the \Bd mass shape. The two Crystal-Ball distributions share the same $\mu_{\B}$, $\alpha$ and $n$. A weighted width $\sigma^w$ is calculated from the width $\sigma_1$ and $\sigma_2$ from the two Crystal-Ball distributions, where $f$ denotes the relative contribution from the first Crystal-Ball distribution.
\begin{equation}
\sigma^w = f \sigma_1 + (1- f) \sigma_2
\end{equation}

The two resulting distributions can be seen in Figure \ref{fig:comparebrems}. The resulting weighted resolutions are listed in Table \ref{tab:sigmaw}.

\begin{figure}[!h]
\vspace*{-0.cm}
  \begin{center}
  \subfigure{\label{fig:oldBremAdder}\includegraphics[width=0.49\textwidth]{oldBrem_crystalfit.jpg}}
  \subfigure{\label{fig:newBremAdder}\includegraphics[width=0.48\textwidth]{newBrem_crystalfit.jpg}} 
  \vspace*{-1.0cm}
  \end{center}
  \caption{\textit{\Bd mass distribution obtained from \lhcb Monte Carlo with \davinci v29 (left) and \davinci v30 (right) and the double counting correction.}}
  \label{fig:comparebrems}
\end{figure}

\begin{table}[!h]
\begin{center}
\begin{tabular}{c|c|c|c|c}
& $\sigma_1$ & $\sigma_2$ & $f$ & $\sigma^w$ \\
\hline
\davinci v29 &  47 \mevcc & 199 \mevcc & 0.76 & 83 \mevcc \\
\hline
\davinci v30 & 45 \mevcc & 251 \mevcc & 0.86 & 74 \mevcc \\
\end{tabular}
\end{center}
\vspace*{-0.5cm}
\caption{\textit{Relevant results of the fit of a double Crystal-Ball distribution to the \BdKstee \lhcb Monte Carlo reconstructed with \davinci v29 and \davinci v30 and the double counting correction. $\sigma^w$ denotes the weighted width.}}
\label{tab:sigmaw}
\end{table}
By using the bremsstrahlung reconstruction in \davinci v30, the resolution of the \Bd mass can by increased about $13\ $\%. \\


\subsection{Effect of bremsstrahlung radiation and reconstruction on the \Bd mass shape}
Despite the implementation of the bremsstrahlung recovery tools, bremsstrahlung radiation affects the reconstructed \Bd mass shape. This can be seen in Figure \ref{fig:electronAndMuonBMass} where the reconstructed \Bd from \BdKstee is shown in direct comparison to the distribution of the \Bd mass from \BdKstmumu.\\
One effect is the large tail of the \Bd mass distribution from \BdKstee towards low \Bd masses. This tail originates from events whose bremsstrahlung photons could not be recuperated, resulting in a 
measured electron energy smaller than the initial electron energy $E^{measured}_{\epm} = (1 - E_{bs}) E^0_{\epm}$. \\
The second effect is the downgraded mass resolution which stems from the great uncertainty on the energy measurement of the calorimeter. In \lhcb the energy measurement of charged particle is performed by using combined information from the tracking system and the particle identification system. The tracking system determines the momentum of the charged particle $p_T^{track}$ with a relative accuracy of about $0.5 $\%. The particle identification system determines the identity of the particle which fixes the invariant mass of the particle track to the PDG value $m^{PDG}$. The particle's energy $E$ is then calculated from the combination of the measured momentum and the PDG \cite{pdg} \footnote{Particle Data Group.} mass.
\begin{equation}
E = \sqrt{p^2_{track} + m^2_{PDG}}
\end{equation}
The relative energy resolution is thus also of the order of $0.5 $\%.\\
For neutral particles however, the energy measurement has to be performed by the calorimeter. Since the relative energy resolution of the calorimeter is  
\begin{equation}
\frac{\sigma_E}{E} = \frac{10 \%}{\sqrt{E}} \oplus 1 \%
\end{equation}
the energy resolution of neutral particles is much worse than the energy resolution of charged particles.\\
Electrons, like muons, are charged particles and their tracks' transverse momenta $p_T^{track}(\epm)$ could be determined with an average relative accuracy of $0.5$\%. The energy of the bremsstrahlung photons however, has to be determined by the \ecal. The increase in uncertainty on the electron energy will propagate to the measured \Bd mass. \\
Figure \ref{fig:electronAndMuonBMass} shows the comparison between the \Bd mass shape of the \BdKstee and the \BdKstmumu.
\begin{figure}[ht]
  \begin{center}
  \subfigure{
  	\label{fig:eeKstar}\includegraphics[width=0.49\textwidth]{newBrem_crystalfit.jpg}}
  \subfigure{\label{fig:mumuKstar}\includegraphics[width=0.49\textwidth]{MC_Bmass_mumuKstar_TM.jpg}} 
  \vspace*{-1.0cm}
  \end{center}
  \caption{\textit{Comparison of the \Bd mass distribution from \BdKstee (left) and \BdKstmumu (right) reconstructed from \lhcb Monte Carlo samples. The distribution from  \BdKstee is much wider and shows a tail at low \Bd masses, while the distribution for \BdKstmumu is just a very narrow peak.}}
  \label{fig:electronAndMuonBMass}
\end{figure}


\section{$M_{inv}(\epem)$ reconstruction}
\label{sec:DiElectronMaker}
The correct reconstruction of the invariant mass of the $e^+ e^-$ pair in the \BdKstee decay is crucial for the measurement of the various amplitudes contributing to the decay since they depend on the $M_{inv}(\epem)$ (see Chapter \ref{chapter1}).\\
In \davinci v31 a new tool designed to make opposite and same sign electron pairs was introduced. This tool -- the so-called \dielectronmaker  \ -- was specially developed to reconstruct low invariant mass $e^+ e^-$ pairs, such as electrons and positrons from converted photon or light resonances.\\
\\
While usually electron pairs are being made by combining two reconstructed electrons, the \dielectronmaker accesses the raw detector information for the electron and the positron candidates directly. Based on the hits associated to the two candidates' tracks the \dielectronmaker creates a \textit{DiElectron object} first and only then computes the properties of the individual electron and positron respectively. The \dielectronmaker chooses tracks with a $p_T$ greater than $100 \mevc$ as electron candidates. It composes the \textit{DiElectron object} from opposite sign electron pair tracks and then applies the bremsstrahlung recovery tool from \davinci v30. In the case of double counting within one \textit{DiElectron object} the four-momentum of the photon candidate is added to one randomly chosen electron, similarly to the double counting correction in Section \ref{sub:doublecountingcorrection}. Then the four-momenta of the \textit{DiElectron object} and the electron and the positron are calculated and a $p_T(e^+ e^-)$ cut greater than $500 \mevc$ is applied.
From this \textit{DiElectron object} the rest of the event reconstruction follows as usual.\\
\\
The \dielectronmaker was created to increase the efficiency of reconstruction for events with low invariant mass $e^+ e^-$ pairs and also yields a more precise reconstruction of the \Bd mass as can be seen on the Monte Carlo sample in Figure \ref{fig:diElectron}. Additionally the selection developed for the 2011 dataset \cite{michellesthesis} is applied to the sample to estimate the effects on the final dataset. The results of the fit of the double Crystal-Ball distribution to the sample are listed in Table \ref{tab:disigmaw}, showing that the use of this new tool decreases the $\sigma^w$ even further than the reconstruction by \davinci v30  yielding a reduction of $27 \ $\% with respect to the reconstruction implemented in \davinci v29.
\begin{figure}[ht]
  \begin{center}
  \subfigure{\includegraphics[width=0.49\textwidth]{oldBrem_crystalfit.jpg}}
  \subfigure{\includegraphics[width=0.49\textwidth]{diElectron_crystalfit.jpg}} 
  \end{center}
  \caption{\textit{\Bd mass distribution obtained from \lhcb Monte Carlo with \davinci v29 and double counting correction (left) and the \dielectronmaker of \davinci v31 (right).}}
  \label{fig:diElectron}
\end{figure}

\begin{table}[hc]
\begin{center}
\begin{tabular}{c|c|c|c|c}
& $\sigma_1$ & $\sigma_2$ & $f$ & $\sigma^w$ \\
\hline
\davinci v29 &  47 \mevcc & 199 \mevcc & 0.76 & 83 \mevcc \\
\hline
\davinci v31 & 54 \mevcc &  306 \mevcc & 0.91 &  77 \mevcc \\
\end{tabular}
\end{center}
\vspace*{-0.5cm}
\caption{\textit{Relevant results of the fit of a double Crystal-Ball distribution to the \BdKstee \lhcb Monte Carlo reconstructed with \davinci v29 and the double counting correction and \davinci v31. $\sigma^w$ denotes the weighted width.}}
\label{tab:disigmaw}
\end{table}

Figure \ref{fig:deltaee} shows the resolution of the dilepton invariant mass obtained from \BdKstee Monte Carlo for a dilepton invariant mass between 20\mevcc and 30\mevcc and 30\mevcc and 50\mevcc respectively. The resolutions are computed for the Monte Carlo sample that was processed with each  \davinci v29 and \davinci v31.\\
The plots show that the histograms from \davinci v31 are better centred and their Root Mean Square (RMS) is reduced by 20\% to 30\%. The first effect originates from the newer Bremsstrahlung recovery algorithm that is implemented in the \dielectronmaker. This recovery algorithm identifies more bremsstrahlung photons than the one in \davinci v29 and adds them to their emitting electron. The use of the \dielectronmaker reduces the RMS of the histograms with respect to those made by \davinci v29 because of the intermediate step of computing the \textit{DiElectron object}.
\begin{figure}[!h]
\vspace*{-0.cm}
  \begin{center}
    \includegraphics[scale=0.6]{deltaee.jpg}
  \vspace*{-0.5cm}
  \end{center}
  \caption{\textit{Resolution of the dilepton invariant mass obtained from the \BdKstee Monte Carlo. \textbf{Left:} The histograms for the region of 20\mevcc < $M^{gen}_{inv}(\epem)$ < 30\mevcc. \textbf{Right:} The histograms for the region of 30\mevcc < $M^{gen}_{inv}(\epem)$ < 50\mevcc. The green histogram represents the resolution of the sample processed with \davinci v29, the pink histogram represents the resolution of the sample processed with \davinci v31.}}
  \label{fig:deltaee}
\end{figure}
\\
\vspace*{1.cm}
\section{Comparison of reconstruction efficiencies}
To evaluate the performance of $M_{inv}(\epem)$ reconstruction the three reconstruction tools implemented in \davinci v29, \davinci 30 and \davinci v31 are applied to the \BdKstee \lhcb Monte Carlo respectively. After this, the double counting correction algorithm is executed on the first two samples and the selection developed for the 2011 dataset \cite{michellesthesis} \cite{paper} is applied to all three samples to estimate the effects on the final datasets.\\
Figure \ref{fig:eff_diElectron} shows the overall efficiency for reconstruction and selection of the three samples in dependency of the true -- that is generated -- invariant \epem mass $M^{gen}_{inv}(\epem)$. Note that there is a cut on the reconstructed invariant \epem mass $M^{reco}_{inv}(\epem)>30 \mevcc$\footnote{The selection developed for the 2011 dataset \cite{michellesthesis} \cite{paper} includes a tighter cut on the invariant dilepton mass than the selection developed in the course of this master's thesis}.\\
Figure \ref{fig:eff_diElectron} shows that the use of the \dielectronmaker yields the most accurate results and the highest reconstruction efficiency, especially in the region of highest interest between $30 \mevcc$ and $65 \mevcc$. The amount of reconstructed events with a generated invariant mass below $30 \mevcc$ is zero up to almost $15 \mevcc$ and then smoothly increases due to multiple scattering.\\
Above $65 \mevcc$ the efficiencies for all three reconstruction algorithms converge to the same value and remain independent of the generated invariant \epem mass.\newpage
\begin{figure}[!h]
  \begin{center}
    \includegraphics[scale=0.6]{eff_diElectron.pdf}
  \vspace*{-0.5cm}
  \end{center}
  \caption{\textit{Absolute event reconstruction and selection efficiency depending on the generated $M^{gen}_{inv}(\epem)$. Obtained from \BdKstee Monte Carlo by comparing the $M^{gen}_{inv}(\epem)$ at generator level with the $M^{rec}_{inv}(\epem)$ distribution after reconstruction and selection. Green: \davinci v29, blue: \davinci v30, pink: \davinci v30 with \dielectronmaker .}}
  \label{fig:eff_diElectron}
\end{figure}
\vspace*{10cm}


%%%%%%%%%%%%%%%%%%%%%%%%%%%%%%%%%%%%%%%%%%%%%%%%%%%%%%%%%%%%%%%%%%%%%%%%%%%%%%%%%%%%%%%%%%%%%%%%%%%%%%%%%%%%
%Events with double counting were identified by comparing the energy of the Bremsstrahlung photon. If this energy was non zero and the same for the photon coming %from the electron and from the positron within $5 \mev$, the event was considers to be a double counting event.

%To compare the performance of the two n-tuples of \lhcb \BdKstee Monte Carlo are created with the different bremsstrahlung recovery algorithms respectively. Then %the double counting correction is applied to both n-tuples.\\

%\subsection{DaVinci v29r1p1 BremAdder}
%\label{sub:oldBrem}
%As a first step a NTuple was created using DaVinci v29r1p1 and its standard Bremsstrahlung recovery tool. This BremAdder matches photons to the electrons by lineraly extrapolating the
%electron track from its last state before the TT to the calorimeter. If the $\chi_{brem}^2$ of the extrapolated track and the photon position in the calorimeter is less than 300, this photon is
%considerd a bremsphoton and its energy is added to the electron.\\
%Complementing this BremAdder algorithm by the \textit{double counting- correction} yields the $B^0$ mass shape in \ref{fig:oldBremAdder}. The weighted sigma of this shape is $\sigma_{w} = $.\\
%\begin{figure}[ht]
%  \begin{center}
%    \includegraphics[scale=0.6]{figs/Bmass_oldBremdcc.pdf}
%  \vspace*{-1.0cm}
%  \end{center}
%  \caption{$B^0$ mass shape for the DaVinci v29r1p1 bremsstrahlung recovery tool and double- counting correction}
%  \label{fig:oldBremAdder}
%\end{figure}
%
%\subsection{DaVinci v31r0 BremAdder}
%\label{sub:newBremAdder}
%For the second procedure a NTuple was created using DaVinci v31r0 and its standard Bremsstrahlung recovery tool. This new BremAdder is an enhanced version of the BremAdder in \ref{sub:oldBrem}. It
%uses the StdAllLoosePhotons as bremsphoton candidates by default. A photon is matched as a bremsphoton if:
%\begin{itemize}
% \item its X- position in the calorimeter is between the extrapolated X- position of the electrons first state and its extrapolated X- position from the TT within $n_x \cdot \sigma_x$ (default:
%$n_x = 2$, $\sigma_x =$ quadratic sum of photon cluster spread and track extrapolation uncertainty)
%\item its Y- position in the calorimeter matches the extrapolated Y- position from the TT within $n_y \cdot \sigma_y$ (default: $n_y = 2$, $\sigma_y =$ quadratic sum of photon
%cluster spread and track extrapolation uncertainty).
%\end{itemize}
%Complementing this BremAdder algorithm by the \textit{double counting- correction} yields the $B^0$ mass shape in \ref{fig:newBremAdder}. The weighted sigma of this shape is $\sigma_{w} = $.\\
%\begin{figure}[ht]
%  \begin{center}
%    \includegraphics[scale=0.6]{figs/Bmass_newBremdcc.pdf}
%  \vspace*{-1.0cm}
%  \end{center}
%  \caption{$B^0$ mass shape for the DaVinci v31r0 bremsstrahlung recovery tool and double- counting correction}
%  \label{fig:newBremAdder}
%\end{figure}
%
%
%\subsection{DaVinci v32 DiElectronMaker}
%\label{sub:DiElectronMaker}
%For the DaVinci v31r0 a new tool to make $e^+ e^-$ pairs has been developed (also configurable to find same sign pairs). This Tool - called \textit{DiElectronMaker} - takes the electron ProtoParticles
%and creates low invariant mass $e^+ e^-$ pairs. The DiElectronMaker uses a method of the new BremAdder (from the DaVinci v31r0) which makes it possible to avoid the double counting. In the
%case of one bremsphoton candidate matching both electrons this method adds the bremsphoton to one randomly chosen electron - 
%The DiElectronMaker then selects $e^+ e^-$ pairs that pass
%
%
%
%
%\begin{figure}[ht]
%  \begin{center}
%    \includegraphics[scale=0.6]{figs/Bmass_diElectron.pdf}
%  \vspace*{-1.0cm}
%  \end{center}
%  \caption{$B^0$ mass shape for the DaVinci v31r0 DiElectronMaker}
%  \label{fig:diElectronMaker}
%\end{figure}
%\section{Effect of bremsstrahlung radiation on the \Bd mass shape}
%The emittance of bremsstrahlung photons has two different effects on the reconstructed \Bd mass shape.\\
%The first effect can be seen on the Monte Carlo sample in Figure \ref{fig:noBremReco}. If the bremsstrahlung photons are not recuperated the measured electron energy will be smaller than the initial electron energy $E^{measured}_{\epm} = (1 - E_{bs}) E^0_{\epm}$ and this offset will propagate to the reconstructed \Bd mass. Figure \ref{fig:noBremReco} shows the \Bd mass distribution obtained without any bremsstrahlung reconstruction. The distribution has an asymmetric, non Gaussian shape with a very large radiative tail in the low \Bd mass region. Since this tail diminishes the resolution on the \Bd mass and even exceeds the reconstructed mass region, it is crucial to recuperate the emitted bremsstrahlung photons and associate them to their electrons.\\
%This bremsstrahlung recovery causes the second effect on the measured \Bd mass distribution.
%In \lhcb the energy measurement of charged particle is performed by using combined information from the tracking system and the particle identification system. The tracking system determines the momentum of the charged particle $p_T^{track}(c.p.)$ with an accuracy of about $0.5 /$\%. The particle identification system determines the identity of the particle which fixes the invariant mass of the particle track to the PDG value $m_{PDG}(c.p.)$. The particle's energy is then calculated from the combination of the measured momentum and the PDG mass $E(c.p.) = \sqrt{p^2_{track}(c.p.) + m_{PDG}(c.p.)}$. This allows for an energy resolution of the order of the momentum resolution for charged particles.\\
%For neutral particles however, the energy measurement has to be performed by the calorimeter. Since the energy resolution of the calorimeter is much worse than the momentum resolution obtained by the tracking system, the energy resolution of neutral particles is much worse than the energy resolution of charged particles.\\
%Electrons, like muons, are charged particles and their tracks' transverse momenta $p_T^{track}(\epm)$ could be determined with an accuracy up to $0.35$\%. The energy of the bremsstrahlung photons however, has to be determined by the \ecal. The increase in uncertainty on the electron energy will be propagate to the measured \Bd mass. In particular, the \Bd mass distribution reconstructed from \BdKstee will show a greater width than the \Bd mass distribution reconstructed from \BdKstmumu for example.
%Figure \ref{fig:electronAndMuonBMass} shows the comparison between the \Bd mass shape of the \BdKstee and the \BdKstmumu.
%\begin{figure}[ht]
%  \begin{center}
%  \subfigure{
%  	\label{fig:eeKstar}\includegraphics[width=0.49\textwidth]{MC_Bmass_dielectron_TM.pdf}}
%  \subfigure{\label{fig:mumuKstar}\includegraphics[width=0.49\textwidth]{MC_Bmass_mumuKstar_TM.pdf}} 
%  \vspace*{-1.0cm}
%  \end{center}
%  \caption{\textit{Comparison of the \Bd mass distribution from \BdKstee (left) and \BdKstmumu (right) reconstructed from \lhcb Monte Carlo samples.}}
%  \label{fig:electronAndMuonBMass}
%\end{figure}

\chapter{Selection of the \BdKstee candidates}
\chaptermark{Chapter5}
\label{Chapter4}
In this chapter the procedure to select \BdKstee candidates is described. This builds the basis for the angular analysis.\\
First the previous \BdKstee analysis is presented and the differences to this analysis are outlined. Then the selection process for the \BdKstee candidates is explained beginning with the preselection and the training of a Boosted Decision Tree to the optimisation of the selection cuts.\\

\section{Previous and current \BdKstee analysis}
Previously a branching ratio measurement of the rare \BdKstee decay in the dilepton mass region from $30\mevcc$ to $1 \gevcc$ has been performed at \lhcb. This analysis was executed on $1 \invfb$ of data collected in 2011 at a center of mass energy of $7 \tev$. 
While the 2011 \lhcb data contained the largest sample of \BdKstee events ever collected this quantity was not enough to perform an angular analysis.  The selected \BdKstee candidates can be seen in Figure \ref{fig:oldsig}.
Before this analysis the only experiments to observe the \BdKstee decay to date are \babar \cite{babar} and \belle \cite{belle}. Each have collected about 30 \BdKstll events in the region of $q^2 < 2\gevcc$ summing over electron and muon final states.\\
\\
The branching ratio measurement of \BdKstee at \lhcb was executed using the \BdToJPsieeKst decay as a normalisation channel and the results were published in the Journal of High Energy Physics \cite{paper}. The measured branching ratio is
\begin{equation}
\BR(\BdKstee)^{30 -1000\mevcc} = (3.1^{+0.9 \ +0.2}_{-0.8 \ -0.3} \pm 0.2) \, 10^{-7}
\end{equation}
where the first uncertainty is statistical, the second systematic, and the third comes from the uncertainties on the branching ratio of the normalisation channel \BdToJPsieeKst. The selected \BdKstee candidates from this analysis can be seen in Figure \ref{fig:oldsig}. As expected, this measurement is in agreement with SM predictions of 
\begin{equation}
\BR(\BdKstee)^{30 -1000\mevcc}_{pred} = 2.43^{+0.66}_{-0.47}\, 10^{-7} \quad \cite{jaeger}.
\end{equation}

\begin{figure}[ht]
\begin{center}
\subfigure{\includegraphics[width = 0.49\textwidth]{L0Ele.pdf}}
\subfigure{\includegraphics[width = 0.49\textwidth]{L0TIS.pdf}}
\end{center}
\caption{\textit{The \BdKstee candidates obtained from the analysis performed on 1\invfb of \lhcb data collected in 2011 for the two independent trigger categories (see Section \ref{sec:triggercat}).}\cite{paper}}
\label{fig:oldsig}
\end{figure}
While the previous analysis yielded very good results the developed selection was not optimised for 
%selecting \BdKstee candidates for 
an angular analysis on the 2011 and 2012 dataset. The reasons are differing requirements for an angular analysis and several changes in the data-collecting and reconstruction process which will be explored in the following.

\subsection{Selection requirements for an angular analysis}
\label{sec:req}
The selection requirements for an angular analysis are different than those for a branching ratio measurement. 
For the angular analysis it is especially important to have good control over the angular acceptances for the three angles $\theta_l$, $\theta_K$ and $\Phi$ by designing a selection that is as unbiased as possible for these three parameters. Furthermore, the branching ratio measurement was made with respect to the \BdToJPsieeKst as normalisation channel. In that analysis it was important to keep systematic effects between the \BdKstee and the \BdToJPsieeKst channel at a minimum which lead do a different prioritising of the trigger lines.  \\


\subsubsection{Biasing the \ctl acceptance: the $\Bd \rightarrow \Dm \ep \neu$ veto cut}
\label{sec:denu}
The previous analysis used a cut to remove the specific background from the $\Bd \rightarrow \Dm \ep \neu$ decay with the \Dm decaying into \en \Kstarz and \neu. For the $\Bd \rightarrow \Dm \ep \neu $ to pass the \BdKstee selection the neutrinos have to be of very low momentum so that the \Kstarz and the \en share almost the total \Dm energy. To remove the $\Bd \rightarrow \Dm \ep \neu $ background, the invariant mass $m(\Kstarz e)$ of the \Kstarz (\Kstarzb) and the \en (\ep) is calculated and a cut of $m(\Kstarz e) > 1900 \mevcc $ is applied \footnote{$m(\Dm) = 1869 \mevcc$\cite{pdg}}.\\
Unfortunately this cuts biases the \ctl distribution as can be seen in Figure \ref{fig:ctl}. The low momentum neutrinos demand the \Kstarz and the \en to be almost back to back in the rest frame of the \Dm, just as the \Dm and the \ep in the rest frame of the \Bd. This gives the \ep a relatively large energy of $\sim 2 \gev$. For the dielectron invariant mass to be still smaller than $1 \gev$ the \en must be in the opposite direction of the \ep. Therefore the cut against the $\Bd \rightarrow \Dm \ep \neu$ background removes events where \ctl is at high values as illustrated in Figure \ref{fig:ctl}.\\
It is still very important to remove the $\Bd \rightarrow \Dm \ep \neu$ background since the branching ratio for this decay is about four orders of magnitude greater than the branching ratio of the signal decay. To obtain better control over the way the $\Bd \rightarrow \Dm \ep \neu$ veto cut impacts the \ctl distribution the cut on the $m(\Kstarz e)$ is removed and a direct cut $\ctl \, < \, 0.8$ is applied. As can be seen in Figure \ref{fig:kstaremass} the cut on \ctl has almost the same effect on the $m(\Kstarz e)$ distribution as the direct $m(\Kstarz e)$ cut.
\begin{figure}[!h]
\begin{center}
\includegraphics[width = 0.5 \textwidth]{ctl.pdf}
\end{center}
\vspace*{-0.5cm}
\caption{\textit{Illustration of the impact of the $m(\Kstarz e)$ cut on the acceptance of \ctl. Distributions obtained from \BdKstee Monte Carlo that was generated flat in \ctl. Green distribution: \ctl distribution after the stripping, blue dashed distribution: \ctl distribution after the stripping and the  $m(\Kstarz e) > 1900 \mevcc $ cut, pink distribution: events that are removed by the $m(\Kstarz e)$ cut.}}
\label{fig:ctl}
\end{figure}

\begin{figure}[!h]
\begin{center}
\subfigure{\includegraphics[width=0.49\textwidth]{kstaremass.pdf}}
    \subfigure{\includegraphics[width=0.49\textwidth]{mckstaremass.pdf}}
\end{center}
\vspace*{-0.8cm}
\caption{\textit{The distribution of the $m(\Kstarz e)$ variable for the \BdKstee candidates after the stripping. Right: \lhcb data from 2011 and 2012, left: \BdKstee Monte Carlo. Pink distribution: candidates that will be removed by the $\ctl \, < \, 0.8$.}}
\label{fig:kstaremass}
\end{figure}
\\
\subsubsection{The mutually exclusive trigger categories}
\label{sec:triggercat}
Due to the different systematics for events in different trigger categories the \BdKstee analysis was and will be performed separately for different independent trigger categories. For the 2011 analysis this categories were the \textit{L0TIS} and the \textit{L0Electron without L0TIS}.\\
The \textit{L0TIS} (L0 Trigger Independent of Signal) candidates are the events where the L0 is triggered by a track that does not come from the signal tracks but from any other track in the detector. The candidates in any \textit{L0TOS} (L0 Trigger On Signal) category are the events where one of the tracks from the \BdKstee candidate enabled the L0 trigger. The candidates that are \textit{L0ElectronTOS} are those that enabled the L0Electron trigger.\\
Obviously some candidates can be L0TIS and L0TOS at the same time, therefore a hierarchy between the trigger categories has to be defined.
The 2011 order was chosen because the \textit{L0TIS} category is obviously independent of the signal decay while the \textit{L0ElectronTOS without L0TIS} induces some systematic differences due to the different momentum distributions of the electrons from \BdKstee and \BdToJPsieeKst. \\
For the angular analysis, the trigger categories were chosen to the:
\begin{itemize}
\item \textbf{L0ElectronTOS}: signal candidate has triggered the L0Electron line
\item \textbf{L0HadronTOS} without L0ElectronTOS: signal candidate has triggered the L0Hadron line but not the L0Electron line
\item \textbf{L0TIS} without L0ElectronTOS and L0HadronTOS: any L0 line was triggered by a track in the event that did not belong to the signal candidate
\end{itemize}
This order was chosen because the events that were triggered by one of the signal candidates (the TOS events) have a cleaner signature as can be seen in Section \ref{sec:fitstrat}.\\
No kind of \textit{L0Hadron} category was used in the 2011 selection because no significant signal could be obtained on the dataset in this category.\\


\subsection{The trigger configurations}
Due to the increase in center of mass energy from $7\tev$ in 2011 to $8\tev$ in 2012, the L0 trigger settings had to be adapted to cope with the increase in the rate of visible interactions and their thresholds were changed several times in 2012. The \BdKstee analysis is most affected by the change in the L0Electron line because of the very low $p_T$ electrons. The threshold energies for the L0Electron trigger line in 2011 and 2012 are listed in Table \ref{tab:loele}. 
\begin{table}[ht]
\begin{center}
\begin{tabular}{c|c|c}
Year & $E_T > $ [\mev] & Recorded luminosity [\invfb]\\
\hline
\hline
2011 & 2500 & 1 \\
\hline 
2012 & 2500 & 0.066 \\
& 2720 & 1.166 \\
& 2860 & 0.260\\
& 2960 & 0.571 \\
\end{tabular}
\caption{\textit{Threshold energies for the L0Electron Trigger in 2011 and 2012.}}
\label{tab:loele}
\vspace*{-0.5cm}
\end{center}
\end{table}

The L0Electron trigger efficiencies depending on the electron transverse energy for the three highest threshold energies are shown in Figure \ref{fig:trigger}. The trigger efficiency is calculated on \BdToJPsieeKst \lhcb data using the \textit{TisTos} method (TIS: Trigger Independant of Signal, TOS: Trigger On Signal) \cite{lopez}. The \BdToJPsieeKst candidates undergo a standard selection and one of the electrons (tag track) must pass a strict selection. The L0Electron efficiency is then calculated using the other electron (probe track). The trigger efficiency depending on the transverse energy of the probe track $E_T$ is calculated by:
\begin{equation}
\epsilon_{L0Ele}(E_T) = \frac{N_{TOS \& TIS}(E_T)}{N_{TIS}(E_T)}
\end{equation}
where $N_{TOS \& TIS}$ denotes the amount of events that are both L0TIS and the probe track triggered the L0Electron and $N_{TIS}$ denotes the events that are L0TIS.

\begin{figure}[ht]
\begin{center}
\includegraphics[width = 0.7 \textwidth]{triggereff.jpg}
\end{center}
\vspace*{-0.5cm}
\caption{\textit{The L0Electron efficiency depending on the electron transverse energy for the three different trigger settings.}}
\label{fig:trigger}
\end{figure}
Note that the curves in Figure \ref{fig:trigger} gradually increase around the threshold energy instead of showing a sharp step. This is due to the fact that the energy that triggers the L0Electron is not exactly equal to the reconstructed electron energy. In the reconstruction the electron energy is determined by examining clusters of $3\times 3$ cells in the \ecal while the readout for the L0Electron is performed on clusters of $2\times 2$ cells.\\
The threshold for the L0Hadron trigger was stable at 3.5\gevcc.\\

\subsection{The \BdKstee stripping line}
\label{sec:strips}
A new \BdKstee stripping line was developed after the 2011 analysis. While the 2011 stripping line (\textit{Stripping17b}) was purely cut-based, the new stripping line (\textit{Stripping20}) contains a Boosted Decision Tree (see Section \ref{sec:bdt}) selection and implements the newly developed DiElectronMaker that was introduced in Chapter \ref{chapter3}. After the processing of the 2012 \lhcb data with the \textit{Stripping20} the entire 2011 dataset was reprocessed with \textit{Stripping20} also.\\
\\
The \textit{Stripping20} \bdtn was trained on \BdKstee Monte Carlo as signal sample and background sample was taken from a 2011 \lhcb sample containing mostly hadronic decays. The input variables for the \bdtn are listed in Table \ref{tab:stripbdt} and are very similar to those used in the previous cut-based stripping (for details on \textit{Stripping17b} see Appendix \ref{ap:strip17}), namely the transverse momentum $p_T$ of all seven particles, the impact parameter significance $\chi^2_{(IP)}$ of all seven particles, the flight distance significance $\chi^2_{(FD)}$ of the \Bd, the \Kstarz and the intermediate (\epem) and the angle between the direction of the \Bd candidate momentum and the direction between the primary vertex and the \B decay vertex $\theta_{flight}$. The list of the variables used in the \textit{Stripping20} \bdtn are listed in Table \ref{tab:stripbdt}. \newpage

 \renewcommand{\arraystretch}{1.5} 
\begin{table}[ht]
\begin{center}
\begin{tabular}{c|l}
Particle & Variable \\
\hline
\hline
 \Bd & $p_T(\Bd)$,\qquad $\chi^2_{IP}(\Bd)$,\qquad \ \ $\chi^2_{FD}(\Bd)$,\qquad $\theta_{flight}$ \\
\hline 
\Kstarz & $p_T(\Kstarz)$,\qquad $\chi^2_{IP}(\Kstarz)$,\qquad \ $\chi^2_{FD}(\Kstarz)$ \\
\hline 
(\epem) & $p_T(\epem)$,\qquad $\chi^2_{IP}(\epem)$,\qquad $\chi^2_{FD}(\epem)$ \\
\hline
\kaon & $p_T(\kaon)$,\qquad $\chi^2_{IP}(\kaon)$ \\
\hline
\pion & $p_T(\pion)$,\qquad $\chi^2_{IP}(\pion)$ \\
\hline
\ep & $p_T(\ep)$,\qquad $\chi^2_{IP}(\ep)$ \\
\hline
\en & $p_T(\en)$,\qquad $\chi^2_{IP}(\en)$ \\
\end{tabular}
\caption{\textit{Variables used in the Boosted Decision Tree for the Stripping20 \BdKstee line.}}
\label{tab:stripbdt}
\end{center}
\end{table}

The efficiency of the \textit{Stripping20} with respect to the efficiency of the \textit{Stripping17b} is calculated on the \BdKstee Monte Carlo. The ratio of efficiencies $r$ is $r = \frac{\epsilon_{Strip20}}{\epsilon_{Strip17b}} = 1.40$ and it can be seen from Figure \ref{fig:stripeff} that $r$ is particularly high for small invariant dielectron masses. This is important because the low $q^2$ events carry the most information about the photon polarisation as was discussed in Chapter \ref{chapter1}.
\begin{figure}[ht]
\begin{center}
\includegraphics[width = 0.5\textwidth]{stripeff.pdf}
\end{center}
\vspace*{-0.5cm}
\caption{\textit{Ratio of efficiencies of the Stripping20 line with respect to the Stripping17b line for different bins of dielectron invariant mass.}}
\label{fig:stripeff}
\end{figure}
\\
The use of the \textit{Stripping20} line also implies the use of the \dielectronmaker (see Chapter \ref{chapter3}). As shown in Section \ref{sec:DiElectronMaker} this tool increases the resolution of the invariant dilepton mass. Therefore the lower cut on the invariant dilepton mass that was previously taken to be $M_{inv}(\epem)>30 \mevcc$ is now lowered to \\ $M_{inv}(\epem)>20 \mevcc$ without loss in precision.\newpage


\section{The Boosted Decision Tree selection}
Due to these changes the 2011 selection is not adapted to optimally select the \BdKstee candidates on the 2011 and 2012 data set. Therefore a completely new selection is developed and optimised.\\
In order to select the rare \BdKstee candidates a multivariate analysis (MVA) is used. The MVA is specially powerful in rejecting the so-called \textit{combinatorial background}, that is background that results from the false combination of tracks in the detector. The MVA chosen for this analysis is the Boosted Decision Tree (\bdtn) with gradient boost.\\
In this section the BDT is explained. Furthermore the data samples to train the \bdtn and the training strategy is summarised. At the end of this section the optimisation of the cut on the \bdtn classifier is shown.\\

\subsection{Boosted Decision Tree: a multivariate method}
\label{sec:bdt}
To separate the interesting signal events from the surplus of background events a set of $n$ discriminative variables is used. Classically each of these variables is examined independently of the others and a cut for each variable is found that rejects most of the background while keeping as much signal as possible. The total selection is then a rectangular set of $n$ cuts. \\
The multivariate analysis methods on the other hand combine the information on the different discriminate variables into one single classifier. The selection has then a linear or non-linear shape in the $n$-dimensional space of the variables. This principle is illustrated in Figure \ref{fig:zandra} \cite{zandri}.\\
\begin{figure}[ht]
\begin{center}
\vspace*{-0.5cm}
\includegraphics[width = 0.85 \textwidth]{zandra.pdf}
\end{center}
\caption{\textit{Illustration of the multivariate analysis for two variables $x_i$ and $x_j$ and two data types $H_0$ and $H_1$ (for example signal and background). The left plot shows a set of rectangular cuts as used in a classic selection. The middle and the right plot represent a selection from a multivariate analysis where the combination of variables is used to find the optimal selection. The middle plot shows a linear discriminant (Fischer discriminant) and the right plot shows a non-linear discriminant (Boosted Decision Tree, Neural Networks, etc.).}\cite{zandri}}
\label{fig:zandra}
\end{figure}

There are different multivariate methods of combining the variables to the final classifier and one of them is the Boosted Decision Tree. The \bdtn is a weighted sum of $m$ simple decision trees, called 'basic classifiers' or 'weak learners'. Each decision tree classifies a given event with $n$ variables $\mathbf{x}$ as either background or signal with the output of $f(\mathbf{x}) = -1$ and $f(\mathbf{x}) = 1$ respectively. The response $F(\mathbf{x})$ of the final \bdtn -- called the boosted event classification -- is then
\begin{equation}
F(\mathbf{x}) =\frac{1}{m} \sum_{i=0}^m \ln(\alpha_i) f_i(\mathbf{x})
\end{equation}
where the $\alpha_i$ represent the weights associated to the $i$th tree. Small values of $F(\mathbf{x})$ indicate that a certain events is more background-like while great values indicate a more signal-like structure.\\
\\
The \bdtn is trained on a sample of events that each have a label $y(\mathbf{x})$, meaning that they are already classified as signal or background.
During the training process, each decision tree aims at minimizing the weighted misclassification rate $err$. When training the first tree all events are given the same weight $\alpha = 1$. The subsequent tree $i$ is trained on a modified event sample where the previously misclassified events are given a weight derived from the previous misidentification rate $err_{(i-1)}$
\begin{equation}
\alpha_{i} = \frac{1 - err_{(i-1)}}{err_{(i-1)}}
\end{equation}
The entire sample is then renormalised such that the sum of the weights over all events remains normalised to 1.\\

\subsubsection{Gradient Boost}
The \bdtn used in this analysis implements the \textit{GradientBoost} method. This method is particularly good for selection of \BdKstee candidates since it performs very well in noisy environment, that is environment where some background events tend to look like signal. This is due to the fact that the loss-function (Equation \ref{eg:l}) varies smoothly and does not over-penalise misclassified events. As a consequence the \textit{GradientBosst} is very robust with respect to overtraining.\\
The \textit{GradientBoost} is a specialisation of the \bdtn. It works by implementing a \textit{loss-function} $L(F,y)$ which represents the deviation between the BDT response $F(\mathbf{x})$ and the true label $y(\mathbf{x})$ of a certain event. Furthermore, all events are given an individual weight corresponding to their loss-function. \\
The \textit{GradientBoost} implements a binomial log-likelihood loss
\begin{equation}
L(F,y) = \ln(1+e^{-2F(\mathbf{x})y})
\end{equation}
From here the average loss over the whole training sample with $k$ events can be calculated as
\begin{equation}
\langle L \rangle^{i} = \sum_{l=0}^k \omega^i_l L^i_l 
\label{eg:l}
\end{equation}
which is the analogon to the misclassification rate $err$. From $\langle L \rangle^{i}$ the boosting coefficient $\alpha_i$ for the $i$th tree can be computed
\begin{equation}
\alpha_i = \frac{\langle L \rangle^{i}}{1-\langle L \rangle^{i}}
\end{equation}
This boosting factor is used to extract the weight $\omega^i_k$ for each event $k$ in the $i$th step
\begin{equation}
\omega^i_k = \omega^{(i-1)}_k \cdot \alpha_i^{1-L^{(i-1)}_k}
\end{equation}
By minimizing the loss-function $ L^i(F(\mathbf{x}), y) $ in each step, that is for each decision tree, the total \bdtn is optimised.\\

\subsection{Strategy}
\label{sec:strategy}
The TMVA4 package implemented in \root is used for this analysis. As mentioned above all \bdtn were chosen to be of the type \textit{GradientBoost}.\\
The \bdtn requires training samples for the background and the signal. Unfortunately, the training samples cannot be used later on in the analysis because this would bias the selection. The \BdKstee decay has already a very small branching ratio and due to the low $p_T$ electrons the reconstruction efficiency is also low. It is crucial to keep as many events as possible for the analysis. However, the training sample for the \bdtn has to be large enough to compensate for statistical fluctuations.\\
To allow for the use of the entire data sample in the analysis while having enough statistics to train the \bdtn a strategy using two \bdts \ -- \bdta and \bdtb \ -- is developed. \bdta and \bdtb have the exact same properties the only difference being the data samples they are trained on. For the training the data sample is divided into four independent subsamples of equal size as shown in Figure \ref{fig:samples}. The sample A1 is used to train the \bdta and B1 is used to train \bdtb . The testing and the optimisation is executed on A2 and B2 for \bdta and \bdtb respectively. \\
\begin{figure}[ht]
\begin{center}
\vspace*{-0.5cm}
\includegraphics[width = 0.8\textwidth]{BDTstrategy.jpg}
\end{center}
\caption{\textit{Schematic illustration of the division of the data sample into four subsamples to train the \bdts and optimise the cut on \bdta and \bdtb.}}
\label{fig:samples}
\end{figure}

For the final selection, the \bdta is applied on the data sample B and the cut that was optimised on the data sample A2 is performed. In return the \bdtb and its cut optimised on the B2 sample are applied to the entire data sample A. Hereby is ensured that the entire data sample can be used in the analysis without any bias from the training procedure.\\
While this procedure has been developed for the \lhcb data sample specifically, it is also applied on the Monte Carlo signal sample to ensure equal treatment of the background and signal sample.\newpage

\subsection{The input variables}
\label{sec:variables}
The variables for the \bdtn must have discriminative power between the combinatorial background and the \BdKstee signal.\\
The tracks originating from \Bd mesons have a distribution of transverse momentum $p_T$ that is greater than for tracks that do not come from a \Bd meson. This is due to the fact that \Bd mesons are relatively heavy. Due to the large \Bd lifetime, the impact parameter significance with respect to the primary vertex of the tracks originating from the \Bd decay is much larger than those of tracks from combinatorial background. Furthermore the quality of the vertices in events that are combinatorial background is lower because the tracks do not really originate from the same vertices. Therefore the interesting variables are the significance of the impact parameter $\chi^2_{IP}$ for all particles, the significance of the decay vertices $\chi^2_{Vertex}$ of the \Bd, the \Kstarz and the (\epem), the flight distance significance $\chi^2_{FD}$ of the \Bd, the \Kstarz and the (\epem) and the $\theta_{flight}$\footnote{As mentioned in the previous section, $\theta_{flight}$ is the angle between the direction of the \Bd candidate momentum and the direction between the primary vertex and the \B decay vertex.} of the \Bd meson. All input variables are listed in Table \ref{tab:newbdt} and a selection of these variables is plotted in Figure \ref{fig:variables} for background and signal respectively (plots of all variables can be found in Appendix \ref{ap:allvars}).\\
\begin{figure}[ht]
\vspace*{-0.5cm}
\begin{center}
\centering
\subfigure{\includegraphics[width = 0.45 \textwidth]{kaonpt.jpg}}
\centering
\subfigure{\includegraphics[width = 0.45 \textwidth]{kaonipchi2.jpg}}
\subfigure{\includegraphics[width = 0.45 \textwidth]{kstarfdchi2.jpg}}
\subfigure{\includegraphics[width = 0.45 \textwidth]{kstarendvertex.jpg}}
\end{center}
\caption{\textit{A choice of variables used in the \bdtn training. The red distribution represents the combinatorial background while the blue distribution comes from the signal. \textbf{Top left:} \kaon transverse momentum $p_T(\kaon)$, \textbf{top right:} logarithm of \kaon impact parameter significance $log(\chi^2_{IP}(\kaon))$, \textbf{bottom left:} logarithm of \Kstarz flight distance significance $log(\chi^2_{FD}(\Kstarz))$, \textbf{bottom right:} logarithm of \Kstarz vertex quality $log(\chi^2_{Vertex}(\Kstarz))$. }}
\label{fig:variables}
\end{figure}

\begin{table}[ht]
\vspace*{-0.5cm}
\begin{center}
\begin{tabular}{c|l}
Particle & Variable \\
\hline
\hline
 \Bd & $p_T$,\qquad $log(\chi^2_{IP})$,\qquad $log(\chi^2_{FD})$,\qquad $log(\chi^2_{Vertex})$,\qquad $\theta_{flight}$ \\
\hline 
\Kstarz & $p_T$,\qquad $log(\chi^2_{IP})$,\qquad $log(\chi^2_{FD})$,\qquad $log(\chi^2_{Vertex})$ \\
\hline 
(\epem) & $p_T$,\qquad $log(\chi^2_{IP})$,\qquad $log(\chi^2_{FD})$,\qquad $log(\chi^2_{Vertex})$ \\
\hline
\kaon & $p_T$,\qquad $log(\chi^2_{IP})$ \\
\hline
\pion & $p_T$,\qquad $log(\chi^2_{IP})$ \\
\hline
\ep & $p_T$,\qquad $log(\chi^2_{IP})$ \\
\hline
\en & $p_T$,\qquad $log(\chi^2_{IP})$ \\
\hline
& number of primary vertices $nPV$
\end{tabular}
\caption{\textit{Variables used in the Boosted Decision Trees \bdta and \bdtb .}}
\label{tab:newbdt}
\end{center}
\end{table}
\vspace*{-0.5cm}
Additionally the number of primary vertices $nPV$ is used as variable in the training. It can be shown, that the distributions of some of the above variables differ for the background depending on $nPV$. Figure \ref{fig:npvvariation} shows an example. This is due to the fact, that events with $nPV = 1 $ are very likely to have come from a real \B meson in order to pass the stripping. Therefore the distribution of the discriminant variables for the background is more similar to the distribution from the signal. Events with several primary vertices are more likely to form combinatorial background that passes the selection due to the track multiplicity in the detector. Since these events do not come from a real \B meson, their distributions in the discriminant variables differ more from the signal. Including $nPV$ in the analysis allows to take into account the differing distributions of the other variables and the fact that events with high $nPV$ have a greater probability to be combinatorial background.\\
\begin{figure}[!h]
\vspace*{-0.5cm}
\begin{center}
\subfigure{\includegraphics[width=0.48\textwidth]{nPV_eeKstar.pdf}}
\subfigure{\includegraphics[width=0.51\textwidth]{nPV_Kstar_FDCHI2.jpg}}
\end{center}
\vspace*{-0.8cm}
\caption{\textit{\textbf{Left:} Normalized distribution of the number of primary vertices for the signal (pink) and the combinatorial background (blue). \textbf{Right:} Illustration of how the distribution of a variable varies with the number of primary vertices. The plot shows the normalized distributions of the logarithm of the \Kstarz flight distance significance. Black distribution: nPV = 1, blue distribution: nPV = 2, violet distribution: nPV>2. These distributions have been obtained from a sample for combinatorial background from \lhcb data.}}
\label{fig:npvvariation}
\end{figure}
Despite their discriminate power the particle identification variables $\mathrm{DLL}$ (see Section \ref{sec:pid}) are not included in the training as their distributions are not well reproduced in the Monte Carlo. They will, however, later be used in the optimisation of the selection.\\

\subsection{The data samples for the \bdtn training}
In this section the data samples used in the \bdtn training are discussed. The sample representing the background is taken from \lhcb data while the signal sample comes from \BdKstee Monte Carlo. Both samples have been processed by the \BdKstee \textit{Stripping20} line and reconstructed in \davinci v33 including the DiElectronMaker (see Chapter \ref{chapter3}). After the stripping a rough preselection that will be discussed in the next section is applied before training the \bdtn.\\
The statistics of the training samples is limited by the available amount of background \lhcb data. This results in training samples of the size of $\approx$3300 events for each \bdtn and for signal and background respectively.\\

\subsubsection{The preselection}
\label{sec:presel}
The preselection is a set of cuts dedicated to removing events that have a very small probability of being \BdKstee candidates. Additionally to the cuts listed in Table \ref{tab:presel} the candidates are asked to be either in the \textit{L0ElectronTOS}, \textit{L0HadronTOS} or \textit{L0TIS} category (see Section \ref{sec:req} for definition) and were triggered by the \textit{Hlt1TrackAllL0Decision} and one of the the \textit{\hlttwo topological lines} (see Section \ref{sec:hlt} for more details).\\
\begin{table}[ht]
\vspace*{-0.5cm}
\begin{center}
\begin{tabular}{c|c}
Particle & Cut \\
\hline
\hline
\Kstarz & $| m(\Kstarz ) -892 \mevcc | \, < \, 100 \mevcc$ \\
\hline
(\epem) & $20 \mevcc \, < \, m(\epem) \, < 1000 \mevcc $\\
\hline
\kaon & $\dllkpi\, > \, 0$ \\
\hline
\pion & $\dllkpi\, < \, 5$ \\
\hline
\epm & $\dllepi \, >\, 0$\\
\end{tabular}
\caption{\textit{Preselection cuts for the \BdKstee analysis.}}
\label{tab:presel}
\end{center}
\end{table}
\vspace*{-0.4cm}
\subsubsection{The \BdKstee signal sample}
\label{sec:splot}
The signal sample for the \bdtn is taken from \BdKstee Monte Carlo. Unfortunately the distribution of number of primary vertices ($nPV$) is not well simulated in the Monte Carlo. To obtain a accurate training of the \bdtn the distribution of $nPV$ has to be reweighted to match the distribution of $nPV$ of the \BdKstee signal on \lhcb data. Therefore the $nPV$ distribution of the signal on \lhcb data has to be extracted and compared to the Monte Carlo $nPV$ distribution. This can be done by using a statistical tool called the the sPlot technique \cite{splot}. \\
\newpage
\textbf{The sPlot technique}\\
The sPlot technique can extract the contribution of the signal to the distribution of a variable $x$ when the signal is not clearly separable from the background. First the distribution of the variable $x$ in the signal window (where signal and background are mixed) is extracted. Then the distribution of $x$ in the region where only background contributes is taken. By calculating the contribution for the background in the signal window, the background-only distribution of $x$ can be subtracted from the signal-background distribution, leaving the distribution of the variable $x$ for the signal decay only.\\
\\
\textbf{Implementation of the sPlot technique}\\
There are not enough signal events in the \BdKstee data sample from 2011 and 2012 to apply the sPlot technique. The \BdToJPsieeKst decay on the other hand has a $ \BR(\BdToJPsieeKst) = 1.34 \cdot 10^{-3} $ which is two orders of magnitude higher and and provides enough statistics to use the sPlot tool implemented in \root.\\
In general the $nPV$ distribution should be independent of the specific decay. But the selection procedure can distort the distribution, particularly there is a cut on the hit multiplicity in the \spd (Section \ref{sec:lhcb}) in the L0 trigger lines. This relates to the number of primary vertices. Nevertheless it is convenient that the \BdToJPsieeKst decay shows the same kinematics as the \BdKstee decay.\\
To extract the correct $nPV$ distribution the \BdToJPsieeKst candidates experience the same reconstruction and selection procedure as the \BdKstee candidates. This is done by taking the output of the \BdKstee \textit{Stripping20} line\footnote{The \BdKstee \textit{Stripping20} line was developed to reconstructed \BdKstee candidates with low $m(\epem)$ but there is no explicit cut on $m(\epem)$ and \BdToJPsieeKst candidates are reconstructed as well.} and applying the same preselection cuts as above with the difference of performing a cut on the mass of the \jpsi of $2400 \mevcc \, < \, m(\jpsi) \, < 3400 \mevcc $.\\
Using the \BdToJPsieeKst decay has another advantage because the \jpsi is a narrow resonance. For the sPlot technique the mass of the \jpsi is fixed to the PDG value $m(\jpsi) = 3096 \mevcc$. From there the electron momenta are recalculated and finally the \Bd mass is recalculated. The fixing of the \jpsi mass yields a more narrow signal peak as can be seen in Figure \ref{fig:jpsi}.\\
The sPlot tool in \root is given the \BdToJPsieeKst candidates along with a probability distribution function (\PDF) for the signal and the background respectively. The signal \PDF was taken to be a double Crystal-Ball distribution (CB) \cite{crystal} with both CBs sharing the same $\mu_{\B}$, $\alpha$ and $n$. All variables of the signal \PDF are determined by a fit to the \BdToJPsieeKst Monte Carlo, except for the \Bd mass. The widths $\sigma_1$ and $\sigma_2$ are free to scale about the same factor. The background shape was assumed to be exponential. The fit to the constrained \Bd mass can be seen in Figure \ref{fig:jpsi}.\\
\\
The extracted $nPV$ distribution is shown in Figure \ref{fig:npv} along with the distribution from Monte Carlo and the reweighted Monte Carlo distribution. From here the $nPV$ distribution from \BdKstee Monte Carlo is reweighted in the same way.
\begin{figure}[ht]
\vspace*{-0.5cm}
\begin{center}
\includegraphics[width = 0.5 \textwidth]{mB_Jpsi.pdf}
\end{center}
\vspace*{-0.8cm}
\caption{\textit{The $m(e^{+}e^{-}K^{*0})$ mass distribution of the 2011 and 2012 \lhcb data from the Stripping20 \BdToJPsieeKst line after preselection (Section \ref{sec:presel}). The black distribution is the normal \Bd mass distribution, the pink distribution is with the constrained \jpsi mass fixed to the PDG value.}}
\label{fig:jpsi}
\end{figure}

\begin{figure}[ht]
\begin{center}
\includegraphics[width = 0.5 \textwidth]{nPV.pdf}
\end{center}
\vspace*{-0.8cm}
\caption{\textit{The $nPV$ distribution for the \BdToJPsieeKst candidates. Black points: signal from \lhcb data, red distribution: Monte Carlo, blue distribution: reweighted Monte Carlo.}}
\label{fig:npv}
\end{figure}


\subsubsection{The background sample}
The \bdtn is very effective against combinatorial background and should hence be trained on combinatorial background only. Therefore the data for the background is taken to be the upper sideband of the 2011 and 2012 data sample as shown in Figure \ref{fig:sb}.
\begin{figure}[ht]
\vspace*{-0.5cm}
\begin{center}
\includegraphics[width = 0.4 \textwidth]{upperSideBand.pdf}
\end{center}
\vspace*{-0.8cm}
\caption{\textit{The $m(e^{+}e^{-}K^{*0})$ mass distribution of the 2011 and 2012 \lhcb data from the \BdKstee Stripping20 line after preselection (Section \ref{sec:presel}). The pink distribution represents the candidates used in the training of the \bdts.}}
\label{fig:sb}
\end{figure}

\subsection{Results of the \bdtn trainings}
The response of \bdta and \bdtb after the training is shown in Figure \ref{fig:BvsS} and Figure \ref{fig:resp}. Figure \ref{fig:BvsS} shows the background rejection efficiency as a function of the signal selection efficiency obtained from applying the trained \bdts on the testing samples.\\
Figure \ref{fig:resp} shows the response of the trained \bdts on the signal and the background. It can be seen from the accordance of the distributions of the testing and the training samples that the \bdts were not overtrained. Furthermore it was tested if the response of the \bdts on the data is well reproduced by the Monte Carlo. The extraction of the \bdtn response of the \lhcb signal was done using the \BdToJPsieeKst decay and the sPlot technique just as explained in Section \ref{sec:splot}. The results are displayed in Figure \ref{fig:comp} and show that the distributions from Monte Carlo and \lhcb signal are in reasonable agreement with each other. \\
\begin{figure}[ht]
\vspace*{-0.2cm}
\begin{center}
\subfigure{\includegraphics[width = 0.49 \textwidth]{BDT0_ROC.pdf}}
\subfigure{\includegraphics[width = 0.49 \textwidth]{BDT1_ROC.pdf}}
\end{center}
\vspace*{-0.8cm}
\caption{\textit{Background rejection efficiency as a function of the signal selection efficiency for the trained \bdta (left) and the \bdtb (right).}}
\label{fig:BvsS}
\end{figure}

\begin{figure}[ht]
\begin{center}
\subfigure{\includegraphics[width = 0.49 \textwidth]{responsea.png}}
\subfigure{\includegraphics[width = 0.49 \textwidth]{responseb.png}}
\end{center}
\vspace*{-0.8cm}
\caption{\textit{\bdtn response on the signal and the background. Red dots: response on the background training sample, red area: response on the background testing sample, blue dots: response on the signal training sample, blue area: response on the signal testing sample. The response for the testing and the training samples coincide for signal and background respectively which shows that \bdts were not overtrained.}}
\label{fig:resp}
\end{figure}

\begin{figure}[ht]
\begin{center}
\includegraphics[width = 0.49 \textwidth]{BDT_comp.pdf}
\end{center}
\vspace*{-0.5cm}
\caption{\textit{Distribution of the \bdtn response on the \BdToJPsieeKst Monte Carlo (red curve) and \BdToJPsieeKst \lhcb data (black dots) in the region around the working point of the \bdtn cut.}}
\label{fig:comp}
\end{figure}

In order to compare the behaviour of \bdta and \bdtb both are applied to all events in the \BdKstee Monte Carlos and the \lhcb data after the preselection. Figure \ref{fig:bdtvsbdt} shows the average value of \bdtb for a given bin of \bdta  value on data as well as Monte Carlo. Due to the limited amount of Monte Carlo events below a \bdtn value < 0.5 the points below this value are not representative. Above 0.8 -- the region in which the working point will lie -- the points show the same systematics for Monte Carlo and data respectively. This behaviour reflects the coherence between \bdta and \bdtb. The deviance from the diagonal is the result of statistical fluctuations in the training samples.

\begin{figure}[ht]
\begin{center}
\subfigure{\includegraphics[width = 0.49 \textwidth]{mcBDT.pdf}}
\subfigure{\includegraphics[width = 0.49 \textwidth]{mcBDT_ZOOM.pdf}}
\end{center}
\caption{\textit{\bdtn response on the \BdKstee Monte Carlo (red dots) and \lhcb data (blue dots). Each point represents the average value of \bdtb for a given bin of \bdta value. In the left plot a zoom on the region around the working point of the \bdtn cuts can be seen.}}
\label{fig:bdtvsbdt}
\end{figure}
\newpage

\section{Optimisation of the selection}
\label{sec:opti}
In order to guarantee the best selection of the \BdKstee candidates a 2-dimensional optimisation of the cut on the BDT and the \dllepi is performed. This optimisation is performed for the three trigger categories independently. As explained in Section \ref{sec:strategy} the \bdta is optimised on the data sample A2 and \bdtb is optimised on B2.\\
The optimal combination of cuts is found by maximising the variable $x = \frac{S}{\sqrt{S+B}}$, where $S$ stands for the number of signal events and $B$ the number of background events in the signal window. Therefore $S$ and $B$ are calculated for every cut combination. The cut on the \dllepi is varied between $>0$ and $>3$ in steps of 1. The cuts on \bdta and \bdtb are varied between $> \, 0.8$ and $> \, 0.98$ in steps of 0.2.\\

\subsection{Estimation of signal candidates $S$}
The number of signal events is calculated using a combination of Monte Carlo and \lhcb data for the \BdKstee and the \BdToJPsieeKst decays.
\begin{eqnarray}
S(\BdKstee) = S(\BdToJPsieeKst) \cdot  & \frac{\epsilon^{sel}(\BdKstee)}{\epsilon^{sel}(\BdToJPsieeKst)} \nonumber \\
 &\frac{\BR(\BdKstee)}{\BR(\BdToJPsieeKst)}
\end{eqnarray}
The ratio of selection efficiencies $\frac{\epsilon^{sel}(\BdKstee)}{\epsilon^{sel}(\BdToJPsieeKst)}$ are obtained from the Monte Carlo while the number of \BdToJPsieeKst signal events after the selection S(\BdToJPsieeKst) is computed from \lhcb data. The strategy of combining Monte Carlo and \lhcb data is chosen because the Monte Carlo does not reproduce certain properties of the data such as the trigger behaviour or the distribution of the particle identification value. The ratio of efficiencies for the \BdKstee and the \BdToJPsieeKst on the other hand should be free of these systematic errors.\\
For the purpose of extracting $S(\BdToJPsieeKst)$ the same selection as for the \BdKstee candidates is applied with the exception $2400 \mevcc \, < \, m(\jpsi) \, < \, 3400 \mevcc $ instead of the $m(\epem) $ cut. \\
As in the reweighting procedure for $nPV $ the constrained on the \jpsi mass is applied. To extract $S(\BdToJPsieeKst)$ a function consisting of a double Crystal-Ball distribution \cite{crystal} representing the signal and a exponential representing the background is fitted to the \Bd mass.
An example can be seen in Figure \ref{fig:ex1} on the left.\\

\subsection{Estimation of background events $B$ in the signal window}
The background is estimated from the \BdKstee \lhcb data.
Therefore the signal region is blinded and an exponential function is fitted to the sidebands which describes the background distribution to a first approximation. For the fit result the amount of background events is extrapolated. An example of the fit to the background can be seen in Figure \ref{fig:ex2}.\\

\begin{figure}[ht]
\vspace*{-0.5cm}
\begin{center}
\subfigure{\includegraphics[width = 0.49 \textwidth]{BDT0_signalesti.pdf}\label{fig:ex1}}
\subfigure{\includegraphics[width = 0.49 \textwidth]{bkgest.jpg}\label{fig:ex2}}
\end{center}
\caption{\textit{\textbf{Left}: Sample B2 \Bd mass distribution from the \BdToJPsieeKst with the cut combination \bdta > 0.88 and \dllepi > 1. Black points: \lhcb data, pink curve: signal \PDF, green curve: \PDF for the combinatorial background. \textbf{Right}: Sample B2 \Bd mass distribution from the \BdKstee with the signal window blinded with the cut combination \bdta > 0.88 and \dllepi > 1. Black points: \lhcb data, green curve: \PDF for the combinatorial background. }}
\label{fig:ex}
\end{figure}


\subsection{The optimised cut combination}
As has been shown in the previous section the values for the \bdta and \bdtb are in agreement for a given event. Therefore and for simplicity reasons the cut on the \bdtn value is chosen to be the same on \bdta and \bdtb within a trigger category. The optimisation is hence performed on one of the \bdtn values only. The Figures \ref{fig:optiele}, \ref{fig:optihad} and \ref{fig:optitis} show the value of $x = \frac{S}{\sqrt{S+B}}$ for the different combination of cuts on the \dllepi and the \bdtn value. The relative error on $x$ is between $5\, \%$ for the L0Electron category and $10\, \%$ for the categories with less statistics L0Hadron and L0TIS.\\
%It can be seen from these plots that the value of $x$ is already on a plateau meaning that the variation of $x$ over the displayed \dllepi and \bdtn value space. Furthermore \\


%\subsubsection{The \dllepi cut}
%The L0Electron category which has the highest statistics clearly prefers the cut of $\dllepi >1$. This statement is also true for the L0TIS category. \\
%The L0Hadron category prefers tighter cuts. However, the distribution of $x$ is not the only and best classifier of the cut combination. Taking into account the values of $S$ and $B$ separately it can be seen, that the tight cut on \dllepi removes a lot of background but also reduces the expected signal events. From these observations the cut on \dllepi for the L0Hadron category was chosen to be $\dllepi >1$. This selection is predicted to have a ration of $\frac{S}{B} \approx 1$ while keeping the signal yield at a significant level.\\
%
%\subsubsection{The \bdtn cut}
%As can be seen in Figures \ref{fig:optiele} and \ref{fig:optitis} the L0Electron and the L0TIS category have both a preferred cut value of $\bdtn \, >\, 0.88$ and $\bdtn \, >\, 0.92$ respectively. Again, Figure \ref{fig:optihad} for the L0Hadron category does not give a clear indication of which \bdtn cut is optimal. However, taking into account the distributions of the signal yield $S$ and the estimated number of background $B$ the chosen cut of $\bdtn \, >\, 0.9$ was chosen.
\begin{figure}[ht]
\begin{center}
\includegraphics[width = 0.8 \textwidth]{BDT_L0Ele_opti.pdf}
\end{center}
\vspace*{-1cm}
\caption{\textit{The variable $x = \frac{S}{\sqrt{S+B}}$ for different combinations of the cut on \dllepi and the \bdtn value evaluated for the events in the L0Electron category. The error on $x$ is about $5\, \%$.}}
\label{fig:optiele}
\end{figure}

\begin{figure}[ht]
\begin{center}
\includegraphics[width = 0.8 \textwidth]{BDT_L0Had_opti.pdf}
\end{center}
\vspace*{-1cm}
\caption{\textit{The variable $x = \frac{S}{\sqrt{S+B}}$ for different combinations of the cut on \dllepi and the \bdtn value evaluated for the events in the L0Hadron category. The error on $x$ is about $10\, \%$. }}
\label{fig:optihad}
\end{figure}
%\begin{figure}[ht]
%\begin{center}
%\includegraphics[width = 0.8 \textwidth]{BDT_L0TIS_opti.pdf}
%\end{center}
%\vspace*{-1cm}
%\caption{\textit{The variable $x = \frac{S}{\sqrt{S+B}}$ for different combinations of the cut on \dllepi and the \bdtn value evaluated for the events in the L0TIS category. The error on $x$ is about $10\, \%$. }}
%\label{fig:optitis}
%\end{figure}
\noindent%
\begin{minipage}{\linewidth}
\makebox[\linewidth]{%
  \includegraphics[width=0.8\textwidth]{BDT_L0TIS_opti.pdf}}
  \vspace*{-1cm}
\captionof{figure}{\textit{The variable $x = \frac{S}{\sqrt{S+B}}$ for different combinations of the cut on \dllepi and the \bdtn value evaluated for the events in the L0TIS category. The error on $x$ is about $10\, \%$. }}
  \label{fig:optitis}
\end{minipage}

From Figure \ref{fig:optiele}, \ref{fig:optihad} and \ref{fig:optitis} the working points can easily be determined since the distributions of $x$ have each a local maximum. This lies at $\dllepi >1$ for all three trigger categories and at $\bdtn \, >\, 0.88$, $\bdtn \, >\, 0.9$ and $\bdtn \, >\, 0.92$ for L0Electron, L0Hadron and L0TIS respectively.\\

\begin{table}[ht]
\begin{center}
\begin{tabular}{c|c|c}
 & \dllepi cut & \bdtn cut\\
 \hline
 \hline
 L0Electron & $\dllepi \, > \, 1$ & $\bdtn >0.88$\\
 \hline
 L0Hadron & $\dllepi \, > \, 1$ & $\bdtn >0.9$\\
 \hline
 L0TIS & $\dllepi \, > \, 1$ & $\bdtn >0.92$\\
\end{tabular}
\end{center}
\caption{\textit{Results of the optimisation of the cuts on \dllepi and the \bdtn value for the three independent trigger categories.}}
\label{tab:bdtcuts}
\vspace{1cm}
\end{table}


\section{Veto cuts against specific background contamination}
While the \bdts show very good performance in rejecting the combinatorial background other more specific sources of background are not gravely affected by the \bdtn cuts. One of this background sources is the $\Bd \rightarrow \Dm \ep \neu$ decay that was discussed in Section \ref{sec:denu}. This background is now suppressed by the cut $\ctl <0.8$.\\
However, there are two other specific backgrounds that contribute to the contamination of the \BdKstee candidates. They will be discussed in the following.\\

\subsection{The $\Phi \rightarrow \Kp \Km$ veto cut}
The decay $\Bs \rightarrow \Phi \epem$ is expected to have a branching ratio of the same order as the \BdKstee. Since the kinematics are similar and since
the $\Phi$ meson has a branching ratio of $50\, \%$ for decaying into two charged charged kaons it is important to reject this specific background. If one of the kaons is misidentified as a pion the $\Phi$ could be reconstructed as a \Kstarz. While this background is already suppressed by the cuts on the particle identification variables for the kaon and the pion there is still a visible contribution from the $\Phi \rightarrow \Kp \Km$. This background can be suppressed further by calculating the mass of the $m(\Kp \Km)$ by assigning the pion-track the mass of a kaon. The distribution of this $m(\Kp \Km)$ variable is shown in Figure \ref{fig:masskk}. \\
The cut against the contamination from the $\Phi \rightarrow \Kp \Km$ is suppressed by a cut of $m(\Kp \Km)>1040\mevcc$\footnote{$m(\Phi) = 1019 \mevcc$\cite{pdg}}.\newpage
\begin{figure}[ht]
\begin{center}
\includegraphics[width = 0.5\textwidth]{masskk.pdf}
\end{center}
\vspace*{-0.8cm}
\caption{\textit{Distribution of the $m(\Kp \Km)$ variable calculated under the assumption that the pion from \BdKstee is a misidentified kaon. The pink curve shows the distribution from the \BdKstee Monte Carlo while the blue curve is the distribution of the \lhcb data. Both samples were processed by the \BdKstee stripping20 line and underwent the preselection cuts. The data curve clearly shows a peak from the $\Phi \rightarrow \Kp \Km$ around 1020 \mevcc.}}
\label{fig:masskk}
\vspace*{1.5cm}
\end{figure}

\subsection{The \BdKstg veto cuts}
The \BdKstg decay has a branching ratio that is two orders of magnitude higher than the branching ratio of the \BdKstee decay and with the photon converting into two electrons the \BdKstg decay will look very similar to the \BdKstee decay. In \lhcb about $40\, \%$ of the photons convert before the calorimeter and while only about $10\, \%$ are reconstructed, the resulting \Bd mass peaks directly under the signal making it a particularly dangerous contamination.\\
Most of this background is removed by the lower cut on the $m(\epem) >20 \mevcc$. Additionally another cut has been developed to reject this background. Unlike for the \BdKstee decay, the $z$ coordinate of the \ep \en pair from the \BdKstg does not have to coincide with the \Kstarz vertex. If this coordinate is measured with a large error the \BdKstg may still be reconstructed as a \BdKstee decay. By applying a tight cut on the error of the $z$ coordinate $\sigma_z(\epem)$ this background can already be suppressed.\\
Even after this cut the \BdKstg decay can make a significant contribution to the \BdKstee signal peak. Therefore another cut is developed which takes into account that electrons from converted photons leave a different signature in the \velo stations (see Chapter \ref{chapter1}) than the electrons from the \BdKstee signal. In particular, the hits in the \velo for the electrons from converted photons commence at a larger $z$ coordinate than expected due to the flight distance of the photon before converting. Therefore a cut has been developed that takes into account the difference between the expected and the measured $z$ coordinate of the electrons first hit in the \velo. The expected $z$ coordinate of the electrons first hit in the \velo is calculated by extrapolating the position of the \Kstarz decay vertex in the direction of the electron momentum. The $z$ coordinate of the first \velo sensor that is crossed by this extrapolation is taken to be the first expected $z$ coordinate. This is illustrated in Figure \ref{fig:deltafm}.\\
A cut on this variable of $\Delta_z\, <\, 20 \mm$ has been found to be optimal. The cut on error of the $z$ coordinate of the (\epem) vertex was chosen to be $\sigma_z(\epem)<30\mm$. 
\begin{figure}[ht]
\begin{center}
\includegraphics[width = 0.6\textwidth]{deltafm.jpg}
\end{center}
\caption{\textit{Illustration of the calculation of the expected $z$ coordinate in the \velo sensors for the first hit of the electron. The expected $z$ coordinate of the electrons first hit in the \velo is calculated by extrapolating the position of the \Kstarz decay vertex in the direction of the electron momentum. The $z$ coordinate of the first \velo sensor that is crossed by this extrapolation is taken to be the first expected $z$ coordinate.}}
\label{fig:deltafm}
\end{figure}

\chapter{The \BdKstee signal yields}
\chaptermark{Chapter6}
\label{Chapter5}
After the set of preselection, optimised and veto cuts have been applied to the \\3 \invfb of data collected by \lhcb in 2011 and 2012 the signal yields are extracted. Therefore a probability density function (\PDF) is fitted to the resulting data sample. The entire fitting procedure is performed for each trigger category independently. The function of the \PDF is the same for all trigger categories but the parameters can vary.\\
The \PDF for the \BdKstee \lhcb data consists of different components that will be explained in the next section. Since the statistics of the final \BdKstee candidates is low the parameters of the \PDF for the \BdKstee data will be constrained using the \BdKstee Monte Carlo and also the \BdToJPsieeKst Monte Carlo and data. The \BdKstee Monte Carlo is used to constrain the shape of the signal while the more numerous \BdToJPsieeKst events are used to account for differences between Monte Carlo and \lhcb data and also to extract information about the background. The procedure of constraining the parameters is explained in Section \ref{sec:fitstrat} followed by the fit of the \PDF to the \BdKstee data and the extraction of signal and background yields.\\
At the end of this chapter the results obtained by the selection developed in this master's thesis are compared to the results that would have been obtained by adopting the selection developed on the 2011 \lhcb data.\\

\section{The components of the Probability Density Function}
\label{sec:pdfs}
The final \PDF for both the \BdKstee and the \BdToJPsieeKst candidates consists of a \PDF for the signal and a \PDF for the background. The background \PDF itself is composed of different components, namely the \textit{combinatorial background} and the \textit{partially reconstructed background}. These \PDF will be explored in the following.\\

\subsection{The signal \PDF}
Due to the energy loss of the electrons through bremsstrahlung radiation a double Crystal-Ball distribution (CB)\cite{crystal} was chosen for the the signal \PDF. \\
A CB distribution consists of a Gaussian core and a power law tail used to model the effect of energy loss during the reconstruction process. The CB function is described by:\\
\begin{equation}
CB(x;\alpha,n,\mu,\sigma) =  \begin{cases} \exp(- \frac{(x - \mu)^2}{2 \sigma^2}), & \mbox{for }\frac{x - \mu}{\sigma} > -\alpha \\ \frac{\left(\frac{n}{|\alpha|}\right)^n e^{-\frac{\alpha^2}{2}} }{\left(\frac{n}{|\alpha|} - |\alpha| -x\right)^n}, & \mbox{for }\frac{x - \mu}{\sigma} \leqslant -\alpha \end{cases} 
\end{equation}
where $\mu$ is the mean of the distribution, $\sigma$ the width, $\alpha$ the transition point from a Gaussian distribution to a power law tail distribution and $n$ the exponent of the power law tail.\\
The signal \PDF is modelled by the sum of two CBs to account for different resolution effects of the \lhcb detector. Therefore the CB distributions share the same mean $\mu$, $\alpha$ and $n$ but do have different widths $\sigma_1$ and $\sigma_2$. Furthermore there is a fraction $f$ between these two CBs.
\begin{equation}
\PDF^{sig} \ = \ f\ CB(x;\alpha,n,\mu,\sigma_1) + (1-f)\ CB(x;\alpha,n,\mu,\sigma_2)
\label{eq:dcb}
\end{equation}
\\
\subsection{The background \PDF}
The background in the \BdKstee decay consists of a \textit{combinatorial background} and a \textit{partially reconstructed background}. The combinatorial background originates from the matching of random tracks in the event that do not come from the same \B meson. The partially reconstructed background on the other hand comes from decays of \B mesons that are not fully reconstructed, that is where one or more tracks are not recuperated.\\

\subsubsection{The \PDF of combinatorial background}
The combinatorial background is fitted with an exponential function for each data sample individually and no constrains are applied.\\
\begin{equation}
\PDF^{comb. bkg} \ = e^{bx}
\end{equation}
\\
\subsubsection{The \PDF of partially reconstructed background}
The partially reconstructed background for the \BdKstee data comes from decays where the electrons originate from the \B meson while the \Kstarz meson is the decay product of an intermediate resonance. This means that the \B meson decay is of the form $\B \rightarrow \epem Y(\rightarrow \Kstarz X)$ where the $X$ is not reconstructed. This background also occurs for the \BdToJPsieeKst.\\
Another kind of partially reconstructed background that only exists for the\\ \BdToJPsieeKst but not for the \BdKstee is of the form\\ $B \rightarrow A (\rightarrow \jpsi B) \Kstarz$ where the $B$ is not reconstructed. This background does not occur for the \BdKstee since the electrons from \BdKstee can not come from any intermediate resonance due to the low dielectron invariant mass\footnote{The only resonance lying in the \epem invariant mass window is the $\rho$ giving $B \rightarrow \rho(\rightarrow \epem) \Kstarz$ which would be a peaking background instead of a partially reconstructed background. However, the contribution of the $\rho$ resonance has been found to be highly negligible.}. Since a full fit on the \BdToJPsieeKst data will be performed this background has to be modelled. \\
\\
The shapes of these two types of partially reconstructed background are determined separately using 1.3 million $\Bd \rightarrow \jpsi \Kstarz X$ and $\Bu / \Bub \rightarrow \jpsi \Kstarz X$ inclusive Monte Carlo events. The \PDF of each type of partially reconstructed background is then represented by a non-parametric function called ``\roopdf ''\cite{rookeys} implemented in \root \footnote{The \roopdf is a class of the \roofit package. It is a one-dimensional estimation which models the distribution of and arbitrary dataset by superposing Gaussian distributions}. The distribution of the $\B \rightarrow \epem Y(\rightarrow \Kstarz X)$ and\\ $B \rightarrow A (\rightarrow \jpsi B) \Kstarz$ background can be seen in Figures \ref{fig:parthad} and \ref{fig:partpsi} respectively. Since the shapes of the partially reconstructed background are the same for each trigger category the events are not divided by categories but the shape is determined on the events of all three trigger categories in order to be less sensitive to statistical fluctuations.
\begin{figure}[ht]
\begin{center}
\includegraphics[width = \textwidth]{PartHad.pdf}
\end{center}
\vspace*{-0.8cm}
\caption{\textit{Shapes of the partially reconstructed background from decays of the form $\B \rightarrow \epem Y(\rightarrow \Kstarz X)$ obtained from inclusive Monte Carlo samples for the three trigger categories. The dots represent the data points while the curves are the \PDF described by the \roopdf objects.}}
\label{fig:parthad}
\end{figure}


\begin{figure}[ht]
\vspace*{-0.5cm}
\begin{center}
\includegraphics[width = \textwidth]{PartPsi.pdf}
\end{center}
\vspace*{-0.8cm}
\caption{\textit{Shapes of the partially reconstructed background from decays of the form $B \rightarrow (Y \rightarrow \jpsi X) \Kstarz$ obtained from inclusive Monte Carlo samples for the three trigger categories. The dots represent the data points while the curves are the \PDF described by the \roopdf objects.}}
\label{fig:partpsi}
\vspace*{0.5cm}
\end{figure}

\section{Fit strategy and results}
\label{sec:fitstrat}
In this section the fit strategy and means of constraining the parameters of the \PDF for the \BdKstee \lhcb data are presented. To obtain all parameters needed to perform the fit on the \BdKstee \lhcb data information has to be extracted and combined from the \BdKstee Monte Carlo, the \BdToJPsieeKst Monte Carlo and the \BdToJPsieeKst \lhcb data.\\
The \BdToJPsieeKst decay is used in this comparison because it has a much larger yield and the same final state particles. Thus the reconstruction follows using the same detector components and reconstruction algorithms which results in the same effects on the reconstructed \Bd mass shape. Furthermore the same selection as for the \BdKstee can be applied to the \BdToJPsieeKst with exception of the cut on the invariant mass of the electron-positron pair. However, the kinematics are not exactly the same for both decays and therefore it is not possible to rely on information from the \BdToJPsieeKst data only. \\
While the shape of the \BdKstee signal \PDF depends on the kinematics of the decay, the effects from detector resolution and calibration predominately depend on the final state particles. Therefore the \BdKstee Monte Carlo is used to extract information about the shape of the signal, that is the parameters of the double CB distribution while the comparison between the signal \PDF from the \BdToJPsieeKst Monte Carlo with the \BdToJPsieeKst data gives the difference in resolution (effect on the width of the signal \PDF) between Monte Carlo and detector and also the difference in calibration (effect on the mean of the double CB distribution). \\
Additionally to the information about the signal \PDF, the portion of partially reconstructed background events with respect to signal events is extracted from the \BdToJPsieeKst data. This approach relies on the assumption that the amount of $\B \rightarrow \jpsi Y(\rightarrow \Kstarz X)$ events with respect to \BdToJPsieeKst events is the same as the amount of $\B \rightarrow \epem Y(\rightarrow \Kstarz X)$ with respect to \BdKstee events. This assumption can be made because partially reconstructed background concerns only the $Y(\rightarrow \Kstarz X)$ part of the decay. Therefore it is independent of the difference between \BdKstee and \BdToJPsieeKst which is the (\epem) and \jpsi part.\\
This strategy is not applicable to the combinatorial background since the number of combinatorial background events as well as the slope of the exponential describing its \PDF depend on the kinematics of the particles in the decay. This results from the almost random matching of traces in the detector. For example, in order to seemingly form a \jpsi or the \epem pair the combinatorial background electrons have to be selected from different parts of their momentum spectrum. Since this momentum spectrum is not flat and differs from the momentum spectrum of the signal electrons, the number of combinatorial background as well as the slope of the exponential depend on the requirements namely the electron-positron invariant mass.\\
All fits are performed with an extended unbinned maximum likelihood fit.\\

\subsection{The fit to the \BdToJPsieeKst Monte Carlo}
First a double CB distribution like in Equation \ref{eq:dcb} is fitted to the \BdToJPsieeKst Monte Carlo. All fitted parameters are extracted and stored. Figure \ref{fig:jpsimc} shows the \BdToJPsieeKst Monte Carlo and the fitted distribution for the three independent trigger categories.
\begin{figure}[ht]
\vspace*{-0.5cm}
\begin{center}
\subfigure{\includegraphics[width = 0.45\textwidth]{MC_L0Ele_JpsiKstar.pdf}}
\subfigure{\includegraphics[width = 0.45\textwidth]{MC_L0Had_JpsiKstar.pdf}}\\
\vspace*{-0.5cm}
\subfigure{\includegraphics[width = 0.45\textwidth]{MC_L0TIS_JpsiKstar.pdf}}
\end{center}
\vspace*{-1.cm}
\caption{\textit{\BdToJPsieeKst Monte Carlo after the entire selection procedure. The violet curve is the fitted double Crystal-Ball distribution.}}
\label{fig:jpsimc}
\end{figure}

\subsection{Fit to the \BdToJPsieeKst \lhcb data}
A fit to the \BdToJPsieeKst \lhcb data is performed using the \PDF for the signal and the three different background components. In this fit the signal \PDF is constrained by:
\begin{itemize}
\item The parameter $\alpha$ between the two CB distribution is taken from the fit to the \BdToJPsieeKst Monte Carlo.
\item The parameter $n$ is taken from the fit to the \BdToJPsieeKst Monte Carlo.
\item The fraction $f$ is taken from the fit to the \BdToJPsieeKst Monte Carlo.
\item The widths $\sigma_1$ and $\sigma_2$ are taken from the fit to the \BdToJPsieeKst Monte Carlo but are allowed to scale by the same factor $s_{\sigma}$ to account for differences in resolution between the Monte Carlo and \lhcb data (for example due to imperfect detector alignment and/or effects from detector ageing that are not modelled in the Monte Carlo).
\end{itemize}
The mean of the double CB distribution is free.\\
While the slope of the combinatorial background \PDF is free to be fitted as well as the number of combinatorial background events, the number of partially reconstructed background events are fitted with respect to the number of \\ \BdToJPsieeKst signal events for the two types of partially reconstructed background respectively. I.e. the numbers of partially reconstructed background are \\
$N^{part. bkg \Kstarz}_{\jpsi \Kstarz} = r^{part. bkg \Kstarz}_{\jpsi \Kstarz} \cdot N^{sig}_{\jpsi \Kstarz}$ and $N^{part. bkg \jpsi}_{\jpsi \Kstarz} = r^{part. bkg \jpsi}_{\jpsi \Kstarz} \cdot N^{sig}_{\jpsi \Kstarz}$ and the ratios $r^{part. bkg \Kstarz}_{\jpsi \Kstarz}$ and $r^{part. bkg \jpsi}_{\jpsi \Kstarz} $ are fitted.\\

\subsubsection{Contribution from the $\Bs \rightarrow \jpsi (\epem)  \Kstarz$ decay}
The  $\Bs \rightarrow \jpsi (\epem)  \Kstarz$ decay can also occur but at a strongly reduced rate due to the CKM elements involved and due to the smaller production rate of \Bs meson compared to \Bd mesons.\\
All parameters of the double CB distribution for the $\Bs \rightarrow \jpsi (\epem)  \Kstarz$ are taken to be the same as for the \BdToJPsieeKst except for the mean and the number of $\Bs \rightarrow \jpsi (\epem)  \Kstarz$ events. The mean of the $\Bs \rightarrow \jpsi (\epem)  \Kstarz$ distribution is fixed to the mean of \BdToJPsieeKst shifted by $88 \mevcc$\footnote{$m^{PDG}(\Bs) - m^{PDG}(\Bd) = 88\mevcc$}.\\
The ratio of numbers of $\Bs \rightarrow \jpsi (\epem) \Kstarz$ events with respect to \BdToJPsieeKst events was constrained by a Gaussian with mean
 $\mu^{constr.}_{\Bs} = \frac{N_{\Bs}}{N_{\Bd}} = 8.5 \cdot 10^{-3}$ and width $\sigma^{constr.}_{\Bs} = 1.2 \cdot 10^{-3}$.
These values are taken from the analysis of the\\ $\Bs \rightarrow \jpsi (\epem) \Kstarz$ \cite{BsJpsi} where the ratio of events from $\Bs \rightarrow \jpsi (\epem) \Kstarz$ to \BdToJPsieeKst was studied.\\
Figures \ref{fig:jpsidata} shows the \BdToJPsieeKst \lhcb data with the different components of the \PDF.

\begin{figure}[!h]
\vspace*{-0.5cm}
\begin{center}
\subfigure{\includegraphics[width = 0.49\textwidth]{Data_L0Ele_JpsiKstar.pdf}}
\subfigure{\includegraphics[width = 0.49\textwidth]{LogData_L0Ele_JpsiKstar.pdf}}\\
%\vspace*{-0.5cm}
\subfigure{\includegraphics[width = 0.49\textwidth]{Data_L0Had_JpsiKstar.pdf}}
\subfigure{\includegraphics[width = 0.49\textwidth]{LogData_L0Had_JpsiKstar.pdf}}\\
%\vspace*{-0.5cm}
\subfigure{\includegraphics[width = 0.49\textwidth]{Data_L0TIS_JpsiKstar.pdf}}
\subfigure{\includegraphics[width = 0.49\textwidth]{LogData_L0TIS_JpsiKstar.pdf}}
\end{center}
%\vspace*{-1cm}
\caption{\textit{\BdToJPsieeKst \lhcb data sample after the entire selection procedure with linear axis on the left and logarithmic axis representation on the right for each trigger category. \textbf{Solid black curve}: the final \PDF , \textbf{dashed pink curve}: the signal \PDF, \textbf{dashed green curve}: the \PDF of the combinatorial background, \textbf{dashed dark blue curve}: the \PDF of the partially reconstructed background of the form $\B \rightarrow \epem Y(\rightarrow \Kstarz X)$, \textbf{dashed clear blue curve}: the \PDF of the partially reconstructed background of the form $B \rightarrow (Y \rightarrow \jpsi X) \Kstarz$, \textbf{dashed gray curve}: the \PDF for the $\Bs \rightarrow \jpsi (\epem) \Kstarz$.}}
\label{fig:jpsidata}
\vspace*{2cm}
\end{figure}


%\begin{figure}[ht]
%\begin{center}
%\subfigure{\includegraphics[width = 0.47\textwidth]{LogData_L0Ele_JpsiKstar.pdf}}
%
%\vspace*{-0.5cm}
%
%\end{center}
%\vspace*{-1cm}
%\caption{\textit{\BdToJPsieeKst \lhcb data sample after the entire selection procedure. \textbf{Solid black curve}: the final \PDF , \textbf{dashed pink curve}: the signal \PDF, \textbf{dashed green curve}: the \PDF of the combinatorial background, \textbf{dashed dark blue curve}: the \PDF of the partially reconstructed background of the form $\B \rightarrow \epem Y(\rightarrow \Kstarz X)$, \textbf{dashed clear blue curve}: the \PDF of the partially reconstructed background of the form $B \rightarrow (Y \rightarrow \jpsi X) \Kstarz$, \textbf{dashed gray curve}: the \PDF for the $\Bs \rightarrow \jpsi (\epem) \Kstarz$.}}
%\label{fig:jpsidatalog}
%\end{figure}


\subsection{Fit to the \BdKstee \lhcb data}
In the fit to the \BdKstee \lhcb data only three parameters are left entirely free: the number of signal events $N^{sig}$, the number of combinatorial background events $N^{comb. bkg}$ and the slope $b$ of the exponential function representing the combinatorial background. All other parameters are constrained:
\begin{itemize}
\item Parameters $\boldsymbol{\alpha}$, $\mathbf{n}$, $\mathbf{f}$: parameters of the signal \PDF taken from the double CB distribution fitted to the \BdKstee Monte Carlo.
\item $\mathbf{s_{\boldsymbol{\sigma}}\cdot \boldsymbol{\sigma_1}}$ and $\mathbf{s_{\boldsymbol{\sigma}}\cdot \boldsymbol{\sigma}_2}$: widths of the signal \PDF, $\sigma_1$ and $\sigma_2$ are taken from the double CB distribution fitted to the \BdKstee Monte Carlo, the scale factor $s_{\sigma}$ is taken from the fit to the \BdToJPsieeKst \lhcb data.
\item $\mathbf{\boldsymbol{\mu}^{data}_{\epem \Kstarz}}$: mean of the signal \PDF, is taken to be the mean of the double CB distribution fitted to the \BdToJPsieeKst \lhcb data summed by the difference $\delta_{MC}$ of the mean of the double CB distributions fitted to\\ \BdKstee and \BdToJPsieeKst Monte Carlo \\ $\delta_{MC} = \mu^{MC}_{\epem \Kstarz} - \mu^{MC}_{\jpsi \Kstarz}$.
\item $\mathbf{N^{part. bkg}}$: the number of partially reconstructed background events is indirectly constrained through the Gaussian constraint put on the ratio $r^{part. bkg}\cdot \frac{N^{part. bkg}}{N^{sig}}$. The mean of this Gaussian constraint is taken from the\\ \BdToJPsieeKst \lhcb data fit to be $r^{part. bkg \Kstarz}_{\jpsi \Kstarz}$ and the width is the error on this parameter.\\
\end{itemize}

\subsubsection{Contamination from \BdKstGam events}
In the 2011 analysis the contribution from the \BdKstGam decays had to be taken into account during the fit. However, using the new reconstruction tools presented in Chapter \ref{chapter3} the \BdKstGam contamination is reduced to the level of 1\% and below depending on the trigger category. Therefore contribution from \BdKstGam decays is neglected. The percentage of \BdKstGam events with respect to \BdKstee events was calculated on the Monte Carlo samples and the results are listed in Table \ref{tab:kstgam}.
\renewcommand{\arraystretch}{1.5} 
\begin{table}[ht]
\begin{center}
\begin{tabular}{l |c}
trigger category & \BdKstGam pollution \\
\hline \hline
L0Electron & 1.1\% \\
\hline
L0Hadron & 0.8\% \\
\hline
L0TIS & 0.0\% \\
\end{tabular}
\end{center}
\caption{\textit{Percentage of \BdKstGam events with respect to \BdKstee events after the entire selection determined on Monte Carlo samples.}}
\label{tab:kstgam}
\end{table}
\newpage
Figures \ref{fig:eedata} show the \BdKstee \lhcb data with the different components of the \PDF.
\begin{figure}[ht]
\begin{center}
\subfigure{\includegraphics[width = 0.49\textwidth]{40_Data_L0Ele_eeKstar.pdf}}
\subfigure{\includegraphics[width = 0.49\textwidth]{40_Data_L0Had_eeKstar.pdf}}\\
\vspace*{-0.5cm}
\subfigure{\includegraphics[width = 0.49\textwidth]{40_Data_L0TIS_eeKstar.pdf}}
\end{center}
\vspace*{-0.5cm}
\caption{\textit{\BdKstee \lhcb data after the entire selection procedure. \textbf{Green area}: combinatorial background events, \textbf{blue area}: partially reconstructed background events, \textbf{pink area}: \BdKstee signal events.}}
\label{fig:eedata}
\end{figure}
\\


%\section{Determination of the parameters for the \BdKstee \lhcb data \PDF}
%\label{sec:paramextra}
%In this section it is described how the parameters for the \BdKstee \lhcb data \PDF are extracted from the \BdKstee Monte Carlo, the \BdToJPsieeKst Monte Carlo and the \BdToJPsieeKst \lhcb data.
%
%The signal-function that is fitted to the \BdKstee \lhcb data is:
%\begin{eqnarray}
%\PDF^{sig} \ = & &N^{sig} \left[ \ f\ CB(x;\alpha,n,\mu^{data}_{\jpsi \Kstarz}+ \delta^{MC},s_{\sigma} \cdot \sigma_1) \\
%& +& (1-f)\ CB(x;\alpha,n,\mu^{data}_{\jpsi \Kstarz} + \delta^{MC},s_{\sigma} \cdot \sigma_2) \right]
%\end{eqnarray}
%where all parameters except for the number of signal events $N^{sig}$  are determined in advance:
%\begin{enumerate}
%\item $\boldsymbol{\alpha}$, $\mathbf{n}$, $\mathbf{f}$, $\mathbf{\boldsymbol{\sigma}_1}$ and $\mathbf{\boldsymbol{\sigma}_2}$: parameters of the double CB distribution fitted to the \BdKstee Monte Carlo
%\item $\mathbf{\boldsymbol{\mu}^{data}_{\jpsi \Kstarz}}$: mean of the double CB distribution fitted to the \BdToJPsieeKst \lhcb data
%\item $\mathbf{\boldsymbol{\delta}^{MC}}$: difference between the means of the double CB distributions fitted to the \BdKstee Monte Carlo and the \BdToJPsieeKst Monte Carlo respectively
%\item $\mathbf{s_{\boldsymbol{\sigma}}}$: scale between the widths of the CB distributions fitted to the \BdToJPsieeKst Monte Carlo and \BdToJPsieeKst \lhcb data
%\end{enumerate}
%\\
%The function for the combinatorial background that is fitted to the \BdKstee \lhcb data is:
%\begin{equation}
%\PDF^{comb. bkg} \ = N^{comb. bkg} \cdot e^{bx}
%\end{equation}
%both the number of combinatorial background events $N^{comb. bkg}$ and the slope of the exponential function $b$ are free in the fit.\\
%\\
%The function for the partially reconstructed background of the form $\B \rightarrow \epem Y(\rightarrow \Kstarz X)$ that is fitted to the \BdKstee \lhcb data is:
%\begin{equation}
%\PDF^{comb. bkg} \ = N^{sig}\cdot r^{part. bkg} \cdot \roopdf
%\end{equation}
%where $r^{part. bkg}$ is the ratio of partially reconstructed background events with respect to the number of signal events. This ratio constrained in the fit by the means of a Gaussian constrained. The parameters of this constraining Gaussian are determined from the \BdToJPsieeKst \lhcb data fit and the mean of Gaussian $\mu^{constr.} = \frac{N^{part. bkg}_{\jpsi \Kstarz}}{N^{sig}_{\jpsi \Kstarz}}$ and the width $\sigma^{constr.}$ is the error on the value of $\frac{N^{part. bkg}_{\jpsi \Kstarz}}{N^{sig}_{\jpsi \Kstarz}}$.
%
%
%\section{Fit results}
%\label{sec:fitresults}
%\subsection{\BdToJPsieeKst Monte Carlo}
%Figure \ref{fig:jpsimc} shows the \BdToJPsieeKst Monte Carlo distribution with the double CB distribution fitted to it.
%
%\subsection{\BdKstee Monte Carlo}
%Figure \ref{fig:eemc} shows the \BdKstee Monte Carlo distribution with the double CB distribution fitted to it.
%\begin{figure}[ht]
%\begin{center}
%\subfigure{\includegraphics[width = 0.49\textwidth]{MC_L0Ele_eeKstar.pdf}}
%\subfigure{\includegraphics[width = 0.49\textwidth]{MC_L0Had_eeKstar.pdf}}
%\subfigure{\includegraphics[width = 0.49\textwidth]{MC_L0TIS_eeKstar.pdf}}
%\end{center}
%\caption{\textit{\BdKstee Monte Carlo after the entire selection procedure. The blue curve is the fitted double Crystal-Ball distribution.}}
%\label{fig:eemc}
%\end{figure}
%
%
%\subsection{\BdToJPsieeKst \lhcb data}
%Figure \ref{fig:jpsidata} shows the \BdToJPsieeKst \lhcb data. The \PDF fitted to the data consists of the signal, the combinatorial background, the partially reconstructed background of the form $\B \rightarrow \epem Y(\rightarrow \Kstarz X)$ and the partially reconstructed background of the form $B \rightarrow (Y \rightarrow \jpsi X) \Kstarz$.\\
%
%
%
%\subsection{\BdKstee \lhcb data}
%The \BdKstee \lhcb data can be seen in Figure \ref{fig:eedata}.

%
\subsection{Results of the fit to \BdKstee \lhcb data}
Table \ref{tab:fitresults} lists the fitted parameters for the entire \PDF to the \BdKstee \lhcb data. All parameters are listed for the three mutually exclusive trigger categories. Furthermore the means by which each parameter is determined is summarised in this Table.\\
The result of the fit yields 130 $\pm$ 17 \BdKstee events selected in 3\invfb of data collected by \lhcb in the 2011 and 2012 summed over the three trigger categories. The resulting statistical signal significance of the signal obtained in this analysis corresponds to 8.3 standard deviations. \\
The number of \BdKstee signal events in this dataset is also in agreement with the number of predicted \BdKstee events calculated in the optimisation process in Section \ref{sec:opti} (see Appendix \ref{ap:SandB} for details on the predicted values). \newpage

\renewcommand{\arraystretch}{1}
\begin{table}[!h]
\begin{center}
\begin{tabular}{c|l|c|l}
parameter & L0Category & value & determined by\\
\hline
\hline
& L0Electron & 5250.1 $\pm$ 0.9 \mevcc & fit to the\\
$\mu^{data}_{\jpsi \Kstarz}$ & L0Hadron & 5231.5 $\pm$ 4.4 \mevcc & \BdToJPsieeKst \lhcb data\\
& L0TIS & 5230.7 $\pm$ 3.0 \mevcc & \\
\hline
 & L0Electron & 8.2 $\pm$ 2.7 \mevcc & difference between the means\\
$\delta_{\mu}$ & L0Hadron & -8.0 $\pm$ 8.0 \mevcc & of to \BdKstee and \\
& L0TIS &  -22.0 $\pm$ 7.6 \mevcc & \BdToJPsieeKst Monte Carlo \\
\hline
 & L0Electron & 37.0 $\pm$ 1.4 \mevcc & fit to the \\
$\sigma_1$ & L0Hadron & 58.8 $\pm$ 4.2 \mevcc & \BdKstee Monte Carlo\\
& L0TIS & 56.1 $\pm$ 3.9 \mevcc & \\
\hline
 & L0Electron & 266.7 $\pm$ 26.0 \mevcc & fit to the \\
$\sigma_2$ & L0Hadron & 280.1 $\pm$ 37.9 \mevcc & \BdKstee Monte Carlo\\
& L0TIS & 237.7 $\pm$ 34.4 \mevcc & \\
\hline
 & L0Electron & 0.87 $\pm$ 0.01 & scale factor between the widths  \\
$s_{\sigma}$ & L0Hadron & 1.05 $\pm$ 0.05 & of the \BdToJPsieeKst\\
& L0TIS &  1.02 $\pm$ 0.04 &  Monte Carlo and \lhcb data\\
\hline
  & L0Electron & 0.55 $\pm$ 0.04 & fit to the \\
$\alpha$ & L0Hadron & 0.56 $\pm$ 0.08 & \BdKstee Monte Carlo\\
& L0TIS & 0.81 $\pm$ 0.25 & \\
\hline
& L0Electron & 1.61 $\pm$ 0.11 & fit to the \\
$n$ & L0Hadron & 1.86 $\pm$ 0.35 & \BdKstee Monte Carlo\\
& L0TIS & 1.69 $\pm$ 0.43 & \\
\hline
& L0Electron & 0.96 $\pm$ 0.01 & fit to the \\
$f$ & L0Hadron & 0.92 $\pm$ 0.02 & \BdKstee Monte Carlo\\
& L0TIS & 0.91 $\pm$ 0.02 & \\
\hline
& L0Electron & 0.48 $\pm$ 0.01 & fit to the \\
$r^{part. bkg}$ & L0Hadron &  0.27 $\pm$ 0.06 & \BdToJPsieeKst \lhcb data \\
& L0TIS & 0.51 $\pm$ 0.06 & \\
\hline
& L0Electron & -0.0037 $\pm$ 0.0003 & fit to the \\
$b$ & L0Hadron &  -0.0040 $\pm$ 0.0006 & \BdKstee \lhcb data\\
& L0TIS & -0.0057 $\pm$ 0.0015 &  \\
\hline
& L0Electron & 373.5 $\pm$ 24.0 &  fit to the \\
$N^{comb. bkg}$ & L0Hadron & 104.1 $\pm$ 13.3 &  \BdKstee \lhcb data\\
& L0TIS & 85.9 $\pm$ 14.0 & \\
\hline
\hline
& \textbf{L0Electron} & \textbf{60.1 $\pm$ 11.6} &  \textbf{fit to the} \\
$\mathbf{N^{sig}}$ & \textbf{L0Hadron} & \textbf{35.0 $\pm$ 8.6} &  \textbf{\BdKstee \lhcb data}\\
& \textbf{L0TIS} & \textbf{34.5 $\pm$ 8.4} & \\
\end{tabular}
\end{center}
\caption{\textit{Parameters of the \PDF fitted to the selected \BdKstee candidates in the 2011 and 2012 \lhcb dataset after the newly developed and optimised selection.}}
\label{tab:fitresults}
\end{table}
\newpage

\section{Comparison to the results of the 2011 selection}
To quantify the quality of the selection developed in the course of this master's thesis the \BdKstee signal yield is compared to the signal yield that would have been obtained by the 2011 selection. Therefore the entire 2011 selection (including the \textit{Stripping17b} line) is applied to the \lhcb data collected in 2011 and 2012. The selected events are divided into the same three independent trigger categories as in Section \ref{sec:triggercat} and the fit procedure from Section \ref{sec:fitstrat} is applied. The results can be seen in Figure \ref{fig:oldeedata} and Table \ref{tab:comp}. The 2011 selection results in 93 $\pm$ 15 \BdKstee events selected in 3\invfb of data collected by \lhcb in the 2011 and 2012 which is about $70 \%$ of the events selected by the selection developed in this master's thesis.\\ 
The amount of combinatorial background events at very low \Bd masses in the new selection is increased with respect to the old selection. This background is removed in the old selection as a by-product of the $\Bd \rightarrow \Dm \ep \neu$ veto cut that has to be removed because of its biasing effect on the angular acceptance (see Section \ref{sec:denu}).\\

\begin{figure}[ht]
\begin{center}
\subfigure{\includegraphics[width = 0.49\textwidth]{oldData_L0Ele_eeKstar.pdf}}
\subfigure{\includegraphics[width = 0.49\textwidth]{oldData_L0Had_eeKstar.pdf}}
\subfigure{\includegraphics[width = 0.49\textwidth]{oldData_L0TIS_eeKstar.pdf}}
\end{center}
\caption{\textit{\BdKstee \lhcb data after the selection procedure developed in the 2011 analysis. \textbf{Green area}: combinatorial background events, \textbf{blue area}: partially reconstructed background events, \textbf{pink area}: \BdKstee signal events.}}
\label{fig:oldeedata}
\end{figure}

\begin{table}[ht]
\begin{center}
\begin{tabular}{c|l|c}
parameter & L0Category & value \\
\hline
\hline
& L0Electron & 35.9 $\pm$ 9.1 \\
$N^{sig}$ & L0Hadron & 21.5 $\pm$ 7.0\\
 & L0TIS &   35.1 $\pm$  9.1\\
\end{tabular}
\end{center}
\caption{\textit{Number of \BdKstee events in the 2011 and 2012 \lhcb dataset after the selection used in the 2011 analysis.}}
\label{tab:comp}
\vspace*{19cm}
\end{table}







%This \PDF is used to fit the \BdKstee and \BdToJPsieeKst Monte Carlo respectively.
%
%\subsubsection{Fit to the \BdToJpsieeKst Monte Carlo}
%The \BdToJpsieeKst Monte Carlo and the fitted signal \PDF from Equation \ref{eq:dcb} can be seen in Figure \ref{fig:jpsimc} for the three independent trigger categories. After the fit all parameters are extracted and stored.

%
%\subsubsection{Fit to the \BdToJPsieeKst \lhcb data}
%
%
%%This \PDF is used for the modelling of any signal involved in this analysis, that is for \lhcb data and Monte Carlo and for the \BdKstee decay as well as the \BdToJPsieeKst decay.\\
%%\\
%After taking into account the between the Monte Carlo and \lhcb data (for example due to imperfect detector alignment and/or effects from detector ageing that are not modelled in the Monte Carlo) the \PDF for the \BdKstee signal is expressed as:
%\begin{equation}
%\PDF^{sig} \ = \ f\ CB(x;\alpha,n,\mu + \delta_{\mu},s_{\sigma} \cdot \sigma_1) + (1-f)\ CB(x;\alpha,n,\mu + \delta_{\mu},s_{\sigma} \cdot \sigma_2)
%\end{equation}
%
%To obtain a most accurate measurement on the number of \BdKstee events all parameters of the signal \PDF are determined and fixed before the final fit.
%Therefore the double CB function in Equation \ref{eq:dcb} is fitted to the \BdKstee Monte Carlo and all parameters are extracted.
%To account for differences in resolution between the Monte Carlo and \lhcb data (for example due to imperfect detector alignment and/or effects from detector ageing that are not modelled in the Monte Carlo) a scale factor $s_{\sigma}$ is multiplied to both widths from Monte Carlo $\sigma_1$ and $\sigma_2$. Furthermore the mean $\mu$ of the double CB distributions -- which corresponds to the mass of the \Bd meson -- is given a addend $\delta_{\mu}$.\\
%% The parameter $s_{\sigma}$ is determined by using the fit result of the \BdToJPsieeKst Monte Carlo fit in the \BdToJPsieeKst data fit and letting the widths scale by $s_{\sigma}$. The parameter $\delta_{\mu}$ is the difference between the mean fitted by \BdToJPsieeKst Monte Carlo and the \BdKstee Monte Carlo. 
%The signal \PDF fitted to the \BdKstee \data is then:
%
%when $\mu$, $\alpha$, $n$, $\sigma_1$ and $\sigma_2$ are the parameters of the \PDF from \BdKstee Monte Carlo.
%\begin{table}
%\begin{center}
%\begin{tabular}{l|c}
%parameter & determined by\\
%\hline
%\hline
%$\alpha$, $n$, $\mu$, $\sigma_1$, $\sigma_2$, $f$ & \BdKstee Monte Carlo\\
%\hline
%$\delta_{\mu}$, $s_{\sigma}$ & comparison of results from \\
%& \BdToJPsieeKst Monte Carlo and data\\
%\end{tabular}
%\end{center}
%\caption{\textit{Parameters of the \BdKstee signal \PDF and the means of constraining them.}}
%\label{tab:mcfit}
%\end{table}

\chapter{Conclusion}
\chaptermark{Chapter6}
\label{chapter6}
The \bsg transition proceeds through a flavour changing neutral current and thus is particularly sensitive to the effects of new physics. These effects could be detectable in details of the decay such as the polarisation of the photon which is predicted by the SM to be predominantly left-handed. Certain extensions of the SM such as the Left-Right Symmetric Model (LRSM) can provide a significant right-handed photon amplitude. The photon amplitudes can be measured by performing an angular analysis of the \BdKstee decays at low invariant dilepton mass.\\
\\
The work in this master's thesis is the first step towards this angular analysis.\\
The comparison between the different reconstruction algorithms for the bremsstrahlung emitted by the electrons shows that the new reconstruction algorithm increases the resolution of the \Bd mass by 13\%. By implementing the tool specially dedicated to reconstructing low dilepton mass pairs the \Bd mass resolution can be augmented by a total of 27\%. Furthermore this tools shows an increased reconstruction efficiency for events with very low invariant dilepton mass which are the ones most important for the angular analysis. Thanks to the new reconstruction algorithms the lower cut on the invariant dilepton mass could be dropped from 30\mevcc in the 2011 analysis to 20\mevcc while keeping a high precision on the angles $\Phi$, $\theta_l$ and $\theta_K$.\\
The new stripping line used in the preselection of the \BdKstee candidates in the 2011 and 2012 \lhcb data features a selection efficiency which is increased by a factor 1.4 with respect to the previous stripping line. Furthermore the selection procedure developed in this work implementing two \bdts yields the largest sample of \BdKstee events ever selected. Summed over the three independent trigger categories for which the analysis was performed the signal yield is 130 $\pm$ 17 events which corresponds to a statistical signal significance of 8.3 standard deviations. This signal yield is 40\% more than the yield that would have been obtained by applying the selection developed for the 2011 analysis.\\
\\
Continuing from this dataset of \BdKstee candidates the angular analysis will be performed. Monte Carlo toy studies have shown that an accuracy on the parameter $A^{(2)}_T$ using 130 \BdKstee events and assuming no contribution from background events of $\sigma(A^{(2)}_T)\approx 0.2$ can be achieved.\\
%\chapter{\appendixname}

\section*{\BdKstee \textit{Stripping17b}}
\label{ap:strip17}
\begin{figure}[!h]
\begin{center}
\includegraphics[width = 0.7\textwidth]{stripping17.jpg}
\end{center}
\label{stripping17}
\caption{The cuts of the 2011 \BdKstee stripping-line.}
\end{figure}
\newpage

\section*{All BDT Input Variables}
\label{ap:allvars}
\begin{figure}[!h]
\begin{center}
\subfigure{\includegraphics[width = \textwidth]{varb1.png}}
\subfigure{\includegraphics[width = \textwidth]{varb2.png}}
\end{center}
\label{fig:varBDTa}
\caption{\textit{Distributions for the different input variables for the \bdta training. Red distribution: background sample, blue distribution: signal sample.}}
\end{figure}
\newpage
\begin{figure}[!h]
\begin{center}
\subfigure{\includegraphics[width = \textwidth]{varb3.png}}
\subfigure{\includegraphics[width = \textwidth]{varb4.png}}
\end{center}
\label{fig:varBDTa2}
\caption{\textit{Distributions for the different input variables for the \bdta training. Red distribution: background sample, blue distribution: signal sample.}}
\end{figure}

\newpage

\begin{figure}[!h]
\begin{center}
\subfigure{\includegraphics[width = \textwidth]{vara1.png}}
\subfigure{\includegraphics[width = \textwidth]{vara2.png}}
\end{center}
\label{fig:varBDTa}
\caption{\textit{Distributions for the different input variables for the \bdtb training. Red distribution: background sample, blue distribution: signal sample.}}
\end{figure}
\vspace*{5cm}
\begin{figure}[!h]
\begin{center}
\subfigure{\includegraphics[width = \textwidth]{vara3.png}}
\subfigure{\includegraphics[width = \textwidth]{vara4.png}}
\end{center}
\label{fig:varBDTa2}
\caption{\textit{Distributions for the different input variables for the \bdtb training. Red distribution: background sample, blue distribution: signal sample.}}
\end{figure}

\vspace*{5cm}

\section*{Distribution of S and B in the optimisation process}
\label{ap:SandB}

\begin{figure}[!h]
\begin{center}
\subfigure{\includegraphics[width = 0.7\textwidth]{BDT_L0Ele_S.pdf}}
\subfigure{\includegraphics[width = 0.7\textwidth]{BDT_L0Ele_B.pdf}}
\end{center}
\label{fig:SBEle}
\caption{\textit{Values for the expected signal yield (top) and number of background events in the signal window (bottom) for the L0Electron trigger categories for a fourth of the 2011 and 2012 \lhcb data.}}
\end{figure}

\begin{figure}[!h]
\begin{center}
\subfigure{\includegraphics[width = 0.7\textwidth]{BDT_L0Had_S.pdf}}
\subfigure{\includegraphics[width = 0.7\textwidth]{BDT_L0Had_B.pdf}}
\end{center}
\label{fig:SBHad}
\caption{\textit{Values for the expected signal yield (top) and number of background events in the signal window (bottom) for the L0Hadron trigger categories for a fourth of the 2011 and 2012 \lhcb data.}}
\end{figure}

\begin{figure}[!h]
\begin{center}
\subfigure{\includegraphics[width = 0.7\textwidth]{BDT_L0TIS_S.pdf}}
\subfigure{\includegraphics[width = 0.7\textwidth]{BDT_L0TIS_B.pdf}}
\end{center}
\label{fig:SBEle}
\caption{\textit{Values for the expected signal yield (top) and number of background events in the signal window (bottom) for the L0TIS trigger categories for a fourth of the 2011 and 2012 \lhcb data.}}
\end{figure}
%\include{2ThePierreAugerObservatory}
%\include{3TheReconstruction}
%\include{4ShowerFootprint}
%\include{5Accidentals}
%\include{6TheQualityofReconstruction}
%\include{7Conclusions}
\cleardoublepage
%% ++++++++++++++++++++++++++++++++++++++++++
%% Anhang
%% ++++++++++++++++++++++++++++++++++++++++++

\renewcommand{\appendixname}{Appendix}
\begin{appendix}
\chapter{\appendixname}

\section*{\BdKstee \textit{Stripping17b}}
\label{ap:strip17}
\begin{figure}[!h]
\begin{center}
\includegraphics[width = 0.7\textwidth]{stripping17.jpg}
\end{center}
\label{stripping17}
\caption{The cuts of the 2011 \BdKstee stripping-line.}
\end{figure}
\newpage

\section*{All BDT Input Variables}
\label{ap:allvars}
\begin{figure}[!h]
\begin{center}
\subfigure{\includegraphics[width = \textwidth]{varb1.png}}
\subfigure{\includegraphics[width = \textwidth]{varb2.png}}
\end{center}
\label{fig:varBDTa}
\caption{\textit{Distributions for the different input variables for the \bdta training. Red distribution: background sample, blue distribution: signal sample.}}
\end{figure}
\newpage
\begin{figure}[!h]
\begin{center}
\subfigure{\includegraphics[width = \textwidth]{varb3.png}}
\subfigure{\includegraphics[width = \textwidth]{varb4.png}}
\end{center}
\label{fig:varBDTa2}
\caption{\textit{Distributions for the different input variables for the \bdta training. Red distribution: background sample, blue distribution: signal sample.}}
\end{figure}

\newpage

\begin{figure}[!h]
\begin{center}
\subfigure{\includegraphics[width = \textwidth]{vara1.png}}
\subfigure{\includegraphics[width = \textwidth]{vara2.png}}
\end{center}
\label{fig:varBDTa}
\caption{\textit{Distributions for the different input variables for the \bdtb training. Red distribution: background sample, blue distribution: signal sample.}}
\end{figure}
\vspace*{5cm}
\begin{figure}[!h]
\begin{center}
\subfigure{\includegraphics[width = \textwidth]{vara3.png}}
\subfigure{\includegraphics[width = \textwidth]{vara4.png}}
\end{center}
\label{fig:varBDTa2}
\caption{\textit{Distributions for the different input variables for the \bdtb training. Red distribution: background sample, blue distribution: signal sample.}}
\end{figure}

\vspace*{5cm}

\section*{Distribution of S and B in the optimisation process}
\label{ap:SandB}

\begin{figure}[!h]
\begin{center}
\subfigure{\includegraphics[width = 0.7\textwidth]{BDT_L0Ele_S.pdf}}
\subfigure{\includegraphics[width = 0.7\textwidth]{BDT_L0Ele_B.pdf}}
\end{center}
\label{fig:SBEle}
\caption{\textit{Values for the expected signal yield (top) and number of background events in the signal window (bottom) for the L0Electron trigger categories for a fourth of the 2011 and 2012 \lhcb data.}}
\end{figure}

\begin{figure}[!h]
\begin{center}
\subfigure{\includegraphics[width = 0.7\textwidth]{BDT_L0Had_S.pdf}}
\subfigure{\includegraphics[width = 0.7\textwidth]{BDT_L0Had_B.pdf}}
\end{center}
\label{fig:SBHad}
\caption{\textit{Values for the expected signal yield (top) and number of background events in the signal window (bottom) for the L0Hadron trigger categories for a fourth of the 2011 and 2012 \lhcb data.}}
\end{figure}

\begin{figure}[!h]
\begin{center}
\subfigure{\includegraphics[width = 0.7\textwidth]{BDT_L0TIS_S.pdf}}
\subfigure{\includegraphics[width = 0.7\textwidth]{BDT_L0TIS_B.pdf}}
\end{center}
\label{fig:SBEle}
\caption{\textit{Values for the expected signal yield (top) and number of background events in the signal window (bottom) for the L0TIS trigger categories for a fourth of the 2011 and 2012 \lhcb data.}}
\end{figure} 
\end{appendix}
% \addcontentsline{toc}{chapter}{Appendix}
%\include{anhang_b}
\cleardoublepage
%% ++++++++++++++++++++++++++++++++++++++++++
%% Literatur
%% ++++++++++++++++++++++++++++++++++++++++++
%  mit dem Befeh\\
%  zitierte Referenzen abgedruckt
%\nocite{*}
%\renewcommand{\bibname}{References}
\bibliographystyle{unsrt}  %siamprsty-mod
% hier den passenden Pfad ergaenzen, es kann ein neuer Style gewählt werden, dazu muss nach Styles ein neuer Styler runtergeladen werden. Dieser Style benutzt organisation um XY et. al Pierre Auger Colloboration gut darzustellen
\bibliography{bib/vorlage}
\addcontentsline{toc}{chapter}{\bibname}

%% ++++++++++++++++++++++++++++++++++++++++++
%% Index
%% ++++++++++++++++++++++++++++++++++++++++++
% \ifnotdraft{
% \addcontentsline{toc}{chapter}{Index}
% \printindex            % Index, Stichwortverzeichnis
% }

\selectlanguage{ngerman}
\chapter*{Acknowledgements}
\chaptermark{Acknowledgements}
\addcontentsline{toc}{chapter}{Acknowledgements}
At first I want to thank Achille Stocchi for giving me the opportunity of doing my master's thesis at 'his' LAL, especially since there is no such thing as a master's thesis in particle physics in France. I also want to thank you for allowing me to participate in the NPAC masters where I applied much to late during my ERASMUS year in Paris. It was the participation in this program that permitted me to meet the \lhcb group. And I want to thank you for persuading Marie-H\'{e}l\`{e}ne Schune to take me as an intern even though I had set to my mind to follow the German master's system.\\
\\
I want to thank my supervisor Marie-H\'{e}l\`{e}ne Schune \textbf{very very much} for teaching more than I thought was possible and for showing me this incredibly amazing world of of ideas and scientific forward-thinking that is particle physics. I will be eternally grateful for all the effort that was put into my training and all the mistakes I made that were immediately forgiven, even if at least one of them almost broke both our hearts. Thank you for giving me confidence in myself and my capabilities and for being so supportive even after I decided to leave. And of course thank you for supplying me with sweets.\\
I also want to thank Jacques Lefrancois for the long discussions about bremsstrahlung, polarisation and photon conversion. I have learned a lot, especially that experimental physicists also can have a descent amount of theoretical knowledge. And I also enjoyed these discussions and enjoyed seeing all the connections and how to derive them. I am stilled awed by the amount of knowledge you posses.\\
Before I continue with thanking individual people, I want to thank the entire \lhcb group at LAL. Everybody has been very kind and welcoming right from the start and while this might seem natural to some I know that that's not the way it is done everywhere. Thank you for being so very willing to help and for including everyone into the group right from the start.\\
\\
Next I want to thank Alexandra for being my office mate and a very good help when it comes to making beautiful plots in \roofit. I enjoyed our little singing sessions and our tee-times very much. Thank you for calling me ''little bunny'', you are truly a bunny yourself.
It was very cushy to have someone going trough the (masters) thesis craziness with me and sharing some nice chocolate in time of stress. I wish you that you'll find some awesome jobs that also makes a lot of money and then you can buy some bunnies and think of me. \\
I want to thank Alexis for always being somehow cheerful and for making me say very inappropriate words during lunch at CESFO. I will definitely call you if somebody in Bristol is mean to me!\\
I want to thank Frederic for his frog-story during \lhcb week in Davos that I will never ever forget.\\
I want to thank Benoit for his countless stories that were often told so fast that I had to extrapolate between bits that I understood. I hope I will not spread false stories about what you said.\\
Of course I also want to thank Chris for all the food he provided while I resided in Aachen. Very delicious indeed! And thanks also for the moral support during the entire thesis.\\
Tine! Thanks for suffering with me and even more! And thanks for watching and sending me pictures of baby animals when nothing else was working. Also thanks a lot for being the link between me and the Aachen-world. I am pretty sure many things -- like my final presentation -- would not have worked without you. Thanks for not letting me hang out to dry in France but for giving me the all the intel, you are my favourite spy! I always have our plan-b in mind and that reassures me, that even if all fails I can still live on a bunny-pony farm somewhere where it's warm and cosy.\\
I also want to thank Djalel for the endless discussions about \bdts, philosophy, psychology and sociology. They have been very enlightening and I enjoyed them a great deal. I will truly miss them.\\
Frank, thank you for reading over this master's thesis even though it is not your area of expertise at all. Your comments have been very helpful.\\
I also want to thank Jibo for his immense help with all \davinci related topics even after he left the group. I don't know what we would have done  without you! Somehow the 10 minutes rule has been burned into my brain and I will apply it all my life.\\
Of course I want to thank Lo and Heiner for answering my countless questions about CMS. You were incredibly helpful and also fun of course.\\
Martino, now it's up to you to make an amazing angular analysis. I am counting on you! Stay funny and panic once in a while since this somehow seems to calm me down.\\
Michelle! Thanks a lot for the electrons and for sharing some of my dislikes. This has cheered me up and given me back faith into my mental sanity. I have enjoyed our pizza and wine evenings and while I am sitting here all alone in the middle of the night I wish somebody would bring me a creme fresh champignon pizza! and just in case Bristol does not work out I hope you still have some space on your roof terrace.\\
Thank you Patrick, for being a very quite co-buero and for tolerating all my unusual eating and working habits. I have enjoyed all your posters in the office and I will hang some in my new office.\\
Great thanks goes to Paul who read my master's thesis more meticulously than everybody else. Your scanning skills must be amazing. I am also very happy about your helpful comments, you truly deserve a big cake.\\
I want to thank Philipp 'The Panda' for the occasional passing on of letters and bills that should have been paid months ago. Your authenticity is inspiring and you are indeed my favourite panda.\\
To Thiemo just as he wished for (even though he could have gotten something better):Thank you!\\
Thanks to Sam and Andrew for their help with English expressions even if I was surprised about how difficult that seemed. \\
Sergey, thanks for the Ukrainian chocolate things that were so sweet they made my mouth burn. And thank you for telling me to go home once in a while even if I didn't listen.\\
Thanks Vagelis for not one but even two afternoon coffees and for always staying positive even if I wanted to see only the negative.\\
\\
Actually I want to thank everybody who contributed in brining me where I am right now, this master's thesis would not have been possible without you. Starting with my parents that taught me to be tough and to work hard and my grand-parents who told me that sometimes being soft is ok too, to my physics teacher in school that believed in me and told me that I was energetic enough to study physics with (or against) the male competition. Of course everything would not have been possible without all my friends who made studying physics not only possible but also fun. Thank every single one of you for all the talks and the fun we had.\\
\\
Overall I am thankful for everybody I met during my master's thesis and everything that I could learn -- be it professional or private. I wish you all the super duper best and I will miss you. See you at \cern or in Bristol or in Paris or in any other place of the world!
%\chaptermark{Acknowledgements}
%\addcontentsline{toc}{chapter}{Acknowledgements}
%Danke Stefan und Anna f\"ur die Vorlage ....  

\chapter*{Erkl\"arung}
Hiermit versichere ich, dass ich diese Arbeit einschlie\ss lich beigef\"ugter Zeichnun\-gen, Darstellungen und Tabellen selbst\-st\"andig an\-gefertigt und keine anderen als die ange\-gebenen Hilfsmittel und Quellen verwendet habe. Alle Stellen, die dem Wortlaut oder dem Sinn nach anderen Werken entnommen sind, habe ich in jedem einzelnen Fall unter genauer Angabe der Quelle deutlich als Entlehnung kenntlich gemacht.
 \vspace{1.5cm}
  \flushright Paris, den \today
  \vskip 3cm
  \flushright 
$\overline{\qquad \qquad \qquad \mbox{Claire Prouve}}$
\blankpage

\end{document}
%% end of file
