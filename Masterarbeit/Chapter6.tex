\chapter{Conclusion}
\chaptermark{Chapter6}
\label{chapter6}
The \bsg transition proceeds through a flavour changing neutral current and thus is particularly sensitive to the effects of new physics. These effects could be detectable in details of the decay such as the polarisation of the photon which is predicted by the SM to be predominantly left-handed. Certain extensions of the SM such as the Left-Right Symmetric Model (LRSM) can provide a significant right-handed photon amplitude. The photon amplitudes can be measured by performing an angular analysis of the \BdKstee decays at low invariant dilepton mass.\\
\\
The work in this master's thesis is the first step towards this angular analysis.\\
The comparison between the different reconstruction algorithms for the bremsstrahlung emitted by the electrons shows that the new reconstruction algorithm increases the resolution of the \Bd mass by 13\%. By implementing the tool specially dedicated to reconstructing low dilepton mass pairs the \Bd mass resolution can be augmented by a total of 27\%. Furthermore this tools shows an increased reconstruction efficiency for events with very low invariant dilepton mass which are the ones most important for the angular analysis. Thanks to the new reconstruction algorithms the lower cut on the invariant dilepton mass could be dropped from 30\mevcc in the 2011 analysis to 20\mevcc while keeping a high precision on the angles $\Phi$, $\theta_l$ and $\theta_K$.\\
The new stripping line used in the preselection of the \BdKstee candidates in the 2011 and 2012 \lhcb data features a selection efficiency which is increased by a factor 1.4 with respect to the previous stripping line. Furthermore the selection procedure developed in this work implementing two \bdts yields the largest sample of \BdKstee events ever selected. Summed over the three independent trigger categories for which the analysis was performed the signal yield is 130 $\pm$ 17 events which corresponds to a statistical signal significance of 8.3 standard deviations. This signal yield is 40\% more than the yield that would have been obtained by applying the selection developed for the 2011 analysis.\\
\\
Continuing from this dataset of \BdKstee candidates the angular analysis will be performed. Monte Carlo toy studies have shown that an accuracy on the parameter $A^{(2)}_T$ using 130 \BdKstee events and assuming no contribution from background events of $\sigma(A^{(2)}_T)\approx 0.2$ can be achieved.\\