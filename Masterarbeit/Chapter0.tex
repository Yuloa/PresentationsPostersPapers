\chapter{Introduction}
\chaptermark{Chapter1}
\label{Chapter0}
The Standard Model (SM) of particle physics is a very successful theory in that it explains a vast range of phenomena observed in nature. Its predictive powers exceed those of any previous theory and its mathematical structure is well defined and organised. The SM describes the fundamental aspects of matter and energy by expressing matter particles (fermions) and particles that mediate interactions (bosons) in a unified theory. The latest achievement of particle physics was the discovery of the Higgs Boson, whose existence had been predicted over forty years ago as inherent part of the SM, by \cms \cite{higgscms} and \atlas \cite{higgsatlas}.\\ 
Despite these achievements there are still questions unanswered by the SM. One open issue is the inclusion of gravity and the explanation of the hierarchy of the fundamental interactions. Another unexplained phenomenon is the discrepancy between the amount of matter and anti-matter observed in the universe. While the SM does implement the mechanism of charge-parity (CP) violation, the generated effects within the SM are not large enough to explain today's observations.\\
For this reason, many other theories -- such as the supersymmetry (SUSY) or the Left-Right Symmetric Model (LRSM) -- have been developed as extensions of the SM. These theories are designed so that their predictions agree with the SM at low energies but differ at higher energies. Usually new particles are introduced with heavier masses, changing the physics at high energy scales with respect to the SM.\\
\\
All high energy physics experiments aim at performing precision measurements of SM parameters or at discovering physics beyond the SM. The \lhc has introduced an optimal environment with an unprecedented center of mass energy and a high luminosity that provides the allocated experiments with a high amount of statistics. The experiments \atlas and \cms \ -- the so-called 'general purpose' detectors -- mainly aim at finding new particles through direct searches. The \lhcb experiment however, performs indirect searches for new physics, for example by identifying deviations from SM predictions in loop-diagram and box-diagram processes. The \lhcb experiment focusses on rare decays of \B and \D mesons and hadrons and performs measurements of CP violation.\\
The \BdKstee decay is of special interested, because it proceeds through a flavour changing neutral current (FCNC) and is therefore forbidden at tree-level in the SM. It has to proceed through loop diagrams and thus is particularly sensitive to new physics and particularly useful for testing the quantum structure of the underlying theory. Since relative contributions from new physics are greater for these processes than for tree-level processes they are more likely to be detectable through thorough analysis.\\
The main contribution to the \BdKstee comes from the transition of \bsg where the photon is virtual and decays into an electron-positron pair. While the branching ratio for this transition has been found to be consistent with SM predictions (measurement via \BdKstg \cite{pdg}), new physics could still be detectable in details of the decay such as the photon polarisation. Due to the chirality structure of the weak interaction, the photons from the \bsg transitions are dominantly left-handed. A measurement of a significant amount of right handed photons from this transitions would be an unambiguous sign of new physics beyond the SM.
The photon polarisation can be probed by performing an angular analysis of the \BdKstee decay. In this case, the photon from the \bsg is virtual and the information about its polarisation is conserved in the kinematics of the electron-positron pair.\\
\\
This master's thesis provides the fundamental basis for an angular analysis of the \BdKstee decay. In the course of this work the reconstruction of the\\ \BdKstee candidates is studied and different reconstruction algorithms are evaluated. Hereby the focus is put on the reconstruction of bremsstrahlung emitted by the electrons and the reconstruction of the invariant mass of the electron-positron pair. As a result the resolution on the \Bd mass is improved by about 27\% and the reconstruction efficiency is augmented in the region of low electron positron invariant mass.\\
Furthermore a new selection procedure for the \BdKstee events in 3\invfb of data collected by \lhcb in 2011 and 2012 is developed. Therefore a method implementing two identical Boosted Decision Trees trained on different parts of the dataset is applied. The selection of the \BdKstee events is optimised and the number of reconstructed and selected \BdKstee events is extracted by fitting a probability density function -- composed of a probability density function for the signal and the background respectively -- to the dataset. This results in a total signal yield of 130 $\pm$ 17 \BdKstee events.\\